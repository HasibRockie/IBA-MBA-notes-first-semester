\documentclass[12pt,a4paper]{book}

% Fonts & Typography — Elegant and Professional
\usepackage[T1]{fontenc}
\usepackage{kpfonts} % Sleek, modern font
\usepackage{microtype}

% Essential Packages
\usepackage{graphicx}
\usepackage{fancyhdr}
\usepackage{tocloft}
\usepackage{titlesec}
\usepackage{datetime}
\usepackage{hyperref}
\usepackage{geometry}
\usepackage{parskip}

% Page Geometry — Slim, Clean Margins
\geometry{
  a4paper,
  left=20mm,
  right=20mm,
  top=20mm,
  bottom=20mm
}

% Header & Footer Styling
\pagestyle{fancy}
\fancyhf{}
\fancyhead[L]{\small \textit{\nouppercase{\leftmark}}}
\fancyhead[R]{\small L501: Legal Environment of Business}
\fancyfoot[C]{\small \thepage}
\renewcommand{\headrulewidth}{0.3pt}
\renewcommand{\footrulewidth}{0.3pt}

% Chapter Title Styling
\titleformat{\chapter}[block]
  {\normalfont\Huge\bfseries}
  {\thechapter.}{12pt}{}

\titleformat{\section}
  {\normalfont\Large\bfseries}
  {\thesection}{1em}{}

% Table of Contents Styling
\renewcommand{\cftchapfont}{\bfseries}
\renewcommand{\cftsecfont}{}
\setlength{\cftbeforechapskip}{5pt}
\setlength{\cftbeforesecskip}{2pt}
\setlength{\cftaftertoctitleskip}{1em}

% Hyperlink Styling
\hypersetup{
    colorlinks=true,
    linkcolor=blue,
    urlcolor=blue,
    pdftitle={L501: Legal Environment of Business},
    pdfauthor={Md Hasibul Islam},
    pdfpagemode=FullScreen,
}

% Custom Command for Notes 
\newcommand{\notesection}[2]{
  \section*{#1\\ \small \textit{#2}}
  \phantomsection
  \addcontentsline{toc}{section}{#1 - #2}
}

% Document Start 
\begin{document}

% Title Page
\begin{titlepage}
    \centering
    \vspace*{3.5cm}
    \includegraphics[width=0.28\textwidth]{logo.png}\par\vspace{1.5cm}
    {\scshape\LARGE University of Dhaka\par}
    \vspace{0.5cm}
    {\Large Institute of Business Administration (IBA)\par}
    \vspace{1.5cm}
    {\Huge\bfseries Master of Business Administration (MBA)\par}
    \vspace{1cm}
    {\Large W501: \textit{Legal Environment of Business}\par}
    \vfill
    {\large Last Updated: \today\par}
\end{titlepage}

% Author Details Section 
\section*{Author Details}
\phantomsection
\addcontentsline{toc}{section}{Author Details}

\begin{center}
    \vspace{1em}
    \begin{tabular}{lll}
        \textbf{Name} & : & Md Hasibul Islam \\
        \textbf{Student ID} & : & 201-67-011 \\
        \textbf{Program} & : & Master of Business Administration (MBA) \\
        \textbf{Institute} & : & Institute of Business Administration (IBA) \\
        \textbf{University} & : & University of Dhaka \\
        \textbf{Email} & : & \href{mailto:hasiee8004@gmail.com}{hasiee8004@gmail.com} \\
        \textbf{LinkedIn} & : & \href{https://www.linkedin.com/in/hasib009}{linkedin.com/in/hasib009} \\
        \textbf{GitHub} & : & \href{https://github.com/HasibRockie}{github.com/HasibRockie} \\
        \textbf{Website} & : & \href{https://hasibrockie.github.io}{hasibrockie.github.io} \\
    \end{tabular}
    \vspace{1em}
\end{center}

\clearpage

% Table of Contents
\tableofcontents
\clearpage

% Notes Sections
\notesection{Proposal, Offer, Promise, Acceptance – Definitions, Differences \& Revocation}{21-04-25 Monday}

\textbf{1. Proposal (Offer):}  
According to Section 2(a) of the Indian Contract Act, 1872,  
\textit{“When one person signifies to another his willingness to do or to abstain from doing anything, with a view to obtaining the assent of that other, he is said to make a proposal.”}  

In business law, "proposal" and "offer" are often used interchangeably. A valid proposal must be:
\begin{itemize}
    \item Communicated to the offeree.
    \item Made with intent to create legal relations.
    \item Definite and certain in terms.
\end{itemize}

\textbf{2. Promise:}  
As per Section 2(b), a proposal when accepted becomes a promise.  
\textit{“When the person to whom the proposal is made signifies his assent thereto, the proposal is said to be accepted. A proposal when accepted becomes a promise.”}  

Thus, \textbf{Promise = Proposal + Acceptance}

\textbf{3. Acceptance:}  
Acceptance must be:
\begin{itemize}
    \item Absolute and unqualified.
    \item Communicated by authorized means.
    \item Given within a reasonable time or stipulated time.
\end{itemize}

\textbf{4. Differences among Proposal, Offer, Promise \& Acceptance:}

\begin{center}
\begin{tabular}{|p{4cm}|p{6cm}|p{6cm}|}
\hline
\textbf{Term} & \textbf{Definition} & \textbf{Key Feature} \\
\hline
Proposal / Offer & Intention to form contract by doing/refraining from act. & Requires acceptance to be binding. \\
\hline
Promise & Accepted offer. & Legally binding. \\
\hline
Acceptance & Assent to the proposal. & Converts proposal into a contract. \\
\hline
\end{tabular}
\end{center}

\textbf{5. Revocation of Proposal and Acceptance:}  
\begin{itemize}
    \item \textbf{Section 5 of the Indian Contract Act:} A proposal may be revoked at any time before the communication of its acceptance is complete as against the proposer.
    \item An acceptance may be revoked at any time before the communication of the acceptance is complete as against the acceptor.
\end{itemize}

\textbf{Examples (Proposal or Not?):}  
Below are 20+ situations analyzed to determine if they constitute a proposal:

\begin{enumerate}
    \item A says, "I will sell my bike to you for \$500" – \textbf{Proposal}.
    \item An advertisement for a sale on TV – \textbf{Not a proposal}, it's an \textit{invitation to offer}.
    \item A menu card in a restaurant – \textbf{Not a proposal}.
    \item Display of goods in a shop with a price tag – \textbf{Not a proposal}.
    \item A says, "Will you buy my car for \$5,000?" – \textbf{Proposal}.
    \item A tells B, “I may consider selling my house” – \textbf{Not a proposal}.
    \item Public auction notice – \textbf{Invitation to offer}.
    \item A quote or tender submission – \textbf{Proposal}.
    \item B submits a bid in auction – \textbf{Proposal}.
    \item A company issues a prospectus – \textbf{Invitation to offer}.
    \item A person asks, “Can you lend me your book?” – \textbf{Not a legal proposal}.
    \item X offers Y a job with fixed terms – \textbf{Proposal}.
    \item A general statement: “I wish someone would buy my house” – \textbf{Not a proposal}.
    \item A mails a signed letter offering to sell goods – \textbf{Proposal}.
    \item A unilateral offer of reward for finding a lost pet – \textbf{Proposal}.
    \item Online listing of a laptop on eBay – \textbf{Proposal} (depending on platform rules).
    \item Negotiation discussion: “I might accept \$1000 for this item” – \textbf{Not a proposal}.
    \item A says, “If you pay today, I will give a discount” – \textbf{Proposal}.
    \item A invitation to tender (e.g., govt contract) – \textbf{Invitation to offer}.
    \item B’s submission of the tender form – \textbf{Proposal}.
    \item A says “Let’s form a partnership” – \textbf{Proposal}.
    \item A makes a standing offer to supply items – \textbf{Proposal}.
\end{enumerate}

\textbf{Note:} The difference between \textit{“invitation to offer”} and an actual \textbf{proposal} is crucial. An invitation invites others to make offers, whereas a proposal is the starting point of contractual intent.

\textbf{Conclusion:}  
Understanding the conceptual differences between proposal, promise, and acceptance is crucial in legal and business transactions. Clear communication, intent to be bound, and adherence to legal formalities determine the enforceability of contracts. Revocation rights add further complexity and protection for both parties involved.



\textbf{Definition of Contract:}  
According to Section 2(h) of the Indian Contract Act, 1872:  
\textit{“A contract is an agreement enforceable by law.”}  
Thus, for any agreement to be a contract, it must fulfill these essentials:
\begin{itemize}
    \item Lawful offer and lawful acceptance.
    \item Intention to create legal relationship.
    \item Lawful consideration.
    \item Capacity of parties.
    \item Free consent.
    \item Lawful object.
\end{itemize}

\vspace{0.4cm}
\textbf{Below are 20+ examples examined to determine if they constitute contracts:}

\begin{enumerate}
    \item A agrees to sell B his house for 50 lakhs. – \textbf{Contract} (meets all essentials).
    \item A invites B for dinner. – \textbf{Not a Contract} (no legal intention).
    \item A promises to give B 1 lakh out of love. – \textbf{Not a Contract} (no consideration).
    \item B offers A 10,000 for his phone, A accepts. – \textbf{Contract}.
    \item X and Y agree to smuggle goods across borders. – \textbf{Not a Contract} (illegal object).
    \item A minor agrees to buy a laptop on EMI. – \textbf{Not a Contract} (incapacity of minor).
    \item A promises B to marry her without any intention of doing so. – \textbf{Not a Contract} (no real intention).
    \item A agrees to supply 100 chairs to B at 500 per chair. – \textbf{Contract}.
    \item A enters into agreement under threat. – \textbf{Not a Contract} (absence of free consent).
    \item A wins a lottery, government promises to pay. – \textbf{Contract} (in case of valid government scheme).
    \item A advertises a reward for lost pet, B finds it and claims reward. – \textbf{Contract}.
    \item A agrees to work for B without salary. – \textbf{Not a Contract} (no consideration unless a gift of service is intended and accepted).
    \item A contracts B to kill a person. – \textbf{Not a Contract} (unlawful object).
    \item Verbal agreement to sell a cow for 20,000, accepted and paid. – \textbf{Contract}.
    \item Casual promise at a party to lend money. – \textbf{Not a Contract} (no legal intention).
    \item A enters into contract with a person of unsound mind. – \textbf{Not a Contract}.
    \item A hires B to paint his house for 15,000. – \textbf{Contract}.
    \item A and B mutually agree to form a business with capital investment. – \textbf{Contract}.
    \item A bets 1000 on a cricket match with B. – \textbf{Not a Contract} (void under wagering agreements).
    \item A contracts with B to deliver goods, consideration given. – \textbf{Contract}.
    \item A bribes an officer to pass a tender. – \textbf{Not a Contract} (against public policy).
    \item A writes a letter offering services to B, B accepts. – \textbf{Contract}.
    \item A sells his bicycle to B for 3,000, B pays instantly. – \textbf{Contract}.
    \item A and B agree to commit cyber fraud. – \textbf{Not a Contract} (unlawful purpose).
    \item A sells a book to B with mutual consent and price. – \textbf{Contract}.
\end{enumerate} 

\textbf{Key Insights:}
\begin{itemize}
    \item A \textbf{social or moral agreement} is not enforceable (e.g., dinner invitations).
    \item \textbf{Illegal or immoral agreements} are void ab initio (from the beginning).
    \item \textbf{Contracts with minors or mentally unsound persons} are generally void.
    \item \textbf{Consideration} and \textbf{legal object} are core components.
\end{itemize}

\textbf{Conclusion:}  
All contracts are agreements, but not all agreements are contracts. The key lies in legal enforceability. Every case must be assessed against the core criteria of a valid contract to determine its legitimacy and enforceability under law.

\vspace{1cm} 

\textbf{Chapter 2: Offer and Acceptance Exercise Problems} 
\vspace{0.5cm} 

\textbf{1. When is an offer completed? How and when may an offer be revoked?}  

An offer is completed when it is communicated to the offeree (Sec. 4, Indian Contract Act).  
\textbf{Revocation of an offer} (Sec. 5) is valid if:
\begin{itemize}
    \item It is done before the communication of acceptance is complete against the proposer.
    \item Revocation must be communicated to the offeree.
\end{itemize}

\textit{Example:} A offers to sell a bike to B. If B posts an acceptance letter and A sends a revocation after that, the revocation is invalid.

\vspace{0.5cm}
\textbf{2. (a) How may an offer be terminated?}

An offer can be terminated in the following ways:
\begin{itemize}
    \item Revocation before acceptance.
    \item Rejection by offeree.
    \item Lapse of time.
    \item Death or insanity of offeror.
    \item Non-fulfillment of a condition precedent.
\end{itemize}

\textbf{(b)} A offers to sell B his horse for one thousand and tells B, "This offer will remain open for one week." B rejects the offer the next day. Later in the week, B changes his mind and accepts the offer.  
\textbf{No contract is formed.} Once the offer is rejected, it cannot be revived unless A makes the offer again.

\vspace{0.5cm}
\textbf{3. "Acceptance is to offer what a lighted match is to a train of gunpowder." Discuss.}  

This metaphor, coined by Anson, means that just as a lighted match completes the chain for an explosion, acceptance finalizes an offer into a contract. There’s no turning back once acceptance is made and communicated – the contract is formed.

\vspace{0.5cm}
\textbf{4. "An offer is made when and not until, it is communicated to the offeree".}  

Communication is key to a valid offer.  
\textit{Example:} A writes a letter offering B his car but never posts it. There is no offer.  
Unless the offeree is made aware, an offer cannot be accepted.

\vspace{0.5cm}
\textbf{5. Define offer and acceptance. When are they complete if made through post?}  

\textbf{Offer:} Sec. 2(a) – Willingness to do or abstain from an act to obtain assent.  
\textbf{Acceptance:} Sec. 2(b) – When the offeree signifies assent.

\textbf{Completion by post:}  
\begin{itemize}
    \item \textbf{Offer} is complete when received by offeree.
    \item \textbf{Acceptance} is complete as against proposer when it is posted, and against acceptor when received.
\end{itemize}

\vspace{0.5cm}
\textbf{6. Rules of Offer via Post and Telephone:}  

\textbf{By Post:}
\begin{itemize}
    \item Acceptance is complete when posted.
    \item Revocation of offer must reach before acceptance is posted.
\end{itemize}

\textbf{By Telephone:}
\begin{itemize}
    \item Acceptance must be clearly heard and understood.
    \item No contract if communication fails or is unclear.
\end{itemize}

\vspace{0.5cm}
\textbf{7. "A mere mental acceptance, not evidenced by words or conduct is in the eye of law no acceptance."}  

True. Acceptance must be communicated. Thinking of accepting is not sufficient.  
\textit{Example:} A decides to accept an offer but doesn’t inform B. No contract arises.

\vspace{0.5cm}
\textbf{8. Define 'Acceptance'. Essentials of Valid Acceptance:}  

Sec. 2(b): Acceptance is the act of signifying assent.  
\textbf{Essentials:}
\begin{itemize}
    \item Must be absolute and unconditional.
    \item Must be communicated.
    \item Must be in the prescribed mode.
    \item Must be made while the offer is still open.
\end{itemize}

\vspace{0.5cm}
\textbf{9. "Acceptance must be absolute, and must correspond with the terms of the offer."}  

This is the \textbf{mirror image rule}.  
\textit{Example:} A offers to sell a product for ten thousand. B replies, “I accept for nine thousand.” → Counter-offer, not acceptance.

\vspace{0.5cm}
\textbf{10. (a) Meaning of Offer and Acceptance:}  

Already defined above in Q5.

\textbf{(b)(i)} A offers to sell goods by letter on March 1. B receives it on March 3.  
A can revoke before B posts acceptance. So yes, A can revoke before March 4.

\textbf{(ii)} B posts acceptance on March 4. A receives it on March 6.  
B cannot revoke after posting. So, \textbf{No}, B cannot revoke acceptance.

\vspace{0.5cm}
\textbf{11. (a) Define Proposal:}  
As per Sec. 2(a), willingness to do or abstain from an act with a view to obtaining assent.

\textbf{(b) Communication of Offer:}
\begin{itemize}
    \item Must reach the offeree.
    \item Must be clear and specific.
\end{itemize}

\vspace{0.5cm}
\textbf{12. Objective Questions:}

\textbf{(a)} Acceptance by conduct: Acceptance shown by action.  
\textit{Example:} Taking a train ticket implies acceptance of the travel terms.

\textbf{(b)} An advertisement to sell a thing by auction is:  
\textbf{Answer: (b) An invitation to offer.}

\vspace{0.5cm}
\textbf{13. Problems:}

\textbf{(a)} A proposes via letter. B accepts by letter.  
\textbf{A can revoke} before B posts acceptance.  
\textbf{B can revoke} before A receives acceptance.

\textbf{(b)} X’s revocation reaches Y before the offer does.  
\textbf{No valid offer} reached Y, hence no contract.

\textbf{(c)} B can revoke acceptance before it reaches A.  
So, if revocation is communicated before delivery, it is valid.

\textbf{(d)} A offers a reward for doing a task. B does the task without knowledge of the reward.  
\textbf{No contract.} Knowledge of offer is essential.

\textbf{(e)} A posts acceptance, but it's lost.  
\textbf{Contract is still valid.} Acceptance is complete upon posting.

\vspace{1cm}
\clearpage


\notesection{Consideration}{24-04-25 Thursday}

\textbf{Definition of Consideration:}  
According to Section 2(d) of the Indian Contract Act,  
\textit{“When, at the desire of the promisor, the promisee or any other person has done or abstained from doing, or does or abstains from doing, or promises to do or abstain from doing something, such act or abstinence or promise is called a consideration for the promise.”}

\textbf{In Simple Words:}  
Consideration is something of value given by both parties to a contract that induces them to enter into the agreement. It may be money, a service, a promise, or even not doing something.

\vspace{0.3cm}
\textbf{Key Points:}
\begin{itemize}
    \item It must move at the desire of the promisor.
    \item It may move from the promisee or any other person.
    \item It may be past, present or future.
    \item It must be something of value in the eyes of law.
\end{itemize}

\vspace{0.5cm}
\textbf{Rules of Consideration:}
\begin{enumerate}
    \item Consideration must move at the desire of the promisor.
    \item It may move from the promisee or any third party.
    \item It may be past, present or future.
    \item It must be real and not illusionary.
    \item It need not be adequate, but must be lawful.
    \item Consideration must be something that the law can regard as valuable.
\end{enumerate}

\vspace{0.5cm}
\textbf{Exceptions to the Rule: “No Consideration, No Contract” (Sec. 25)}  
Normally, a contract without consideration is void. But there are exceptions:
\begin{itemize}
    \item \textbf{Natural love and affection:} Between close relations if made in writing and registered.
    \item \textbf{Compensation for past voluntary service:} If a person voluntarily does something and the other person promises to pay later.
    \item \textbf{Promise to pay a time-barred debt:} If made in writing and signed by the debtor.
    \item \textbf{Completed gifts:} Gifts already made are valid even without consideration.
    \item \textbf{Agency agreements:} No consideration is necessary to create an agency.
    \item \textbf{Contract under seal:} In English law, consideration is not needed for contracts made under seal.
\end{itemize}

\vspace{0.5cm}
\textbf{Examples of Consideration (15+):}
\begin{enumerate}
    \item A agrees to sell a house and B agrees to pay a price – mutual consideration.
    \item A promises to deliver goods and B promises to pay on delivery.
    \item A serves B’s company without pay, later B promises to pay – past consideration.
    \item A gives tuition to B’s son, B promises to pay later.
    \item A gives up smoking at B’s request, B promises to pay.
    \item A does not file a legal case against B, B agrees to pay.
    \item A lends B a book, B promises to return or replace it.
    \item A promises to pay for goods received in the past.
    \item A cleans B’s garden for free; B later promises a gift – voluntary act.
    \item A supports B during illness, B later promises payment.
    \item A marries with B’s daughter; B promises a house – valid if written and registered.
    \item A saves B from drowning, B promises a reward – past voluntary service.
    \item A owes B money; B forgives the debt if A transfers a property – consideration.
    \item A agrees to give B a job, B agrees to pay fees – reciprocal promises.
    \item A agrees not to open a competing shop near B, B pays for that promise.
    \item A provides free legal advice, B promises to gift a car – no consideration.
    \item A promises to donate to a school, does not pay – generally unenforceable.
\end{enumerate}

\vspace{0.5cm}
\textbf{Stranger to Consideration vs. Stranger to Contract}

\textbf{Stranger to consideration:} A person who has not provided consideration can still enforce the contract, if done on behalf of someone.  
\textbf{Stranger to contract:} A person who is not a party to the contract cannot sue upon it.

\vspace{0.3cm}
\textbf{Exceptions to Stranger to Contract Rule:}
\begin{itemize}
    \item \textbf{Trust:} Beneficiaries of trust can enforce the contract.
    \item \textbf{Family settlements:} Members receiving benefit can sue.
    \item \textbf{Agency:} Principal can sue on agent’s contract.
    \item \textbf{Assignment:} Assignee can sue even if not an original party.
    \item \textbf{Covenants running with land:} Successors can enforce.
\end{itemize}

\vspace{0.3cm}
\textbf{Examples of Stranger to Contract and Exceptions:}
\begin{enumerate}
    \item A makes a contract with B to pay C – C cannot sue (general rule).
    \item A (father) agrees to pay C (daughter-in-law) under a family arrangement – C can sue.
    \item A contracts with B to pay debt to C – if C is a trust beneficiary, C can sue.
    \item A promises to pay B on behalf of C. If C is the principal, he can sue via agent rule.
    \item A assigns his rights in a contract with B to C – C can sue B.
    \item Landlord promises tenant to repair property, tenant’s sub-tenant may not sue directly.
\end{enumerate}

\vspace{0.5cm}

\textbf{1. Define consideration. Critically discuss the essential elements of consideration.}

\textbf{Definition:}  
Section 2(d) of the Indian Contract Act defines consideration as:  
\textit{"When, at the desire of the promisor, the promisee or any other person has done or abstained from doing... something, such act... is called a consideration for the promise."}

\textbf{Essential Elements:}
\begin{enumerate}
    \item \textbf{Desire of the promisor:} The act or forbearance must be at the promisor’s request.
    \item \textbf{From promisee or any other person:} Indian law allows a stranger to consideration.
    \item \textbf{Past, present or future:} Indian law recognizes all three types.
    \item \textbf{Lawful consideration:} It must not be illegal, immoral or opposed to public policy.
    \item \textbf{Something of value:} Even if not adequate, it must be real and tangible in law.
\end{enumerate}

\vspace{0.3cm}
\textbf{Critically:} Indian law is more flexible than English law in accepting past and third-party consideration.

\vspace{0.5cm}
\textbf{2. "Past consideration is no consideration." – Comment.}

\textbf{English Law:} Past consideration is generally invalid. A promise to reward a past act is unenforceable unless a prior request existed.

\textbf{Indian Law:} Past consideration is valid. If someone does something voluntarily or at the promisor’s request and the promisor later promises to pay, it is enforceable.

\textbf{Example:} A saves B’s goods from fire. Later B promises to pay A. In India, this is a valid contract. In English law, it may not be.

\vspace{0.5cm}
\textbf{3. Define consideration and point out the differences between English law and Indian law in this respect.}

\textbf{Definition:} (Same as Q1)

\textbf{Key Differences:}\\
\begin{tabular}{|p{0.3\linewidth}|p{0.3\linewidth}|p{0.3\linewidth}|}
\hline
\textbf{Point} & \textbf{Indian Law} & \textbf{English Law} \\
\hline
Past Consideration & Valid & Invalid \\
\hline
Third-party Consideration & Valid (any person can furnish) & Invalid (only promisee must provide) \\
\hline
Adequacy of Consideration & Immaterial & Immaterial \\
\hline
Written/Oral Contract & Valid in both forms & Valid in both forms \\
\hline
\end{tabular}

 
\vspace{0.5cm}
\textbf{4. "Insufficiency of consideration is immaterial; but an agreement without consideration is void." – Explain.}

\textbf{Explanation:} The law requires that there must be some consideration, but it need not be equal or adequate to the promise. Even a nominal consideration is sufficient.

\textbf{Example:} A sells his laptop worth 50,000 for 5,000. The court will not question the fairness if both parties agreed.

\textbf{Void without consideration:} If nothing is given or promised in return, the contract is usually void, unless it falls under exceptions (see Q5).

\vspace{0.5cm}
\textbf{5. Circumstances in which a contract without consideration may be valid}

Under Section 25 of the Indian Contract Act, an agreement made without consideration is void unless it falls within certain well-defined exceptions:

\begin{enumerate}
    \item \textbf{Natural love and affection:} If the agreement is between parties standing in a near relationship, made out of natural love and affection, and is in writing and registered, it is valid without consideration.

    \textit{Example:} A father promises to transfer property to his daughter by a registered document, out of love and affection. This is enforceable.

    \item \textbf{Past voluntary services:} If a person has voluntarily done something for another and the latter promises to compensate, that promise is enforceable.

    \textit{Example:} A saves B’s drowning child. Later, B promises to pay A for his brave act. This is enforceable in India.

    \item \textbf{Promise to pay time-barred debt:} A written and signed promise to pay a debt that is barred by the Limitation Act is valid.

    \textit{Example:} A owes B a sum of money, which becomes time-barred. A later promises in writing to repay. This is enforceable.

    \item \textbf{Completed gift:} Once a gift is made and delivered voluntarily, it does not require consideration.

    \textit{Example:} A donates a laptop to B. Later, A cannot demand it back on the ground of lack of consideration.

    \item \textbf{Agency:} According to Section 185 of the Indian Contract Act, no consideration is necessary to create an agency relationship.

    \textit{Example:} A authorizes B to act as his agent without any payment. This is valid.

    \item \textbf{Charitable subscriptions:} If a person promises to contribute to a charitable cause and on the faith of that promise, the promisee incurs liability, the promise is enforceable.

    \textit{Example:} A promises Rs. 10,000 to build a hospital. The hospital starts construction relying on it. A is bound to pay.
\end{enumerate}

\vspace{0.5cm}

\textbf{6. Stranger to a contract cannot sue – Rule and Exceptions}

\textbf{General Rule:}  
Only parties to a contract can sue to enforce it. A stranger to the contract has no *locus standi* (legal standing), even if the contract is made for his benefit.

\textbf{Exceptions:}
\begin{enumerate}
    \item \textbf{Trust:} A beneficiary can enforce rights under a trust created in his favor.

    \textit{Example:} X contracts with Y to hold property in trust for Z. Z can sue to enforce this trust.

    \item \textbf{Family arrangements:} In joint families, agreements made for the benefit of family members can be enforced by them.

    \textit{Example:} An elder brother agrees with others to provide for the marriage of his sister. The sister can enforce it.

    \item \textbf{Agency:} A principal can sue third parties for contracts entered into by his agent.

    \textit{Example:} A agent buys goods from B on behalf of C. C can sue B.

    \item \textbf{Assignment:} The assignee of a contractual right can sue in his own name.

    \textit{Example:} A assigns his right to receive rent to B. B can sue the tenant.

    \item \textbf{Covenants running with land:} The purchaser of immovable property may sue on covenants attached to it.

    \textit{Example:} A leases land to B with a covenant to repair a wall. B sells to C. C can sue if the wall is not repaired.
\end{enumerate}

\vspace{0.5cm}
\textbf{7. A stranger to the consideration may sue on a contract but not a stranger to the contract. - Explain.}

In Indian law, it is not necessary that consideration should move from the promisee. A third party (not being a stranger to the contract) can provide consideration.

\textit{Example:} A agrees to pay B if C delivers goods to B. C delivers, but A refuses to pay. B can sue A even though C gave the consideration.

But if someone is not a party to the contract at all (a stranger to contract), they cannot sue—even if they benefit from it.

\vspace{0.5cm}
\textbf{8. "A stranger to a contract cannot sue to enforce the contract." Discuss.}

This is a general principle of "privity of contract." Only parties who have entered into a contract can enforce it. However, this rule has notable exceptions, as discussed above (See Q6).

\textbf{Legal Justification:} Enforcing rights requires reciprocal obligations. A third party has not undertaken any obligation.

\vspace{0.5cm}
\textbf{9.}

(a) \textbf{Meaning of Consideration:} See Q1.

(b) \textbf{Valid agreements without consideration:}
\begin{itemize}
    \item Promise out of natural love and affection.
    \item Past voluntary services.
    \item Promise to pay time-barred debt.
    \item Completed gifts.
    \item Creation of agency.
\end{itemize}

\textit{Example:} A promises to pay B who saved his life. Valid under past voluntary service.

\vspace{0.5cm}
\textbf{10. When can a non-party sue a contract?}

Refer to the five exceptions in Q6. A non-party can sue:
\begin{itemize}
    \item If they are a trust beneficiary,
    \item A family settlement beneficiary,
    \item A principal in an agency relationship,
    \item An assignee,
    \item Or if covenants run with land.
\end{itemize}

\vspace{0.5cm}
\textbf{11.}

(a) \textbf{Definition:} (See Q1)

\textbf{Elements:}
\begin{itemize}
    \item Must move at promisor’s desire.
    \item May be past, present, or future.
    \item Can move from promisee or any third party.
    \item Must be lawful.
\end{itemize}

(b) \textbf{Agreement valid without consideration:} (See Q5)

\vspace{0.5cm}
\textbf{12. "An agreement without consideration is void unless it is in writing and registered." – Explain.}

This refers to Section 25(1) of the Indian Contract Act. A promise without consideration can be valid if:
\begin{itemize}
    \item It is made out of natural love,
    \item Between close relatives,
    \item In writing and registered.
\end{itemize}

\textit{Example:} A registered gift deed by a father to son is enforceable.

\vspace{0.5cm}
\textbf{13.}

(a) \textbf{Essential Factors of Consideration:}
\begin{itemize}
    \item Must be at promisor’s request.
    \item Can be past, present, or future.
    \item Must be lawful.
    \item Need not be adequate, but must be real.
\end{itemize}

(b) \textbf{Promise to pay time-barred debt:}  
Yes, valid. Section 25(3) allows a written and signed promise to revive time-barred debts.

\textit{Example:} A writes and signs a note promising to repay B a debt that is time-barred. This is enforceable.

\vspace{0.5cm}
\textbf{14. Objective Questions}

(i) \textbf{Two examples of valid contracts without consideration:}
\begin{itemize}
    \item Promise to pay time-barred debt (written and signed).
    \item Agreement out of natural love and affection (written and registered).
\end{itemize}

(ii) \textbf{Two exceptions to privity of contract rule:}
\begin{itemize}
    \item Beneficiary under a trust.
    \item Member in a family settlement.
\end{itemize}

\vspace{1cm}

\textbf{Case Title:} \textbf{Chinnaya v. Ramaya (1882) ILR 4 Mad 137}

\textbf{Facts of the Case:}  
A lady (the promisor) gifted some landed property to her daughter (the promisee). As a condition of the gift, the daughter was required to pay an annuity to the donor's brother. The daughter accepted the gift and agreed to make the payment. However, she subsequently refused to pay the annuity. The brother sued the daughter to enforce the payment.

\textbf{Legal Issue:}  
Can a person who is not a party to a contract (a stranger to the contract) but has furnished consideration, sue to enforce the contract?

\textbf{Arguments:}  
- The defendant (daughter) argued that there was no direct contract between her and the plaintiff (the brother), and hence, the plaintiff had no legal right to sue.
- The plaintiff argued that the gift deed and the condition to pay annuity were part of one transaction, and he was entitled to enforce the condition.

\textbf{Judgment:}  
The Madras High Court held in favour of the brother. The court ruled that although the brother was not a party to the contract between the lady and her daughter, the contract was made for his benefit, and he had furnished the consideration. Therefore, he could enforce the promise.

\textbf{Legal Principle Established:}  
The case laid down an important exception to the rule of privity of contract: A person who is a beneficiary under a contract and has furnished consideration (even if not a direct party to the contract) may sue to enforce the contract.

\textbf{Importance:}  
- Supports the Indian position that consideration may move from a person other than the promisee.
- Distinguishes Indian law from strict English law which does not allow such claims.
- Reinforces that a “stranger to the contract” cannot sue, but a “stranger to consideration” may sue in India.

\textbf{Key Learning:}
\begin{itemize}
    \item Indian Contract Law is more flexible than English Law regarding consideration.
    \item A third party can enforce a contract if the contract was made for their benefit and consideration is furnished.
    \item This case is an important authority in exceptions to the doctrine of privity of contract in India.
\end{itemize}

\textbf{Relevant Section:}  
Section 2(d) of the Indian Contract Act, 1872 – "When, at the desire of the promisor, the promisee or any other person has done or abstained from doing... such act is called consideration."

\vspace{1cm}
\clearpage

\textbf{Case Title:} \textbf{Alka Bose v. Parmatma Devi \& Ors (2009) 2 SCC 582}

\textbf{Facts of the Case:}  
Alka Bose (plaintiff) entered into an oral agreement with Parmatma Devi (defendant) for the purchase of a house property. The agreement was not registered or formally documented, but the plaintiff contended that possession had been delivered, and part payment had been made. When the seller later refused to honour the sale, the plaintiff sued for specific performance of the contract.

\textbf{Legal Issue:}  
Can an oral contract for sale of immovable property be specifically enforced in court when no written agreement exists?

\textbf{Arguments:}
- The plaintiff argued that the oral agreement was valid under the Indian Contract Act and that part performance had taken place, making it enforceable.
- The defendant argued that without a written and registered agreement, the contract could not be enforced under the Transfer of Property Act, 1882 and Registration Act, 1908.

\textbf{Judgment:}  
The Supreme Court held that although an agreement to sell immovable property is required to be in writing under Section 54 of the Transfer of Property Act, **an oral agreement is not invalid** per se under the Indian Contract Act. However, **specific performance** of such an agreement will not be granted unless the terms are certain and unambiguous, and the existence of the agreement can be proved by **evidence of part performance, conduct, or other corroborating factors**.

\textbf{Legal Principle Established:}  
- Oral contracts are valid under Indian law.
- Specific performance of oral contracts may be allowed where the contract is clearly proven and part performance (such as possession or payment) is demonstrated.
- However, for immovable property transactions, written agreements are strongly advisable due to evidentiary and statutory requirements.

\textbf{Importance:}  
This case demonstrates the judicial flexibility in India regarding oral contracts, while also highlighting the practical and evidentiary challenges involved.

\textbf{Key Learning:}
\begin{itemize}
    \item Oral agreements are valid and enforceable unless specifically barred by law.
    \item Courts will examine evidence of conduct, part performance, and third-party corroboration.
    \item In real estate, written agreements are safer and legally stronger due to statutory requirements.
\end{itemize}

\textbf{Relevant Laws:}
\begin{itemize}
    \item Section 10, Indian Contract Act, 1872 – Essentials of a valid contract.
    \item Section 54, Transfer of Property Act, 1882 – Sale of immovable property.
    \item Section 17, Registration Act, 1908 – Mandatory registration of certain documents.
\end{itemize}

\vspace{1cm}

\clearpage

\notesection{Fraud, Misstatement, Mistake, Undue Influence, Duress, Contracts of Adhesion, and Unconscionable Contracts}{28-04-25 Monday}

\textbf{1. Fraud}  
\textit{Definition:} Fraud involves intentionally deceiving someone for personal gain. The deceived party enters a contract they wouldn’t have otherwise agreed to if they knew the truth.  

\textbf{Example 1:}  
A car seller tells a buyer that a used car is brand new, even though it's several years old and has hidden defects. The buyer agrees to purchase it, but the seller is committing fraud.

\textbf{Example 2:}  
A person sells a piece of artwork claiming it is an original painting by a famous artist, but it's actually a counterfeit.

\vspace{0.5cm}
\textbf{2. Misstatement}  
\textit{Definition:} A misstatement is an incorrect or misleading statement made either intentionally or unintentionally. It often occurs during contract negotiations.  

\textbf{Example 1:}  
A real estate agent tells a buyer that the house has no structural issues, but later, the buyer discovers it has major foundation problems.

\textbf{Example 2:}  
A seller advertises a used laptop as “like new,” but it has scratches and battery issues that weren’t disclosed.

\vspace{0.5cm}
\textbf{3. Mistake}  
\textit{Definition:} A mistake occurs when one or both parties are wrong about a key element of the contract. This could be a mistake about the facts or the law.  

\textbf{Example 1:}  
A buys a painting believing it's by a famous artist, but it turns out to be a reproduction. Both parties were unaware of the mistake.

\textbf{Example 2:}  
Two people agree on a price for a piece of land, but one party mistakenly believes the land is bigger than it actually is. This mistake may invalidate the contract if it’s significant enough.

\vspace{0.5cm}
\textbf{4. Undue Influence}  
\textit{Definition:} Undue influence happens when one party uses their position of power or trust to manipulate the other party into agreeing to a contract that they wouldn't have otherwise agreed to.  

\textbf{Example 1:}  
An elderly parent signs over assets to their child, not because they want to, but because the child used their position of trust to pressure them.

\textbf{Example 2:}  
An employer pressures an employee to sign a non-compete agreement by threatening to fire them if they don’t.

\vspace{0.5cm}
\textbf{5. Duress}  
\textit{Definition:} Duress occurs when one party is forced to enter into a contract through threats or pressure.  

\textbf{Example 1:}  
A person signs a contract to sell their house under threat of violence if they don't comply.

\textbf{Example 2:}  
A company forces an employee to sign a non-disclosure agreement under the threat of firing them.

\vspace{0.5cm}
\textbf{6. Contract of Adhesion}  
\textit{Definition:} A contract of adhesion is one where one party has significantly more power than the other and dictates the terms of the agreement. The weaker party is left with little to no room for negotiation.  

\textbf{Example 1:}  
A consumer agrees to terms and conditions for a mobile phone service, but the terms are non-negotiable, and they have no choice but to accept them.

\textbf{Example 2:}  
A bank gives a customer a loan agreement with a very high interest rate and few options for negotiation. The customer has no choice but to accept the terms.

\vspace{0.5cm}
\textbf{7. Unconscionable Contracts}  
\textit{Definition:} An unconscionable contract is one that is so unfair to one party that it shocks the conscience. These contracts often involve one party taking advantage of the other’s lack of knowledge or bargaining power.  

\textbf{Example 1:}  
A payday loan company offers a loan with a 500\% annual interest rate to someone with poor credit. The terms are outrageously unfair to the borrower.

\textbf{Example 2:}  
A wealthy business owner forces a poor supplier to sign a contract that gives them only a small fraction of the value of the goods they provide, knowing the supplier has no other option.

\vspace{1cm}
\clearpage


\notesection{Free Consent and Related Concepts}{25-04-25 Saturday}

\textbf{Free Consent:}  
According to Section 14 of the Indian Contract Act, 1872, consent is said to be free when it is not caused by:
\begin{enumerate}
    \item \textbf{Coercion}
    \item \textbf{Undue influence}
    \item \textbf{Fraud}
    \item \textbf{Misrepresentation}
    \item \textbf{Mistake}
\end{enumerate}

If consent is obtained through any of the above means, the contract is voidable at the option of the party whose consent was affected.

\vspace{0.5cm}
\textbf{1. Coercion (Section 15)}  
\textit{Definition:} Coercion is the committing or threatening to commit any act forbidden by the Indian Penal Code, or the unlawful detaining or threatening to detain any property to induce the other party to agree to a contract.  

\textbf{Example 1:}  
A threatens to kill B unless B signs a contract for the sale of land. B signs the contract under duress, and it is not enforceable.

\textbf{Example 2:}  
A’s business is at risk, and B demands money under threat of harm. If A pays the money under this threat, the consent is obtained by coercion.

\vspace{0.5cm}
\textbf{2. Undue Influence (Section 16)}  
\textit{Definition:} Undue influence arises when one party uses their position of power over the other party to persuade them to enter into a contract. It involves manipulation of trust or authority.  

\textbf{Example 1:}  
An elderly person signs a will under the influence of a relative who has power over them. This can be challenged as the will was made under undue influence.

\textbf{Example 2:}  
A company executive forces a subordinate to sign a non-compete agreement under threat of job loss, which constitutes undue influence.

\vspace{0.5cm}
\textbf{3. Fraud (Section 17)}  
\textit{Definition:} Fraud involves intentionally misleading another party to gain an unfair advantage. It includes false statements, concealment of facts, or misrepresentation of information.  

\textbf{Example 1:}  
A seller knowingly sells a defective product to a buyer while hiding the faults, making a fraudulent misrepresentation.

\textbf{Example 2:}  
A builder falsely claims that the house is free of legal encumbrances, knowing it is under litigation, which constitutes fraud.

\vspace{0.5cm}
\textbf{4. Misrepresentation (Section 18)}  
\textit{Definition:} Misrepresentation occurs when one party makes an untrue statement about a fact to another party, leading them to enter into a contract. Unlike fraud, misrepresentation is typically innocent or unintentional.  

\textbf{Example 1:}  
A tells B that the house they are selling has been renovated recently, when it has not. This is a misrepresentation and the contract is voidable by B.

\textbf{Example 2:}  
A describes a used car as being in good condition, but without intending to deceive, he was mistaken. B can void the contract based on misrepresentation.

\vspace{0.5cm}
\textbf{5. Mistake (Section 20, 21, 22)}  
\textit{Definition:} A mistake occurs when one or both parties are under an incorrect belief about a key aspect of the contract. This could be a mistake about the subject matter, law, or other fundamental elements.

\textbf{Example 1:}  
A buys a painting from B, thinking it is an original, but later finds out it is a reproduction. The contract can be voided due to a mutual mistake about the subject matter.

\textbf{Example 2:}  
A and B agree on the sale of a house, but A mistakenly believes the house is on a larger plot of land than it actually is. This is a unilateral mistake and may affect the contract's validity.

\vspace{0.5cm}
\textbf{Consequences of Lack of Free Consent:}  
If consent is not free due to any of the above factors, the contract is voidable at the option of the party whose consent was affected. The affected party can choose to either affirm or cancel the contract.  

\textbf{Example:}  
A contracts to sell goods to B under the threat of violence (coercion). B can either affirm the contract (if they wish to honor it) or void it based on coercion.

\vspace{1cm}


\textbf{1. (a) State when consent is not said to be free.}  

Consent is not said to be free when it is obtained by:  
\begin{enumerate}
    \item \textbf{Coercion:} Consent is obtained by threatening to commit an illegal act or unlawfully detaining property (Section 15 of the Indian Contract Act).
    \item \textbf{Undue Influence:} Consent is obtained when one party uses their power or position over the other party to induce them into an agreement (Section 16).
    \item \textbf{Fraud:} Consent is obtained by intentionally deceiving the other party with false statements or concealment of facts (Section 17).
    \item \textbf{Misrepresentation:} Consent is obtained when a false statement is made about a material fact that leads the other party into agreement (Section 18).
    \item \textbf{Mistake:} Consent is affected when both parties are under a misunderstanding about a material fact or law (Sections 20-22).
\end{enumerate}

\textbf{(b) What is the effect of such consent on the formation of a contract?}  

If consent is not free due to any of the factors above, the contract is \textbf{voidable} at the option of the party whose consent was affected. This means that the affected party can choose to either affirm or rescind the contract. The contract is not automatically void, but the affected party can take legal action to void it.

\vspace{0.5cm}
\textbf{2. What is meant by undue influence? Give two examples.}  

\textbf{Undue Influence:}  
Undue influence occurs when one party uses their position of power or trust to pressure the other party into entering a contract they wouldn’t have otherwise agreed to. This often involves the use of a dominant position to control or manipulate the weaker party’s will.  

\textbf{Example 1:}  
A father pressures his son into signing over his property to him by exploiting the son’s trust and dependence.

\textbf{Example 2:}  
An employer forces an employee to sign a non-compete agreement under threat of termination if the employee refuses.

\vspace{0.5cm}
\textbf{3. When is consent said to be free? Distinguish between coercion and undue influence.}  

Consent is said to be free when it is given voluntarily, without any form of pressure or manipulation. It must be the result of the parties' own will, without any of the factors such as coercion, undue influence, fraud, misrepresentation, or mistake.  

\textbf{Coercion vs. Undue Influence:}  
\begin{itemize}
    \item \textbf{Coercion:} Involves threats of physical harm or illegal acts to force someone into agreeing to a contract (e.g., threatening to kill or injure).
    \item \textbf{Undue Influence:} Involves using a position of trust or power over the other party to influence them unfairly (e.g., a parent pressuring a child to sign a property transfer).
\end{itemize}

\vspace{0.5cm}
\textbf{4. Define and distinguish 'misrepresentation' and 'fraud'. What remedies are available to the aggrieved party?}  

\textbf{Misrepresentation:}  
Misrepresentation occurs when one party makes an untrue statement about a material fact, either intentionally or unintentionally, that leads the other party to enter into a contract.

\textbf{Fraud:}  
Fraud involves intentionally deceiving another party by making false statements or concealing material facts with the intent to deceive the other party into entering the contract.

\textbf{Differences:}  
\begin{itemize}
    \item \textbf{Intent:} Fraud is intentional, while misrepresentation may be innocent or unintentional.
    \item \textbf{Knowledge:} Fraud involves the knowledge of falsity, whereas misrepresentation may occur without knowledge of the falsehood.
\end{itemize}

\textbf{Remedies for Aggrieved Party:}  
- In case of \textbf{fraud}, the aggrieved party can sue for damages or seek rescission of the contract.
- In case of \textbf{misrepresentation}, the aggrieved party can seek rescission of the contract and may claim damages if the misrepresentation was fraudulent.

\vspace{0.5cm}
\textbf{5. "Mere silence as to facts is not fraud." Explain with two illustrations.}  

\textbf{Explanation:}  
Mere silence regarding facts is generally not considered fraud. Fraud occurs only when there is a deliberate concealment of material facts or a misleading statement made to deceive the other party.

\textbf{Example 1:}  
A seller does not inform a buyer that the car he is selling has a mechanical issue. The buyer later discovers the issue but cannot claim fraud, as the seller’s silence was not intentional deceit.

\textbf{Example 2:}  
A seller of land doesn’t mention to the buyer that the land is part of a disputed property, but does not actively conceal this fact. The buyer cannot claim fraud solely because of this silence.

\vspace{1cm}

\textbf{6. "A contract caused by mistake is void." Explain.}

\textbf{Explanation:}  
Under Sections 20, 21, and 22 of the Indian Contract Act, a contract caused by mistake is void if the mistake is about a material fact or a law that is essential to the agreement.  
\textit{Example:}  
A and B enter into a contract to sell a piece of land. Both parties believe that the land is free of encumbrances. Later, it is discovered that the land is subject to a mortgage. This mistake about a material fact makes the contract void.

However, a contract based on an erroneous belief about the quality or value of an item (such as purchasing goods thinking they are of a particular quality) is generally voidable, not void.

\vspace{0.5cm}
\textbf{7. Give answers with reasons whether the following cases are instances of fraud:}

\textbf{(a)} A sells, by auction, to B, a horse which A knows to be unsound. A declares nothing to B about the horse's unsoundness.  
\textit{Answer:}  
This is \textbf{fraud}. A knowingly withheld material information about the horse’s unsoundness, which is intended to deceive B into buying it. This constitutes fraud, and B can sue for damages or rescind the contract.

\textbf{(b)} Suppose, B is A's daughter and has just come of age. Is A then bound to tell B that the horse is unsound?  
\textit{Answer:}  
Even though B is A's daughter, A has a duty to disclose any material facts about the horse’s condition. The relationship does not absolve A from disclosing the unsoundness. Failing to do so would still be considered fraud, and B could claim damages.

\textbf{(c)} B says to A, "If you do not deny it, I shall take that the horse is sound." A says nothing.  
\textit{Answer:}  
This is \textbf{not fraud}. A's silence in response to B’s statement does not amount to fraud, as there was no active misrepresentation or concealment of facts by A. Silence alone, in the absence of a duty to speak, does not constitute fraud.

\vspace{0.5cm}
\textbf{8. Problems:}

\textbf{(a)} A and B make a contract on the mistaken belief that a particular debt is barred by the Indian law of limitation. Is the contract void? Is the contract voidable? 

\textit{Answer:}  
The contract is \textbf{voidable}, not void, because both A and B entered into the contract based on a mutual mistake of law. If the debt is not barred, the contract can still stand unless either party seeks to rescind it due to the mistake.

\textbf{(b)} A fraudulently informs B that A's house is free from encumbrance. B thereupon buys the house. The house is subject to a mortgage. What are the rights of B? 

\textit{Answer:}  
B has the right to \textbf{rescind the contract} and claim damages for fraud. A's fraudulent misrepresentation of the house being free from encumbrance misled B into purchasing the property. B can seek a remedy for the loss suffered due to A’s fraud.

\textbf{(c)} A agrees to sell B a specific cargo of goods per S. S. Malwa supposed to be on its way from London to Bombay. It turns out that before the day of the bargain, S. S. Malwa had been cast away and the goods were lost. Discuss the respective rights of A and B. 

\textit{Answer:}  
In this case, the contract is \textbf{frustrated} under Section 56 of the Indian Contract Act due to the loss of the goods before the sale was concluded. A cannot be held liable for failing to deliver the goods as they were lost due to an external event. B is not bound to pay, and the contract is void due to the impossibility of performance.

\vspace{1cm}

\textbf{(d)} A agrees to buy from B a certain elephant. It turns out that the elephant was dead at the time of the bargain, though neither party was aware of the fact. Discuss the rights of A and B. 

\textit{Answer:}  
The contract is \textbf{void} due to \textit{frustration of contract} under Section 56 of the Indian Contract Act. Both parties were unaware of the death of the elephant, making the subject matter of the contract impossible to perform. Neither party can be held liable for the non-performance, and A is entitled to a refund of the purchase price.  

\vspace{0.5cm}
\textbf{(e)} A sells a horse to B knowing full well that the horse is vicious. A does not disclose the nature of the horse to B. Is the sale valid? 

\textit{Answer:}  
The sale is \textbf{invalid} based on \textit{fraud}. A deliberately concealed the dangerous nature of the horse from B, which amounts to fraud under Section 17 of the Indian Contract Act. B can sue A for damages or seek to rescind the contract. A's failure to disclose the vicious nature of the horse constitutes fraudulent misrepresentation.

\vspace{0.5cm}
\textbf{(f)} A, a man enfeebled by disease, is induced by B, his medical attendant, to agree to pay B a sum of one lakh rupees for his professional services. Is the agreement valid? Give reasons for your answer.

\textit{Answer:}  
The agreement is likely to be \textbf{voidable} due to \textit{undue influence}. B, being A's medical attendant, is in a position of power and trust over A, who is vulnerable due to his weakened health. The contract may be rescinded by A if it is found that B took advantage of A's weakness to induce the agreement. This constitutes undue influence under Section 16 of the Indian Contract Act.

\vspace{0.5cm}
\textbf{(g)} A buys a piece of ordinary cloth from B. A thinks erroneously that the cloth is of high quality. B knows that A is under a mistake but keeps quiet on this matter. When A realizes his mistake, he wants to set aside the contract on the ground that B had knowingly committed fraud in not pointing out his mistake. Discuss if the contract is voidable. 

\textit{Answer:}  
This case involves \textit{fraudulent misrepresentation} by B. B knew that A was under a mistake regarding the quality of the cloth, but B chose to remain silent and not correct the mistake. Silence in the presence of a duty to disclose can amount to fraud. Therefore, the contract is \textbf{voidable} by A, who can rescind the contract on the grounds of fraud.

\vspace{0.5cm}
\textbf{(h)} A sells B his horse for Rs. 500. The horse is blind in one eye, but B does not know this until after the sale is completed. Is A liable to B on the ground of fraud? 

\textit{Answer:}  
A may be liable for \textit{fraud} if he deliberately concealed the fact that the horse was blind in one eye. Since A did not inform B of the defect, B could claim that A committed fraud under Section 17 of the Indian Contract Act. If A intentionally withheld information, the contract may be voidable at B's discretion, and B can seek compensation or rescission of the contract.

\vspace{0.5cm}
\textbf{(i)} X sold a mare to B which had a cracked hoof. X filled up the hoof in order to prevent detection even after diligent examination. What's the right of B?  

\textit{Answer:}  
This is an instance of \textit{fraudulent misrepresentation}. X intentionally concealed the defect (the cracked hoof) by filling it up, which misled B into believing that the mare was in better condition. B has the right to \textbf{rescind the contract} and claim damages for fraud, as X's actions deliberately misrepresented the condition of the mare.

\vspace{1cm}

\textbf{9. Objective Questions:}

\textbf{(i) Give two examples where undue influence has been exercised in the contract.}

\begin{itemize}
    \item \textbf{Example 1:} A businessman forces an employee to sign an agreement waiving their rights to a bonus, under threat of termination. This is undue influence.
    \item \textbf{Example 2:} A father pressures his adult son, who relies on him financially, to transfer his property in the father’s name. This constitutes undue influence due to the power imbalance.
\end{itemize}

\vspace{0.5cm}
\textbf{(ii) Suicide is no crime. True or False?}  

\textit{Answer: False.}  
Under Section 309 of the Indian Penal Code (IPC), suicide was a punishable offense until the law was amended in 2014. However, attempts to suicide may not be punishable now, but aiding or abetting suicide remains a criminal act.

\vspace{0.5cm}
\textbf{(iii) Does silence as to fact amount to fraud? If so, give one example.}

\textit{Answer: Yes, silence may amount to fraud if there is a duty to disclose the fact.}  
\textbf{Example:} A sells a house to B, knowing it has a serious structural defect but does not disclose it. A's silence constitutes fraud, as he had a duty to inform B of the defect.

\vspace{1cm}

\clearpage

\notesection{Void and Voidable Agreements}{2025-05-09 Wednesday}

\textbf{Void Agreements}  

A \textbf{void agreement} is one that is not enforceable by law. It is treated as if it never existed in the first place, and no party can claim rights or obligations under it. This may happen for various reasons, including:

\begin{itemize}
    \item \textbf{Illegality:} If the agreement involves illegal actions or violates public policy (e.g., a contract to commit a crime).
    \item \textbf{Lack of capacity:} If one or more parties lack legal capacity to contract (e.g., minors, persons of unsound mind).
    \item \textbf{Lack of consideration:} If the agreement lacks any form of lawful consideration.
    \item \textbf{Impossible terms:} If the terms of the agreement are impossible to perform.
\end{itemize}

\textbf{Examples of Void Agreements:}
\begin{enumerate}
    \item A contract to sell stolen goods.
    \item A contract to murder someone.
    \item A contract made by a minor (e.g., a 17-year-old contracts to buy a car without parental consent).
    \item A contract to do something physically impossible (e.g., promising to fly unaided to the moon).
    \item An agreement that involves fraudulent activity or misrepresentation.
    \item A contract where one party is intoxicated and cannot understand the terms.
    \item An agreement that contravenes public policy, such as a contract to bribe a government official.
\end{enumerate}

\vspace{0.5cm}

\textbf{Voidable Agreements}  

A \textbf{voidable agreement} is a valid contract that may be legally voided by one of the parties to the agreement. This usually occurs due to some form of unfairness or incapacity of one of the parties.

\begin{itemize}
    \item \textbf{Fraud or Misrepresentation:} If a party is induced to enter into an agreement by fraudulent or misleading information.
    \item \textbf{Coercion:} If one party is forced or threatened to enter the contract.
    \item \textbf{Undue Influence:} If one party takes advantage of a position of power over the other to force them into a contract.
    \item \textbf{Lack of consent:} If one of the parties did not give free consent due to mistake or misapprehension.
\end{itemize}

\textbf{Examples of Voidable Agreements:}
\begin{enumerate}
    \item A contract entered under threat or coercion.
    \item An agreement induced by fraudulent misrepresentation.
    \item A contract signed by a person under the influence of alcohol or drugs.
    \item A contract where a person signs without fully understanding its terms (e.g., due to misunderstanding or lack of knowledge).
    \item An agreement made with a person of unsound mind, but with their understanding after recovery.
    \item A contract entered into by a minor, but later ratified by the minor when they reach adulthood.
    \item An agreement made under undue influence, such as a contract where one party manipulates another due to a superior relationship (e.g., guardian and ward).
    \item A contract signed under duress or threats of harm.
\end{enumerate}

\vspace{1cm}

\begin{table}[h!]
\centering
\begin{tabular}{|p{5cm}|p{5cm}|p{5cm}|}
\hline
\textbf{Void Agreements} & \textbf{Voidable Agreements} & \textbf{Enforceable Agreements} \\
\hline
An agreement that is not legally binding. It has no legal effect. & An agreement that is initially valid but one party can choose to void it. & An agreement that is legally binding and enforceable by law. \\
\hline
\textbf{Examples:} & \textbf{Examples:} & \textbf{Examples:} \\
1. Agreement to commit a crime. & 1. Contract entered into under coercion. & 1. Contract for the sale of goods. \\
2. Contract to sell stolen goods. & 2. Agreement induced by fraud. & 2. Employment contract with terms clearly outlined. \\
3. Agreement with a person of unsound mind. & 3. Contract made under undue influence. & 3. Lease agreement between a landlord and tenant. \\
4. Impossible agreements (e.g., to fly unaided to the moon). & 4. Contract signed by a minor, but later ratified. & 4. Loan agreement with legal terms. \\
5. Agreement involving illegal activities. & 5. Agreement made under misrepresentation. & 5. Real estate transaction with a written contract. \\
6. Agreements with no lawful consideration. & 6. An agreement that can be ratified by the affected party. & 6. Contract for services between two individuals. \\
\hline
\end{tabular}
\caption{Comparison of Void, Voidable, and Enforceable Agreements}
\end{table}

\vspace{1cm}
\clearpage


\notesection{Minority and Legal Implications of Minor's Agreements}{2025-05-09 Wednesday}

\textbf{1. Definition of Minority}  
In legal terms, a \textbf{minor} is someone who has not yet reached the legal age of majority, typically \textbf{18 years old} (in most jurisdictions, including India and common law countries). This is the age when a person gains full legal capacity to enter into binding contracts.

\vspace{0.3cm}

\textbf{2. Law Regarding Minor's Agreements}  
The general rule is that \textbf{contracts entered into by minors are not legally enforceable}. This is because:
\begin{itemize}
    \item A minor is presumed to lack the legal capacity to understand the consequences of entering into a contract.
    \item The law recognizes that minors may lack the maturity needed to make informed decisions.
\end{itemize}

\textbf{However, there are exceptions to this rule:}

\vspace{0.3cm}

\textbf{General Rule: Agreements by Minors are Void}  
A contract made by a minor is generally considered \textbf{void}, meaning it cannot be enforced by either party. Since minors are presumed incapable of understanding the contract's terms, these contracts are treated as if they never existed.

\textbf{Example:}  
A 16-year-old enters into an agreement to buy a smartphone on credit. Later, the minor refuses to pay. The contract is \textbf{void}, and the seller cannot enforce the agreement in court.

\vspace{0.3cm}

\textbf{Exceptions to the General Rule}  
There are specific situations where minors can enter into contracts that may be enforceable:

\begin{itemize}
    \item \textbf{Contracts for Necessaries (Basic Needs):}  
    A minor can validly contract for goods or services that are essential for their life, such as food, clothing, or medical treatment. These contracts are enforceable against the minor, but the minor will only be liable to pay for the actual cost of the goods or services provided.

    \textbf{Example:}  
    A minor buys necessary medicines from a pharmacy. Although the minor cannot enter into most contracts, they are still responsible for paying for these essential goods.

    \item \textbf{Beneficial Contracts (Agreements for Minor's Benefit):}  
    Agreements that are clearly for the minor’s benefit may be enforceable, especially when they are advantageous to the minor.

    \textbf{Example:}  
    A minor signs an agreement to receive education or training for their future career. Such agreements are generally considered beneficial and may be enforceable.

    \item \textbf{Contract of Employment (with Parent's or Guardian's Consent):}  
    If a minor enters into an employment contract with the approval of their parents or guardians, this agreement may be enforceable. It must be for the minor’s benefit and with the guardian’s consent.

    \textbf{Example:}  
    A 17-year-old enters into an apprenticeship agreement to learn a trade under the supervision of a qualified professional. If the guardian agrees, the minor can be bound by the terms of the contract.

\end{itemize}

\vspace{0.3cm}

\textbf{Voidable Contracts}  
In certain cases, contracts that are voidable by the minor may arise, particularly if the minor has been \textbf{fraudulently induced} or misled into the contract. In such cases, the minor may choose to either \textbf{ratify} (approve) or \textbf{disaffirm} (cancel) the contract upon reaching the age of majority.

\textbf{Example:}  
A 17-year-old is tricked into signing a contract to pay an inflated price for a used car. Once the minor turns 18, they can choose to cancel (disaffirm) the contract.

\vspace{0.3cm}

\textbf{Ratification Upon Reaching Majority}  
When a minor turns 18, they may choose to \textbf{ratify} a contract they made while still a minor. If they choose to ratify it, the contract becomes enforceable.

\textbf{Example:}  
A minor enters into a contract to purchase a laptop. Upon reaching the age of majority, the minor may choose to continue with the contract and pay the agreed amount, thus ratifying the contract.

\vspace{1cm}


\textbf{Persons of Unsound Mind}  

A person is considered of unsound mind when they are unable to understand the nature and consequences of their actions, due to mental illness or disorder. In legal terms, \textbf{such persons lack the capacity to contract}. According to Section 12 of the Indian Contract Act, 1872, a contract made by a person of unsound mind is void, unless the contract is made during a lucid interval.

\textbf{Example:}  
If a person with schizophrenia enters into a contract to buy goods, that contract is void unless it can be proved that they were in a sound state of mind at the time of signing.

\vspace{0.5cm}

\textbf{Disqualified Persons}  

Certain persons are disqualified by law from entering into contracts, regardless of their age or mental state. These include:
- \textbf{Minors}: Persons under the age of 18, as discussed earlier.
- \textbf{Persons who are disqualified by law}: This includes individuals banned from contracting due to bankruptcy, insolvency, or legal restrictions.

\textbf{Example:}  
A person declared bankrupt is legally incapable of contracting certain financial agreements.

\vspace{0.5cm}

\textbf{Answers to Queries:}

\textbf{(a) What is the minor's position in the law of contract? What is the leading case on the point?}

A minor, under Section 11 of the Indian Contract Act, 1872, \textbf{lacks the legal capacity to enter into contracts}. As a result, contracts made by minors are generally \textbf{void} (not enforceable by law). There is an exception for contracts that are made for "necessaries" (such as food, clothing, or education) or contracts made for the minor’s benefit. In such cases, the minor may be liable to pay for the goods or services provided.

\textbf{Leading Case:}  
The leading case on this point is \textbf{Mohiri Bibi v. Dharmodas Ghose (1903)}, where the Privy Council ruled that a contract entered into by a minor is void, except where the contract is for necessaries.

\vspace{0.5cm}

\textbf{(b) What do you mean by capacity to enter into contract?}

\textbf{Capacity to contract} refers to the legal ability of a person to enter into a binding contract. A person must meet the following conditions:
- They must be of the age of majority (usually 18 years or older).
- They must be of sound mind and mentally capable of understanding the nature and consequences of the contract.
- They must not be disqualified by law (e.g., due to bankruptcy or insolvency).

\vspace{0.5cm}

\textbf{(c) A, a trader, supplied B, a minor, with rice needed for his consumption. B refuses to pay the price. Can A recover the price?}

In this case, since B is a minor and the rice was necessary for their consumption, the agreement could be classified as a contract for "necessaries." According to the law, a minor is bound to pay for necessaries supplied to them, even though the general rule is that contracts with minors are void.

\textbf{Answer:}  
Yes, A can recover the price of the rice, as it falls under the category of "necessaries," which a minor is liable for.

\vspace{0.5cm}

\textbf{(d) What is the effect of agreements entered into by persons of unsound mind?}

An agreement made by a person of unsound mind is generally \textbf{void}, as they lack the mental capacity to understand the nature of the contract. However, if the person enters the agreement during a lucid interval (a period of clear mental understanding), the contract may be valid. In case of mental illness, the contract may be voidable at the discretion of the person once they recover or by their legal representative.

\textbf{Example:}  
If a person who is mentally ill signs a contract to buy property, the contract is void, but if they were in a lucid interval at the time of signing, the contract may be enforceable.

\vspace{0.5cm}

\textbf{(e) Write a note on the contractual capacity of aliens; foreign sovereigns; women.}

1. \textbf{Aliens (Foreign Nationals):}  
Aliens (persons who are not citizens of the country) generally have the same capacity to contract as a citizen, but they may face restrictions or special conditions depending on the nature of the contract (e.g., land ownership restrictions in some countries).

\textbf{Example:}  
A foreigner may enter into a contract for the purchase of goods in India but may be restricted from owning land without special permission.

2. \textbf{Foreign Sovereigns:}  
Foreign sovereigns (heads of state) or representatives of foreign governments may also enter into contracts, but these contracts are often subject to international law, diplomatic agreements, and treaties.

\textbf{Example:}  
A contract signed between the Indian government and a foreign nation for trade agreements is enforceable under international law, though not directly enforceable in domestic courts.

3. \textbf{Women:}
In many jurisdictions, women have the same contractual capacity as men. However, historically, there were certain legal restrictions on women’s ability to contract independently, especially in marriage or family matters. Today, women are generally treated equally in terms of entering into contracts.

\textbf{Example:}  
A married woman in most modern legal systems can enter into contracts, such as a business agreement, without needing consent from her husband.

\vspace{0.5cm}

\textbf{Case Study 1: Mohiri Bibi v. Dharmodas Ghose (1903) 30 Cal 539 (Privy Council)}

\textbf{Facts of the Case:}  
In this landmark case, Dharmodas Ghose, a minor (aged 18), borrowed money from a moneylender, Mohiri Bibi, and executed a mortgage deed. The minor was fully aware that he was underage, but he did not inform the lender. The minor later refused to pay the money, arguing that the contract was void because he was a minor at the time of entering into the contract.

\textbf{Legal Issue:}  
The key issue was whether a contract entered into by a minor was void and unenforceable, even though the minor knew he was underage.

\textbf{Judgment:}  
The Privy Council ruled that the contract was indeed \textbf{void} because minors lack the capacity to contract under the Indian Contract Act. The ruling reinforced the principle that a minor's agreement is void, regardless of whether the minor had knowledge of their age or not.

\textbf{Confusing Aspect:}  
The confusion arises because, in this case, the minor was aware of his legal incapacity but still entered into the agreement. This leads to the question: should the knowledge of the minor affect the enforceability of the contract? In this case, the law took a strict view, making the contract void despite the minor's awareness.

\textbf{Legal Principle:}  
This case established that a minor’s contract is void and unenforceable regardless of the minor's knowledge or intentions, as the law does not allow minors to make binding contracts.

\textbf{Example of Confusion:}  
If a minor knowingly enters into a contract, can they be held responsible for the consequences of their actions? The case suggests not, but some jurisdictions have made exceptions in cases involving fraud or misrepresentation.

\vspace{0.5cm}

\textbf{Case Study 2: Bell v. The Waltham Forest London Borough Council (2000) 2 WLR 1070}

\textbf{Facts of the Case:}  
In this case, Bell, who suffered from a mental disorder, was admitted to a hospital. While receiving treatment, Bell entered into a contract with the Waltham Forest London Borough Council for the provision of services that he didn’t need. He later tried to cancel the contract, arguing that the agreement should be void because he was of unsound mind at the time of entering into it.

\textbf{Legal Issue:}  
The issue here was whether a contract entered into by a person of unsound mind should automatically be void, and if the person’s mental state could affect the contract's validity.

\textbf{Judgment:}  
The Court ruled that the contract was \textbf{voidable} at the discretion of the person of unsound mind. However, the contract was not automatically void. The court decided that the key factor was whether the person of unsound mind could understand the nature of the transaction at the time it was entered into.

\textbf{Confusing Aspect:}  
This case raises a crucial point about the enforceability of contracts by persons of unsound mind. Unlike minors, whose contracts are void, the contracts entered by persons of unsound mind are \textbf{voidable} if they can prove they lacked the ability to understand the agreement at the time it was made. However, the burden of proving unsoundness rests with the person challenging the contract.

\textbf{Legal Principle:}  
A contract made by a person of unsound mind is voidable at the discretion of the person once they regain mental capacity, but not automatically void. The law takes a more flexible approach than with minors.

\textbf{Example of Confusion:}  
If someone of unsound mind enters into a contract during an episode of their condition but later recovers, can the contract be enforced? In this case, the contract was voidable, but this principle isn’t always consistently applied across jurisdictions.

\vspace{1cm}
\clearpage




% new note 
\notesection{2025-04-23}{Wednesday}

\vspace{1cm}
\clearpage

% end of the note 


\end{document}
