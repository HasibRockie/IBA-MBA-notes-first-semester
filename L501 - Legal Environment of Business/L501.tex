\documentclass[12pt,a4paper]{book}

% Fonts & Typography — Elegant and Professional
\usepackage[T1]{fontenc}
\usepackage{kpfonts} % Sleek, modern font
\usepackage{microtype}

% Essential Packages
\usepackage{graphicx}
\usepackage{fancyhdr}
\usepackage{tocloft}
\usepackage{titlesec}
\usepackage{datetime}
\usepackage{hyperref}
\usepackage{geometry}
\usepackage{parskip}

% Page Geometry — Slim, Clean Margins
\geometry{
  a4paper,
  left=20mm,
  right=20mm,
  top=20mm,
  bottom=20mm
}

% Header & Footer Styling
\pagestyle{fancy}
\fancyhf{}
\fancyhead[L]{\small \textit{\nouppercase{\leftmark}}}
\fancyhead[R]{\small L501: Legal Environment of Business}
\fancyfoot[C]{\small \thepage}
\renewcommand{\headrulewidth}{0.3pt}
\renewcommand{\footrulewidth}{0.3pt}

% Chapter Title Styling
\titleformat{\chapter}[block]
  {\normalfont\Huge\bfseries}
  {\thechapter.}{12pt}{}

\titleformat{\section}
  {\normalfont\Large\bfseries}
  {\thesection}{1em}{}

% Table of Contents Styling
\renewcommand{\cftchapfont}{\bfseries}
\renewcommand{\cftsecfont}{}
\setlength{\cftbeforechapskip}{5pt}
\setlength{\cftbeforesecskip}{2pt}
\setlength{\cftaftertoctitleskip}{1em}

% Hyperlink Styling
\hypersetup{
    colorlinks=true,
    linkcolor=blue,
    urlcolor=blue,
    pdftitle={L501: Legal Environment of Business},
    pdfauthor={Md Hasibul Islam},
    pdfpagemode=FullScreen,
}

% Custom Command for Notes 
\newcommand{\notesection}[2]{
  \section*{#1\\ \small \textit{#2}}
  \phantomsection
  \addcontentsline{toc}{section}{#1 - #2}
}

% Document Start 
\begin{document}

% Title Page
\begin{titlepage}
    \centering
    \vspace*{3.5cm}
    \includegraphics[width=0.28\textwidth]{logo.png}\par\vspace{1.5cm}
    {\scshape\LARGE University of Dhaka\par}
    \vspace{0.5cm}
    {\Large Institute of Business Administration (IBA)\par}
    \vspace{1.5cm}
    {\Huge\bfseries Master of Business Administration (MBA)\par}
    \vspace{1cm}
    {\Large W501: \textit{Legal Environment of Business}\par}
    \vfill
    {\large Last Updated: \today\par}
\end{titlepage}

% Author Details Section 
\section*{Author Details}
\phantomsection
\addcontentsline{toc}{section}{Author Details}

\begin{center}
    \vspace{1em}
    \begin{tabular}{lll}
        \textbf{Name} & : & Md Hasibul Islam \\
        \textbf{Student ID} & : & 201-67-011 \\
        \textbf{Program} & : & Master of Business Administration (MBA) \\
        \textbf{Institute} & : & Institute of Business Administration (IBA) \\
        \textbf{University} & : & University of Dhaka \\
        \textbf{Email} & : & \href{mailto:hasiee8004@gmail.com}{hasiee8004@gmail.com} \\
        \textbf{LinkedIn} & : & \href{https://www.linkedin.com/in/hasib009}{linkedin.com/in/hasib009} \\
        \textbf{GitHub} & : & \href{https://github.com/HasibRockie}{github.com/HasibRockie} \\
        \textbf{Website} & : & \href{https://hasibrockie.github.io}{hasibrockie.github.io} \\
    \end{tabular}
    \vspace{1em}
\end{center}

\clearpage

% Table of Contents
\tableofcontents
\clearpage

% Notes Sections
\notesection{Proposal, Offer, Promise, Acceptance – Definitions, Differences \& Revocation}{21-04-25 Monday}

\textbf{1. Proposal (Offer):}  
According to Section 2(a) of the Indian Contract Act, 1872,  
\textit{“When one person signifies to another his willingness to do or to abstain from doing anything, with a view to obtaining the assent of that other, he is said to make a proposal.”}  

In business law, "proposal" and "offer" are often used interchangeably. A valid proposal must be:
\begin{itemize}
    \item Communicated to the offeree.
    \item Made with intent to create legal relations.
    \item Definite and certain in terms.
\end{itemize}

\textbf{2. Promise:}  
As per Section 2(b), a proposal when accepted becomes a promise.  
\textit{“When the person to whom the proposal is made signifies his assent thereto, the proposal is said to be accepted. A proposal when accepted becomes a promise.”}  

Thus, \textbf{Promise = Proposal + Acceptance}

\textbf{3. Acceptance:}  
Acceptance must be:
\begin{itemize}
    \item Absolute and unqualified.
    \item Communicated by authorized means.
    \item Given within a reasonable time or stipulated time.
\end{itemize}

\textbf{4. Differences among Proposal, Offer, Promise \& Acceptance:}

\begin{center}
\begin{tabular}{|p{4cm}|p{6cm}|p{6cm}|}
\hline
\textbf{Term} & \textbf{Definition} & \textbf{Key Feature} \\
\hline
Proposal / Offer & Intention to form contract by doing/refraining from act. & Requires acceptance to be binding. \\
\hline
Promise & Accepted offer. & Legally binding. \\
\hline
Acceptance & Assent to the proposal. & Converts proposal into a contract. \\
\hline
\end{tabular}
\end{center}

\textbf{5. Revocation of Proposal and Acceptance:}  
\begin{itemize}
    \item \textbf{Section 5 of the Indian Contract Act:} A proposal may be revoked at any time before the communication of its acceptance is complete as against the proposer.
    \item An acceptance may be revoked at any time before the communication of the acceptance is complete as against the acceptor.
\end{itemize}

\textbf{Examples (Proposal or Not?):}  
Below are 20+ situations analyzed to determine if they constitute a proposal:

\begin{enumerate}
    \item A says, "I will sell my bike to you for \$500" – \textbf{Proposal}.
    \item An advertisement for a sale on TV – \textbf{Not a proposal}, it's an \textit{invitation to offer}.
    \item A menu card in a restaurant – \textbf{Not a proposal}.
    \item Display of goods in a shop with a price tag – \textbf{Not a proposal}.
    \item A says, "Will you buy my car for \$5,000?" – \textbf{Proposal}.
    \item A tells B, “I may consider selling my house” – \textbf{Not a proposal}.
    \item Public auction notice – \textbf{Invitation to offer}.
    \item A quote or tender submission – \textbf{Proposal}.
    \item B submits a bid in auction – \textbf{Proposal}.
    \item A company issues a prospectus – \textbf{Invitation to offer}.
    \item A person asks, “Can you lend me your book?” – \textbf{Not a legal proposal}.
    \item X offers Y a job with fixed terms – \textbf{Proposal}.
    \item A general statement: “I wish someone would buy my house” – \textbf{Not a proposal}.
    \item A mails a signed letter offering to sell goods – \textbf{Proposal}.
    \item A unilateral offer of reward for finding a lost pet – \textbf{Proposal}.
    \item Online listing of a laptop on eBay – \textbf{Proposal} (depending on platform rules).
    \item Negotiation discussion: “I might accept \$1000 for this item” – \textbf{Not a proposal}.
    \item A says, “If you pay today, I will give a discount” – \textbf{Proposal}.
    \item A invitation to tender (e.g., govt contract) – \textbf{Invitation to offer}.
    \item B’s submission of the tender form – \textbf{Proposal}.
    \item A says “Let’s form a partnership” – \textbf{Proposal}.
    \item A makes a standing offer to supply items – \textbf{Proposal}.
\end{enumerate}

\textbf{Note:} The difference between \textit{“invitation to offer”} and an actual \textbf{proposal} is crucial. An invitation invites others to make offers, whereas a proposal is the starting point of contractual intent.

\textbf{Conclusion:}  
Understanding the conceptual differences between proposal, promise, and acceptance is crucial in legal and business transactions. Clear communication, intent to be bound, and adherence to legal formalities determine the enforceability of contracts. Revocation rights add further complexity and protection for both parties involved.



\textbf{Definition of Contract:}  
According to Section 2(h) of the Indian Contract Act, 1872:  
\textit{“A contract is an agreement enforceable by law.”}  
Thus, for any agreement to be a contract, it must fulfill these essentials:
\begin{itemize}
    \item Lawful offer and lawful acceptance.
    \item Intention to create legal relationship.
    \item Lawful consideration.
    \item Capacity of parties.
    \item Free consent.
    \item Lawful object.
\end{itemize}

\vspace{0.4cm}
\textbf{Below are 20+ examples examined to determine if they constitute contracts:}

\begin{enumerate}
    \item A agrees to sell B his house for 50 lakhs. – \textbf{Contract} (meets all essentials).
    \item A invites B for dinner. – \textbf{Not a Contract} (no legal intention).
    \item A promises to give B 1 lakh out of love. – \textbf{Not a Contract} (no consideration).
    \item B offers A 10,000 for his phone, A accepts. – \textbf{Contract}.
    \item X and Y agree to smuggle goods across borders. – \textbf{Not a Contract} (illegal object).
    \item A minor agrees to buy a laptop on EMI. – \textbf{Not a Contract} (incapacity of minor).
    \item A promises B to marry her without any intention of doing so. – \textbf{Not a Contract} (no real intention).
    \item A agrees to supply 100 chairs to B at 500 per chair. – \textbf{Contract}.
    \item A enters into agreement under threat. – \textbf{Not a Contract} (absence of free consent).
    \item A wins a lottery, government promises to pay. – \textbf{Contract} (in case of valid government scheme).
    \item A advertises a reward for lost pet, B finds it and claims reward. – \textbf{Contract}.
    \item A agrees to work for B without salary. – \textbf{Not a Contract} (no consideration unless a gift of service is intended and accepted).
    \item A contracts B to kill a person. – \textbf{Not a Contract} (unlawful object).
    \item Verbal agreement to sell a cow for 20,000, accepted and paid. – \textbf{Contract}.
    \item Casual promise at a party to lend money. – \textbf{Not a Contract} (no legal intention).
    \item A enters into contract with a person of unsound mind. – \textbf{Not a Contract}.
    \item A hires B to paint his house for 15,000. – \textbf{Contract}.
    \item A and B mutually agree to form a business with capital investment. – \textbf{Contract}.
    \item A bets 1000 on a cricket match with B. – \textbf{Not a Contract} (void under wagering agreements).
    \item A contracts with B to deliver goods, consideration given. – \textbf{Contract}.
    \item A bribes an officer to pass a tender. – \textbf{Not a Contract} (against public policy).
    \item A writes a letter offering services to B, B accepts. – \textbf{Contract}.
    \item A sells his bicycle to B for 3,000, B pays instantly. – \textbf{Contract}.
    \item A and B agree to commit cyber fraud. – \textbf{Not a Contract} (unlawful purpose).
    \item A sells a book to B with mutual consent and price. – \textbf{Contract}.
\end{enumerate} 

\textbf{Key Insights:}
\begin{itemize}
    \item A \textbf{social or moral agreement} is not enforceable (e.g., dinner invitations).
    \item \textbf{Illegal or immoral agreements} are void ab initio (from the beginning).
    \item \textbf{Contracts with minors or mentally unsound persons} are generally void.
    \item \textbf{Consideration} and \textbf{legal object} are core components.
\end{itemize}

\textbf{Conclusion:}  
All contracts are agreements, but not all agreements are contracts. The key lies in legal enforceability. Every case must be assessed against the core criteria of a valid contract to determine its legitimacy and enforceability under law.

\vspace{1cm} 

\textbf{Chapter 2: Offer and Acceptance Exercise Problems} 
\vspace{0.5cm} 

\textbf{1. When is an offer completed? How and when may an offer be revoked?}  

An offer is completed when it is communicated to the offeree (Sec. 4, Indian Contract Act).  
\textbf{Revocation of an offer} (Sec. 5) is valid if:
\begin{itemize}
    \item It is done before the communication of acceptance is complete against the proposer.
    \item Revocation must be communicated to the offeree.
\end{itemize}

\textit{Example:} A offers to sell a bike to B. If B posts an acceptance letter and A sends a revocation after that, the revocation is invalid.

\vspace{0.5cm}
\textbf{2. (a) How may an offer be terminated?}

An offer can be terminated in the following ways:
\begin{itemize}
    \item Revocation before acceptance.
    \item Rejection by offeree.
    \item Lapse of time.
    \item Death or insanity of offeror.
    \item Non-fulfillment of a condition precedent.
\end{itemize}

\textbf{(b)} A offers to sell B his horse for one thousand and tells B, "This offer will remain open for one week." B rejects the offer the next day. Later in the week, B changes his mind and accepts the offer.  
\textbf{No contract is formed.} Once the offer is rejected, it cannot be revived unless A makes the offer again.

\vspace{0.5cm}
\textbf{3. "Acceptance is to offer what a lighted match is to a train of gunpowder." Discuss.}  

This metaphor, coined by Anson, means that just as a lighted match completes the chain for an explosion, acceptance finalizes an offer into a contract. There’s no turning back once acceptance is made and communicated – the contract is formed.

\vspace{0.5cm}
\textbf{4. "An offer is made when and not until, it is communicated to the offeree".}  

Communication is key to a valid offer.  
\textit{Example:} A writes a letter offering B his car but never posts it. There is no offer.  
Unless the offeree is made aware, an offer cannot be accepted.

\vspace{0.5cm}
\textbf{5. Define offer and acceptance. When are they complete if made through post?}  

\textbf{Offer:} Sec. 2(a) – Willingness to do or abstain from an act to obtain assent.  
\textbf{Acceptance:} Sec. 2(b) – When the offeree signifies assent.

\textbf{Completion by post:}  
\begin{itemize}
    \item \textbf{Offer} is complete when received by offeree.
    \item \textbf{Acceptance} is complete as against proposer when it is posted, and against acceptor when received.
\end{itemize}

\vspace{0.5cm}
\textbf{6. Rules of Offer via Post and Telephone:}  

\textbf{By Post:}
\begin{itemize}
    \item Acceptance is complete when posted.
    \item Revocation of offer must reach before acceptance is posted.
\end{itemize}

\textbf{By Telephone:}
\begin{itemize}
    \item Acceptance must be clearly heard and understood.
    \item No contract if communication fails or is unclear.
\end{itemize}

\vspace{0.5cm}
\textbf{7. "A mere mental acceptance, not evidenced by words or conduct is in the eye of law no acceptance."}  

True. Acceptance must be communicated. Thinking of accepting is not sufficient.  
\textit{Example:} A decides to accept an offer but doesn’t inform B. No contract arises.

\vspace{0.5cm}
\textbf{8. Define 'Acceptance'. Essentials of Valid Acceptance:}  

Sec. 2(b): Acceptance is the act of signifying assent.  
\textbf{Essentials:}
\begin{itemize}
    \item Must be absolute and unconditional.
    \item Must be communicated.
    \item Must be in the prescribed mode.
    \item Must be made while the offer is still open.
\end{itemize}

\vspace{0.5cm}
\textbf{9. "Acceptance must be absolute, and must correspond with the terms of the offer."}  

This is the \textbf{mirror image rule}.  
\textit{Example:} A offers to sell a product for ten thousand. B replies, “I accept for nine thousand.” → Counter-offer, not acceptance.

\vspace{0.5cm}
\textbf{10. (a) Meaning of Offer and Acceptance:}  

Already defined above in Q5.

\textbf{(b)(i)} A offers to sell goods by letter on March 1. B receives it on March 3.  
A can revoke before B posts acceptance. So yes, A can revoke before March 4.

\textbf{(ii)} B posts acceptance on March 4. A receives it on March 6.  
B cannot revoke after posting. So, \textbf{No}, B cannot revoke acceptance.

\vspace{0.5cm}
\textbf{11. (a) Define Proposal:}  
As per Sec. 2(a), willingness to do or abstain from an act with a view to obtaining assent.

\textbf{(b) Communication of Offer:}
\begin{itemize}
    \item Must reach the offeree.
    \item Must be clear and specific.
\end{itemize}

\vspace{0.5cm}
\textbf{12. Objective Questions:}

\textbf{(a)} Acceptance by conduct: Acceptance shown by action.  
\textit{Example:} Taking a train ticket implies acceptance of the travel terms.

\textbf{(b)} An advertisement to sell a thing by auction is:  
\textbf{Answer: (b) An invitation to offer.}

\vspace{0.5cm}
\textbf{13. Problems:}

\textbf{(a)} A proposes via letter. B accepts by letter.  
\textbf{A can revoke} before B posts acceptance.  
\textbf{B can revoke} before A receives acceptance.

\textbf{(b)} X’s revocation reaches Y before the offer does.  
\textbf{No valid offer} reached Y, hence no contract.

\textbf{(c)} B can revoke acceptance before it reaches A.  
So, if revocation is communicated before delivery, it is valid.

\textbf{(d)} A offers a reward for doing a task. B does the task without knowledge of the reward.  
\textbf{No contract.} Knowledge of offer is essential.

\textbf{(e)} A posts acceptance, but it's lost.  
\textbf{Contract is still valid.} Acceptance is complete upon posting.

\vspace{1cm}
\clearpage


\notesection{Consideration}{24-04-25 Thursday}

\textbf{Definition of Consideration:}  
According to Section 2(d) of the Indian Contract Act,  
\textit{“When, at the desire of the promisor, the promisee or any other person has done or abstained from doing, or does or abstains from doing, or promises to do or abstain from doing something, such act or abstinence or promise is called a consideration for the promise.”}

\textbf{In Simple Words:}  
Consideration is something of value given by both parties to a contract that induces them to enter into the agreement. It may be money, a service, a promise, or even not doing something.

\vspace{0.3cm}
\textbf{Key Points:}
\begin{itemize}
    \item It must move at the desire of the promisor.
    \item It may move from the promisee or any other person.
    \item It may be past, present or future.
    \item It must be something of value in the eyes of law.
\end{itemize}

\vspace{0.5cm}
\textbf{Rules of Consideration:}
\begin{enumerate}
    \item Consideration must move at the desire of the promisor.
    \item It may move from the promisee or any third party.
    \item It may be past, present or future.
    \item It must be real and not illusionary.
    \item It need not be adequate, but must be lawful.
    \item Consideration must be something that the law can regard as valuable.
\end{enumerate}

\vspace{0.5cm}
\textbf{Exceptions to the Rule: “No Consideration, No Contract” (Sec. 25)}  
Normally, a contract without consideration is void. But there are exceptions:
\begin{itemize}
    \item \textbf{Natural love and affection:} Between close relations if made in writing and registered.
    \item \textbf{Compensation for past voluntary service:} If a person voluntarily does something and the other person promises to pay later.
    \item \textbf{Promise to pay a time-barred debt:} If made in writing and signed by the debtor.
    \item \textbf{Completed gifts:} Gifts already made are valid even without consideration.
    \item \textbf{Agency agreements:} No consideration is necessary to create an agency.
    \item \textbf{Contract under seal:} In English law, consideration is not needed for contracts made under seal.
\end{itemize}

\vspace{0.5cm}
\textbf{Examples of Consideration (15+):}
\begin{enumerate}
    \item A agrees to sell a house and B agrees to pay a price – mutual consideration.
    \item A promises to deliver goods and B promises to pay on delivery.
    \item A serves B’s company without pay, later B promises to pay – past consideration.
    \item A gives tuition to B’s son, B promises to pay later.
    \item A gives up smoking at B’s request, B promises to pay.
    \item A does not file a legal case against B, B agrees to pay.
    \item A lends B a book, B promises to return or replace it.
    \item A promises to pay for goods received in the past.
    \item A cleans B’s garden for free; B later promises a gift – voluntary act.
    \item A supports B during illness, B later promises payment.
    \item A marries with B’s daughter; B promises a house – valid if written and registered.
    \item A saves B from drowning, B promises a reward – past voluntary service.
    \item A owes B money; B forgives the debt if A transfers a property – consideration.
    \item A agrees to give B a job, B agrees to pay fees – reciprocal promises.
    \item A agrees not to open a competing shop near B, B pays for that promise.
    \item A provides free legal advice, B promises to gift a car – no consideration.
    \item A promises to donate to a school, does not pay – generally unenforceable.
\end{enumerate}

\vspace{0.5cm}
\textbf{Stranger to Consideration vs. Stranger to Contract}

\textbf{Stranger to consideration:} A person who has not provided consideration can still enforce the contract, if done on behalf of someone.  
\textbf{Stranger to contract:} A person who is not a party to the contract cannot sue upon it.

\vspace{0.3cm}
\textbf{Exceptions to Stranger to Contract Rule:}
\begin{itemize}
    \item \textbf{Trust:} Beneficiaries of trust can enforce the contract.
    \item \textbf{Family settlements:} Members receiving benefit can sue.
    \item \textbf{Agency:} Principal can sue on agent’s contract.
    \item \textbf{Assignment:} Assignee can sue even if not an original party.
    \item \textbf{Covenants running with land:} Successors can enforce.
\end{itemize}

\vspace{0.3cm}
\textbf{Examples of Stranger to Contract and Exceptions:}
\begin{enumerate}
    \item A makes a contract with B to pay C – C cannot sue (general rule).
    \item A (father) agrees to pay C (daughter-in-law) under a family arrangement – C can sue.
    \item A contracts with B to pay debt to C – if C is a trust beneficiary, C can sue.
    \item A promises to pay B on behalf of C. If C is the principal, he can sue via agent rule.
    \item A assigns his rights in a contract with B to C – C can sue B.
    \item Landlord promises tenant to repair property, tenant’s sub-tenant may not sue directly.
\end{enumerate}

\vspace{0.5cm}

\textbf{1. Define consideration. Critically discuss the essential elements of consideration.}

\textbf{Definition:}  
Section 2(d) of the Indian Contract Act defines consideration as:  
\textit{"When, at the desire of the promisor, the promisee or any other person has done or abstained from doing... something, such act... is called a consideration for the promise."}

\textbf{Essential Elements:}
\begin{enumerate}
    \item \textbf{Desire of the promisor:} The act or forbearance must be at the promisor’s request.
    \item \textbf{From promisee or any other person:} Indian law allows a stranger to consideration.
    \item \textbf{Past, present or future:} Indian law recognizes all three types.
    \item \textbf{Lawful consideration:} It must not be illegal, immoral or opposed to public policy.
    \item \textbf{Something of value:} Even if not adequate, it must be real and tangible in law.
\end{enumerate}

\vspace{0.3cm}
\textbf{Critically:} Indian law is more flexible than English law in accepting past and third-party consideration.

\vspace{0.5cm}
\textbf{2. "Past consideration is no consideration." – Comment.}

\textbf{English Law:} Past consideration is generally invalid. A promise to reward a past act is unenforceable unless a prior request existed.

\textbf{Indian Law:} Past consideration is valid. If someone does something voluntarily or at the promisor’s request and the promisor later promises to pay, it is enforceable.

\textbf{Example:} A saves B’s goods from fire. Later B promises to pay A. In India, this is a valid contract. In English law, it may not be.

\vspace{0.5cm}
\textbf{3. Define consideration and point out the differences between English law and Indian law in this respect.}

\textbf{Definition:} (Same as Q1)

\textbf{Key Differences:}\\
\begin{tabular}{|p{0.3\linewidth}|p{0.3\linewidth}|p{0.3\linewidth}|}
\hline
\textbf{Point} & \textbf{Indian Law} & \textbf{English Law} \\
\hline
Past Consideration & Valid & Invalid \\
\hline
Third-party Consideration & Valid (any person can furnish) & Invalid (only promisee must provide) \\
\hline
Adequacy of Consideration & Immaterial & Immaterial \\
\hline
Written/Oral Contract & Valid in both forms & Valid in both forms \\
\hline
\end{tabular}

 
\vspace{0.5cm}
\textbf{4. "Insufficiency of consideration is immaterial; but an agreement without consideration is void." – Explain.}

\textbf{Explanation:} The law requires that there must be some consideration, but it need not be equal or adequate to the promise. Even a nominal consideration is sufficient.

\textbf{Example:} A sells his laptop worth 50,000 for 5,000. The court will not question the fairness if both parties agreed.

\textbf{Void without consideration:} If nothing is given or promised in return, the contract is usually void, unless it falls under exceptions (see Q5).

\vspace{0.5cm}
\textbf{5. Circumstances in which a contract without consideration may be valid}

Under Section 25 of the Indian Contract Act, an agreement made without consideration is void unless it falls within certain well-defined exceptions:

\begin{enumerate}
    \item \textbf{Natural love and affection:} If the agreement is between parties standing in a near relationship, made out of natural love and affection, and is in writing and registered, it is valid without consideration.

    \textit{Example:} A father promises to transfer property to his daughter by a registered document, out of love and affection. This is enforceable.

    \item \textbf{Past voluntary services:} If a person has voluntarily done something for another and the latter promises to compensate, that promise is enforceable.

    \textit{Example:} A saves B’s drowning child. Later, B promises to pay A for his brave act. This is enforceable in India.

    \item \textbf{Promise to pay time-barred debt:} A written and signed promise to pay a debt that is barred by the Limitation Act is valid.

    \textit{Example:} A owes B a sum of money, which becomes time-barred. A later promises in writing to repay. This is enforceable.

    \item \textbf{Completed gift:} Once a gift is made and delivered voluntarily, it does not require consideration.

    \textit{Example:} A donates a laptop to B. Later, A cannot demand it back on the ground of lack of consideration.

    \item \textbf{Agency:} According to Section 185 of the Indian Contract Act, no consideration is necessary to create an agency relationship.

    \textit{Example:} A authorizes B to act as his agent without any payment. This is valid.

    \item \textbf{Charitable subscriptions:} If a person promises to contribute to a charitable cause and on the faith of that promise, the promisee incurs liability, the promise is enforceable.

    \textit{Example:} A promises Rs. 10,000 to build a hospital. The hospital starts construction relying on it. A is bound to pay.
\end{enumerate}

\vspace{0.5cm}

\textbf{6. Stranger to a contract cannot sue – Rule and Exceptions}

\textbf{General Rule:}  
Only parties to a contract can sue to enforce it. A stranger to the contract has no *locus standi* (legal standing), even if the contract is made for his benefit.

\textbf{Exceptions:}
\begin{enumerate}
    \item \textbf{Trust:} A beneficiary can enforce rights under a trust created in his favor.

    \textit{Example:} X contracts with Y to hold property in trust for Z. Z can sue to enforce this trust.

    \item \textbf{Family arrangements:} In joint families, agreements made for the benefit of family members can be enforced by them.

    \textit{Example:} An elder brother agrees with others to provide for the marriage of his sister. The sister can enforce it.

    \item \textbf{Agency:} A principal can sue third parties for contracts entered into by his agent.

    \textit{Example:} A agent buys goods from B on behalf of C. C can sue B.

    \item \textbf{Assignment:} The assignee of a contractual right can sue in his own name.

    \textit{Example:} A assigns his right to receive rent to B. B can sue the tenant.

    \item \textbf{Covenants running with land:} The purchaser of immovable property may sue on covenants attached to it.

    \textit{Example:} A leases land to B with a covenant to repair a wall. B sells to C. C can sue if the wall is not repaired.
\end{enumerate}

\vspace{0.5cm}
\textbf{7. A stranger to the consideration may sue on a contract but not a stranger to the contract. - Explain.}

In Indian law, it is not necessary that consideration should move from the promisee. A third party (not being a stranger to the contract) can provide consideration.

\textit{Example:} A agrees to pay B if C delivers goods to B. C delivers, but A refuses to pay. B can sue A even though C gave the consideration.

But if someone is not a party to the contract at all (a stranger to contract), they cannot sue—even if they benefit from it.

\vspace{0.5cm}
\textbf{8. "A stranger to a contract cannot sue to enforce the contract." Discuss.}

This is a general principle of "privity of contract." Only parties who have entered into a contract can enforce it. However, this rule has notable exceptions, as discussed above (See Q6).

\textbf{Legal Justification:} Enforcing rights requires reciprocal obligations. A third party has not undertaken any obligation.

\vspace{0.5cm}
\textbf{9.}

(a) \textbf{Meaning of Consideration:} See Q1.

(b) \textbf{Valid agreements without consideration:}
\begin{itemize}
    \item Promise out of natural love and affection.
    \item Past voluntary services.
    \item Promise to pay time-barred debt.
    \item Completed gifts.
    \item Creation of agency.
\end{itemize}

\textit{Example:} A promises to pay B who saved his life. Valid under past voluntary service.

\vspace{0.5cm}
\textbf{10. When can a non-party sue a contract?}

Refer to the five exceptions in Q6. A non-party can sue:
\begin{itemize}
    \item If they are a trust beneficiary,
    \item A family settlement beneficiary,
    \item A principal in an agency relationship,
    \item An assignee,
    \item Or if covenants run with land.
\end{itemize}

\vspace{0.5cm}
\textbf{11.}

(a) \textbf{Definition:} (See Q1)

\textbf{Elements:}
\begin{itemize}
    \item Must move at promisor’s desire.
    \item May be past, present, or future.
    \item Can move from promisee or any third party.
    \item Must be lawful.
\end{itemize}

(b) \textbf{Agreement valid without consideration:} (See Q5)

\vspace{0.5cm}
\textbf{12. "An agreement without consideration is void unless it is in writing and registered." – Explain.}

This refers to Section 25(1) of the Indian Contract Act. A promise without consideration can be valid if:
\begin{itemize}
    \item It is made out of natural love,
    \item Between close relatives,
    \item In writing and registered.
\end{itemize}

\textit{Example:} A registered gift deed by a father to son is enforceable.

\vspace{0.5cm}
\textbf{13.}

(a) \textbf{Essential Factors of Consideration:}
\begin{itemize}
    \item Must be at promisor’s request.
    \item Can be past, present, or future.
    \item Must be lawful.
    \item Need not be adequate, but must be real.
\end{itemize}

(b) \textbf{Promise to pay time-barred debt:}  
Yes, valid. Section 25(3) allows a written and signed promise to revive time-barred debts.

\textit{Example:} A writes and signs a note promising to repay B a debt that is time-barred. This is enforceable.

\vspace{0.5cm}
\textbf{14. Objective Questions}

(i) \textbf{Two examples of valid contracts without consideration:}
\begin{itemize}
    \item Promise to pay time-barred debt (written and signed).
    \item Agreement out of natural love and affection (written and registered).
\end{itemize}

(ii) \textbf{Two exceptions to privity of contract rule:}
\begin{itemize}
    \item Beneficiary under a trust.
    \item Member in a family settlement.
\end{itemize}

\vspace{1cm}

\textbf{Case Title:} \textbf{Chinnaya v. Ramaya (1882) ILR 4 Mad 137}

\textbf{Facts of the Case:}  
A lady (the promisor) gifted some landed property to her daughter (the promisee). As a condition of the gift, the daughter was required to pay an annuity to the donor's brother. The daughter accepted the gift and agreed to make the payment. However, she subsequently refused to pay the annuity. The brother sued the daughter to enforce the payment.

\textbf{Legal Issue:}  
Can a person who is not a party to a contract (a stranger to the contract) but has furnished consideration, sue to enforce the contract?

\textbf{Arguments:}  
- The defendant (daughter) argued that there was no direct contract between her and the plaintiff (the brother), and hence, the plaintiff had no legal right to sue.
- The plaintiff argued that the gift deed and the condition to pay annuity were part of one transaction, and he was entitled to enforce the condition.

\textbf{Judgment:}  
The Madras High Court held in favour of the brother. The court ruled that although the brother was not a party to the contract between the lady and her daughter, the contract was made for his benefit, and he had furnished the consideration. Therefore, he could enforce the promise.

\textbf{Legal Principle Established:}  
The case laid down an important exception to the rule of privity of contract: A person who is a beneficiary under a contract and has furnished consideration (even if not a direct party to the contract) may sue to enforce the contract.

\textbf{Importance:}  
- Supports the Indian position that consideration may move from a person other than the promisee.
- Distinguishes Indian law from strict English law which does not allow such claims.
- Reinforces that a “stranger to the contract” cannot sue, but a “stranger to consideration” may sue in India.

\textbf{Key Learning:}
\begin{itemize}
    \item Indian Contract Law is more flexible than English Law regarding consideration.
    \item A third party can enforce a contract if the contract was made for their benefit and consideration is furnished.
    \item This case is an important authority in exceptions to the doctrine of privity of contract in India.
\end{itemize}

\textbf{Relevant Section:}  
Section 2(d) of the Indian Contract Act, 1872 – "When, at the desire of the promisor, the promisee or any other person has done or abstained from doing... such act is called consideration."

\vspace{1cm}
\clearpage

\textbf{Case Title:} \textbf{Alka Bose v. Parmatma Devi \& Ors (2009) 2 SCC 582}

\textbf{Facts of the Case:}  
Alka Bose (plaintiff) entered into an oral agreement with Parmatma Devi (defendant) for the purchase of a house property. The agreement was not registered or formally documented, but the plaintiff contended that possession had been delivered, and part payment had been made. When the seller later refused to honour the sale, the plaintiff sued for specific performance of the contract.

\textbf{Legal Issue:}  
Can an oral contract for sale of immovable property be specifically enforced in court when no written agreement exists?

\textbf{Arguments:}
- The plaintiff argued that the oral agreement was valid under the Indian Contract Act and that part performance had taken place, making it enforceable.
- The defendant argued that without a written and registered agreement, the contract could not be enforced under the Transfer of Property Act, 1882 and Registration Act, 1908.

\textbf{Judgment:}  
The Supreme Court held that although an agreement to sell immovable property is required to be in writing under Section 54 of the Transfer of Property Act, **an oral agreement is not invalid** per se under the Indian Contract Act. However, **specific performance** of such an agreement will not be granted unless the terms are certain and unambiguous, and the existence of the agreement can be proved by **evidence of part performance, conduct, or other corroborating factors**.

\textbf{Legal Principle Established:}  
- Oral contracts are valid under Indian law.
- Specific performance of oral contracts may be allowed where the contract is clearly proven and part performance (such as possession or payment) is demonstrated.
- However, for immovable property transactions, written agreements are strongly advisable due to evidentiary and statutory requirements.

\textbf{Importance:}  
This case demonstrates the judicial flexibility in India regarding oral contracts, while also highlighting the practical and evidentiary challenges involved.

\textbf{Key Learning:}
\begin{itemize}
    \item Oral agreements are valid and enforceable unless specifically barred by law.
    \item Courts will examine evidence of conduct, part performance, and third-party corroboration.
    \item In real estate, written agreements are safer and legally stronger due to statutory requirements.
\end{itemize}

\textbf{Relevant Laws:}
\begin{itemize}
    \item Section 10, Indian Contract Act, 1872 – Essentials of a valid contract.
    \item Section 54, Transfer of Property Act, 1882 – Sale of immovable property.
    \item Section 17, Registration Act, 1908 – Mandatory registration of certain documents.
\end{itemize}

\vspace{1cm}

\clearpage


% new note 
\notesection{2025-04-23}{Wednesday}

\vspace{1cm}
\clearpage

% end of the note 


\end{document}
