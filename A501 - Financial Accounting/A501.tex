\documentclass[12pt,a4paper]{book}

% Fonts & Typography — Elegant and Professional
\usepackage[T1]{fontenc}
\usepackage{kpfonts} % Sleek, modern font
\usepackage{microtype}

% Essential Packages
\usepackage{graphicx}
\usepackage{fancyhdr}
\usepackage{tocloft}
\usepackage{titlesec}
\usepackage{datetime}
\usepackage{hyperref}
\usepackage{geometry}
\usepackage{parskip}
\usepackage{array, booktabs}
\usepackage{amsmath}
\usepackage{multicol}
\usepackage{longtable}
\usepackage{footnote}
\usepackage{multicol}

% Page Geometry — Slim, Clean Margins
\geometry{
  a4paper,
  left=20mm,
  right=20mm,
  top=20mm,
  bottom=20mm
}

% Header & Footer Styling
\pagestyle{fancy}
\fancyhf{}
\fancyhead[L]{\small \textit{\nouppercase{\leftmark}}}
\fancyhead[R]{\small A501: Financial Accounting} 
\fancyfoot[C]{\small \thepage}
\renewcommand{\headrulewidth}{0.3pt}
\renewcommand{\footrulewidth}{0.3pt}

% Chapter Title Styling
\titleformat{\chapter}[block]
  {\normalfont\Huge\bfseries}
  {\thechapter.}{12pt}{}

\titleformat{\section}
  {\normalfont\Large\bfseries}
  {\thesection}{1em}{}

% Table of Contents Styling
\renewcommand{\cftchapfont}{\bfseries}
\renewcommand{\cftsecfont}{}
\setlength{\cftbeforechapskip}{5pt}
\setlength{\cftbeforesecskip}{2pt}
\setlength{\cftaftertoctitleskip}{1em}

% Hyperlink Styling
\hypersetup{
    colorlinks=true,
    linkcolor=blue,
    urlcolor=blue,
    pdftitle={A501: Financial Accounting},
    pdfauthor={Md Hasibul Islam},
    pdfpagemode=FullScreen,
}

% Custom Command for Notes 
\newcommand{\notesection}[2]{
  \section*{#1\\ \small \textit{#2}}
  \phantomsection
  \addcontentsline{toc}{section}{#1 - #2}
}

% Document Start 
\begin{document}

% Title Page
\begin{titlepage}
    \centering
    \vspace*{3.5cm}
    \includegraphics[width=0.28\textwidth]{logo.png}\par\vspace{1.5cm}
    {\scshape\LARGE University of Dhaka\par}
    \vspace{0.5cm}
    {\Large Institute of Business Administration (IBA)\par}
    \vspace{1.5cm}
    {\Huge\bfseries Master of Business Administration (MBA)\par}
    \vspace{1cm}
    {\Large A501: \textit{Financial Accounting}\par}
    \vfill
    {\large Last Updated: \today\par}
\end{titlepage}

% Author Details Section 
\section*{Author Details}
\phantomsection
\addcontentsline{toc}{section}{Author Details}

\begin{center}
    \vspace{1em}
    \begin{tabular}{lll}
        \textbf{Name} & : & Md Hasibul Islam \\
        \textbf{Student ID} & : & 201-67-011 \\
        \textbf{Program} & : & Master of Business Administration (MBA) \\
        \textbf{Institute} & : & Institute of Business Administration (IBA) \\
        \textbf{University} & : & University of Dhaka \\
        \textbf{Email} & : & \href{mailto:hasiee8004@gmail.com}{hasiee8004@gmail.com} \\
        \textbf{LinkedIn} & : & \href{https://www.linkedin.com/in/hasib009}{linkedin.com/in/hasib009} \\
        \textbf{GitHub} & : & \href{https://github.com/HasibRockie}{github.com/HasibRockie} \\
        \textbf{Website} & : & \href{https://hasibrockie.github.io}{hasibrockie.github.io} \\
    \end{tabular}
    \vspace{1em}
\end{center}

\clearpage

% Table of Contents
\tableofcontents
\clearpage 

% Notes Sections
\notesection{Definition and Characteristics: Accounting and Transaction}{10-04-25 Thursday}

\textbf{Accounting} is the systematic process of identifying, recording, classifying, summarizing, and communicating financial information to aid economic decision-making. It serves as the language of business and provides a framework for understanding a firm's financial position and performance.

\textbf{Key Characteristics of Accounting:}
\begin{itemize}
    \item \textbf{Systematic}: Follows a standard process and principles (GAAP or IFRS).
    \item \textbf{Historical and Predictive}: Reflects past events and supports future planning.
    \item \textbf{Quantitative}: Concerned primarily with monetary information.
    \item \textbf{Reliable and Verifiable}: Based on evidence (e.g., invoices, receipts).
    \item \textbf{Comparable and Consistent}: Enables year-on-year or firm-to-firm comparison.
\end{itemize}

\textbf{Example:} A company sells a product for \$1,000. The sale is recorded in the accounting books under revenue (income), and cash or receivable is increased by \$1,000.

\textbf{Transaction:} Any business event that has a financial impact on the entity and can be reliably measured.

\textbf{Characteristics of a Transaction:}
\begin{itemize}
    \item \textbf{Financial Impact}: Affects assets, liabilities, equity, income, or expenses.
    \item \textbf{Dual Aspect}: Every transaction has a debit and a credit effect.
    \item \textbf{Measurable in Money}: Must be quantifiable in monetary terms.
\end{itemize}

\textbf{Example:} Paying salary of \$5,000 – decreases cash (asset) and increases expense.

\vspace{0.5cm}

\textbf{Examples: Is it a Transaction?}
\begin{itemize}
    \item \textbf{Purchased machinery for cash} – \textit{Yes, it affects assets (machinery and cash)}.
    \item \textbf{Signed a contract to deliver goods next month (no advance received)} – \textit{No, no financial impact yet}.
    \item \textbf{Paid salaries to employees} – \textit{Yes, it reduces cash and increases expenses}.
    \item \textbf{Owner invests capital into the business} – \textit{Yes, it increases assets and equity}.
    \item \textbf{Employee promoted, no monetary change} – \textit{No, it is not measurable in monetary terms}.
    \item \textbf{Received utility bill (not yet paid)} – \textit{Yes, it increases liabilities and expenses}.
    \item \textbf{Customer places an order (no payment yet)} – \textit{No, it's a future event with no current financial effect}.
    \item \textbf{Sold goods on credit} – \textit{Yes, it increases accounts receivable and revenue}.
    \item \textbf{Declared dividend (not paid yet)} – \textit{Yes, it creates a liability}.
    \item \textbf{Paid electricity bill} – \textit{Yes, it decreases cash and increases expense}.
    \item \textbf{Depreciation of machinery recorded} – \textit{Yes, it increases expense and reduces asset value}.
    \item \textbf{Appointed new auditor} – \textit{No, unless fees are paid or accrued}.
\end{itemize}

\vspace{0.5cm}

\textbf{Types of Accounting Reports and Their Uses}

\begin{itemize}
    \item \textbf{Income Statement (Profit \& Loss Statement)}: Shows revenues and expenses over a period; indicates profitability.\\
    \textit{Use: Assess operational performance.}
    
    \item \textbf{Balance Sheet (Statement of Financial Position)}: Displays assets, liabilities, and owner’s equity at a specific point in time.\\
    \textit{Use: Understand financial position and solvency.}
    
    \item \textbf{Cash Flow Statement (Statement of Cashflows)}: Reports cash inflows and outflows from operating, investing, and financing activities.\\
    \textit{Use: Monitor liquidity and cash management.}
    
    \item \textbf{Statement of changes in Equity (Remained earned statement)}: Explains changes in equity from investments, withdrawals, and retained earnings.\\
    \textit{Use: Analyze changes in owner’s interest in the firm.}
\end{itemize}

\textbf{Characteristics of Good Accounting Reports:}
\begin{itemize}
    \item \textbf{Relevance}: Information must aid decision-making.
    \item \textbf{Faithful Representation}: Complete, neutral, and free from error.
    \item \textbf{Understandability}: Clear presentation for users.
    \item \textbf{Comparability}: Allows evaluation across periods and entities.
    \item \textbf{Timeliness}: Provided in time to be useful.
\end{itemize}

\vspace{0.5cm}
\textbf{Users of Accounting Information}

\begin{itemize}
    \item \textbf{Internal Users:}
    \begin{itemize}
        \item Managers – for planning and control.
        \item Employees – job security and performance.
        \item Owners – profit evaluation.
    \end{itemize}
    
    \item \textbf{External Users:}
    \begin{itemize}
        \item Investors – profitability and risk analysis.
        \item Creditors – repayment ability.
        \item Government – taxation and regulatory compliance.
        \item Customers – supplier stability.
        \item Regulatory Bodies – ensure legal conformity.
    \end{itemize}
\end{itemize}

\vspace{0.5cm}
\textbf{Examples of Elements in Accounting :}

\begin{tabular}{|p{4cm}|p{12cm}|}
\hline
\textbf{Category} & \textbf{Examples} \\
\hline
\textbf{Assets} & 
Cash, Bank Balance, Accounts Receivable, Inventory, Prepaid Insurance, Office Supplies, Furniture, Equipment, Buildings, Land, Vehicles, Patents, Copyrights, 
Trademarks, Goodwill, Long-term Investments, Marketable Securities \\
\hline
\textbf{Liabilities} & 
Accounts Payable, Notes Payable, Accrued Expenses (e.g., Wages Payable, Interest Payable), Unearned Revenue, Loans Payable, Bonds Payable, Taxes Payable, Deferred Revenue, Lease Obligations, Credit Card Payable, Mortgage Payable \\
\hline
\textbf{Owner’s Equity} & 
Capital, Retained Earnings, Drawings, Owner’s Contributions, Common Stock, Preferred Stock, Additional Paid-in Capital, Treasury Stock, Accumulated Other Comprehensive Income \\
\hline
\textbf{Income (Revenue)} & 
Sales Revenue, Service Revenue, Interest Income, Rental Income, Commission Revenue, Dividend Income, Consulting Revenue, Royalties Earned, Investment Income, Subscription Revenue, Gain on Sale of Asset, Foreign Exchange Gain \\
\hline
\textbf{Expenses} & 
Salaries and Wages Expense, Rent Expense, Utilities Expense, Insurance Expense, Depreciation Expense, Amortization Expense, Interest Expense, Advertising Expense, Supplies Expense, Repairs and Maintenance Expense, Legal Fees, Audit Fees, Delivery Expense, Telephone Expense, Training Expense, Bad Debts Expense, Loss on Disposal of Asset \\
\hline
\end{tabular}

\vspace{0.5cm}

\textbf{Sub-types of Owner’s Equity:}

\begin{tabular}{|p{4cm}|p{6cm}|p{6cm}|}
\hline
\textbf{Type} & \textbf{Definition} & \textbf{Example} \\
\hline
\textbf{Share Capital} & Amount invested by shareholders in exchange for shares of ownership. & An investor purchases 1,000 shares of a company at \$10 each; \$10,000 is recorded as share capital. \\
\hline
\textbf{Retained Earnings} & Portion of net income retained in the business after dividends are paid. & A company earns \$50,000 profit and pays \$10,000 in dividends; \$40,000 is retained earnings. \\
\hline
\end{tabular}

\vspace{0.5cm} 

\textbf{Types of Revenue and Expenses:}

\begin{itemize}
    \item \textbf{Revenue:}
    \begin{itemize}
        \item \textbf{Sales Revenue} – Income from selling goods (e.g., merchandise sales by a retailer).
        \item \textbf{Service Revenue} – Income from providing services (e.g., consulting fees, legal services).
    \end{itemize}

    \item \textbf{Operating Expenses:}
    \begin{itemize}
        \item Rent Expense
        \item Salaries and Wages
        \item Utilities Expense
        \item Insurance Expense
        \item Advertising Expense
        \item Depreciation and Amortization
        \item Repairs and Maintenance
        \item Office Supplies
    \end{itemize}

    \item \textbf{Non-Operating Income:}
    \begin{itemize}
        \item Interest Income
        \item Dividend Income
        \item Gain on Sale of Assets
        \item Rental Income (if not core business)
        \item Foreign Exchange Gain
    \end{itemize}
\end{itemize}

\vspace{0.5cm} 

\textbf{Types of Business Entities and Comparison:}

\begin{tabular}{|p{3cm}|p{4cm}|p{4cm}|p{5cm}|}
\hline
\textbf{Criteria} & \textbf{Proprietorship} & \textbf{Partnership} & \textbf{Company (Corporation)} \\
\hline
\textbf{Ownership} & Single individual & 2 - 20 partners & 2 - unlimited Shareholders \\
\hline
\textbf{Legal Identity} & No separate legal entity & Not a separate legal entity (except LLP) & Separate legal entity \\
\hline
\textbf{Liability} & Unlimited liability & Unlimited (except LLP) & Limited to shareholding \\
\hline
\textbf{Capital Source} & Personal funds & Partner contributions & Share issuance, retained earnings \\
\hline
\textbf{Regulation} & Minimal legal formalities & Moderate regulation & Heavily regulated by company laws \\
\hline
\textbf{Continuity} & Ends on owner's death or withdrawal & Ends on partner exit (unless reconstituted) & Perpetual succession \\
\hline
\textbf{Taxation} & Taxed as personal income & Taxed as personal income of partners & Separate entity, corporate tax applies \\
\hline
\end{tabular}

\vspace{1cm}
\clearpage

\vspace{0.5cm}

\notesection{Standard Accounting Reports with Examples}{17-04-25 Thursday}
\textbf{Standard Accounting Reports with Examples:}

\begin{itemize}
    \item \textbf{1. Income Statement (Profit and Loss Statement):}

    \begin{tabular}{|p{10cm}|p{5.5cm}|}
    \hline
    \textbf{Particulars} & \textbf{Amount (USD)} \\
    \hline
    Sales Revenue & 120,000 \\
    \hline
    Less: Cost of Goods Sold (COGS) & 70,000 \\
    \hline
    \textbf{Gross Profit\footnotemark[1]} & \textbf{50,000} \\
    \hline
    Less: Operating Expenses & 25,000 \\
    \hline
    \textbf{Net Operating Income\footnotemark[2]} & \textbf{25,000} \\
    \hline
    Add: Non-operating Income (Interest Income) & 3,000 \\
    \hline
    Less: Interest Expense & 2,000 \\
    \hline
    \textbf{Profit Before Tax (PBT)\footnotemark[3]} & \textbf{26,000} \\
    \hline
    Less: Income Tax Expense & 6,000 \\
    \hline
    \textbf{Net Income\footnotemark[4]} & \textbf{20,000} \\
    \hline
    \end{tabular}

    \footnotetext[1]{Also known as: Gross Margin}
    \footnotetext[2]{Also known as: Operating Profit, EBIT (Earnings Before Interest and Tax)}
    \footnotetext[3]{Also known as: Earnings Before Tax (EBT)}
    \footnotetext[4]{Also known as: Net Profit, Net Earnings, Bottom Line}

    \vspace{0.5cm}
    \item \textbf{2. Balance Sheet (as of a specific date):}

    \begin{tabular}{|p{5.5cm}|p{5cm}|p{5cm}|}
    \hline
    \textbf{Assets} & \textbf{Liabilities} & \textbf{Owner’s Equity} \\
    \hline
    Cash: 15,000 & Accounts Payable: 20,000 & Capital: 10,000 \\
    Accounts Receivable: 10,000 & Bank Loan: 30,000 & Retained Earnings: 15,000 \\
    Inventory: 25,000 &  &  \\
    Equipment: 35,000 &  &  \\
    \hline
    \textbf{Total: 85,000} & \textbf{Total: 50,000} & \textbf{Total: 35,000} \\
    \hline
    \end{tabular}

    \vspace{0.5cm}
    \item \textbf{3. Cash Flow Statement:}

    \begin{tabular}{|p{10cm}|p{5cm}|}
    \hline
    \textbf{Cash Flows from Operating Activities} & \textbf{Amount (USD)} \\
    \hline
    Cash received from customers & 110,000 \\
    Cash paid for operating expenses & (80,000) \\
    \textbf{Net Cash from Operating Activities} & \textbf{30,000} \\
    \hline
    \textbf{Cash Flows from Investing Activities} & \\
    \hline
    Purchase of equipment & (20,000) \\
    \textbf{Net Cash from Investing Activities} & \textbf{(20,000)} \\
    \hline
    \textbf{Cash Flows from Financing Activities} & \\
    \hline
    Proceeds from bank loan & 10,000 \\
    Owner capital contribution & 5,000 \\
    \textbf{Net Cash from Financing Activities} & \textbf{15,000} \\
    \hline
    \textbf{Net Increase in Cash} & \textbf{25,000} \\
    \hline
    \end{tabular}

    \vspace{0.5cm}
    \item \textbf{4. Statement of Owner’s Equity:}

    \begin{tabular}{|p{8cm}|p{5cm}|}
    \hline
    \textbf{Particulars} & \textbf{Amount (USD)} \\
    \hline
    Beginning Capital & 30,000 \\
    
    Add: Owner Contribution & 5,000 \\
    Less: Owner Withdrawals & (10,000) \\
    Add: Net Profit & 20,000 \\
    Less: Divident & (10,000) \\
    \hline
    \textbf{Ending Capital} & \textbf{35,000} \\
    \hline
    \end{tabular}

\end{itemize}


\vspace{0.5cm} 

\vspace{0.5cm}
\section*{Question}
Sakiful Alam and Aziz Khan opened a web consulting business called \textbf{Money Laundering Services Pvt. Ltd.} and completed the following transactions in its first month of operations:

\begin{itemize}
    \item \textbf{June 1:} Alam and Khan invested BDT 80,000 cash along with office equipment valued at BDT 26,000 in the company.
    \item \textbf{June 2:} The company prepaid BDT 9,000 cash for 12 months' rent for office space. \textit{(Hint: Debit Prepaid Rent for BDT 9,000)}
    \item \textbf{June 3:} The company made credit purchases for BDT 8,000 in office equipment and BDT 3,600 in office supplies. Payment is due within 10 days.
    \item \textbf{June 6:} The company completed services for a client and immediately received BDT 4,000 cash.
    \item \textbf{June 9:} The company completed a BDT 6,000 project for a client, who must pay within 30 days.
    \item \textbf{June 13:} The company paid BDT 11,600 cash to settle the account payable created on June 3.
    \item \textbf{June 19:} The company paid BDT 2,400 cash for the premium on a 12-month insurance policy. \textit{(Hint: It is prepayment)}
    \item \textbf{June 22:} The company received BDT 4,400 cash as partial payment for the work completed on June 9.
    \item \textbf{June 25:} The company completed work for another client for BDT 2,890 on credit.
    \item \textbf{June 28:} The company gave BDT 5,500 to Alam and Sakiful in cash as dividends.
    \item \textbf{June 29:} The company purchased BDT 600 of additional office supplies on credit.
    \item \textbf{June 30:} The company paid BDT 435 cash for this month's utility bill.
\end{itemize}

\vspace{0.5cm}
\clearpage
\begin{center}
\textbf{Balance Sheet of Money Laundering Services Pvt. Ltd.} \\
\textbf{As at June 30, 2025} 

\vspace{0.5cm}
\begin{tabular}{|p{8cm}|p{8cm}|}
\hline
\multicolumn{1}{|c|}{\textbf{Assets}} & \multicolumn{1}{c|}{\textbf{Liabilities and Owner’s Equity}} \\
\hline
\textbf{Current Assets} & \textbf{Current Liabilities} \\
Cash \dotfill BDT 66,465 & Accounts Payable \dotfill BDT 600 \\
Accounts Receivable \dotfill BDT 4,490 &  \\
Office Supplies \dotfill BDT 4,200 & \\
Prepaid Rent \dotfill BDT 9,000 & \\
Prepaid Insurance \dotfill BDT 2,400 & \\
\hline
\textbf{Non-current Assets} & \textbf{Owner’s Equity} \\
Office Equipment \dotfill BDT 60,000 & Owner Capital \dotfill BDT 106,000 \\
 & Retained Earnings \dotfill BDT 39,955 \\
\hline
\textbf{Total Assets} \dotfill \textbf{BDT 146,555} & \textbf{Total Liabilities and Equity} \dotfill \textbf{BDT 146,555} \\
\hline
\end{tabular}
\end{center}
\vspace{0.5cm} 

\textbf{Clarification Notes:}

\begin{multicols}{2}

    \textbf{Cash (BDT 66,465):}
    \begin{itemize}
        \item Initial investment: BDT 80,000
        \item Service revenue (June 6): BDT 4,000
        \item Partial payment from client (June 22): BDT 4,400
        \item Prepaid rent (June 2): BDT (9,000)
        \item Payment for June 3 purchases: BDT (11,600)
        \item Prepaid insurance (June 19): BDT (2,400)
        \item Dividends (June 28): BDT (5,500)
        \item Utilities expense (June 30): BDT (435)
    \end{itemize}
    
    \textbf{Office Equipment (BDT 34,000):}
    \begin{itemize}
        \item Initial contribution: BDT 26,000
        \item Purchased on credit (June 3): BDT 8,000
    \end{itemize}
    
    \textbf{Office Supplies (BDT 4,200):}
    \begin{itemize}
        \item Purchased (June 3): BDT 3,600
        \item Purchased (June 29): BDT 600
    \end{itemize}
    
    \textbf{Prepaid Rent (BDT 9,000):}
    \begin{itemize}
        \item Paid in advance for 12 months (June 2)
    \end{itemize}
    
    \textbf{Prepaid Insurance (BDT 2,400):}
    \begin{itemize}
        \item 12-month insurance premium paid (June 19)
    \end{itemize}
    
    \textbf{Accounts Receivable (BDT 4,490):}
    \begin{itemize}
        \item Service provided (June 9): BDT 6,000
        \item Service provided (June 25): BDT 2,890
        \item Partial collection (June 22): BDT (4,400)
    \end{itemize}
    
    \textbf{Accounts Payable (BDT 600):}
    \begin{itemize}
        \item Equipment purchased (June 3): BDT 8,000
        \item Supplies purchased (June 3): BDT 3,600
        \item Payment made (June 13): BDT (11,600)
        \item Supplies purchased (June 29): BDT 600
    \end{itemize}
    
    \textbf{Owner Capital (BDT 106,000):}
    \begin{itemize}
        \item Cash invested: BDT 80,000
        \item Equipment invested: BDT 26,000
    \end{itemize}
    
    \textbf{Retained Earnings (BDT 6,955):}
    \begin{itemize}
        \item Total Revenue: BDT 12,890
        \item Total Expenses \& Dividends: BDT (5,935)
        \item Net Income: BDT 6,955
    \end{itemize}
    
    \end{multicols}
    


\vspace{1cm} 
\clearpage 

% double entry accounting system 
\notesection{Double Entry System and Debit-Credit Rules}{23-04-2025 Wednesday}

\textbf{Double Entry System:}
\begin{itemize}
    \item Every transaction affects at least two accounts.
    \item Based on the accounting equation: \textbf{Assets = Liabilities + Owner’s Equity}.
    \item Ensures accounting records are balanced.
    \item Promotes accuracy and reduces fraud through self-checking mechanisms.
\end{itemize}

\textbf{Key Characteristics:}
\begin{itemize}
    \item \textbf{Dual Aspect:} Every transaction has a dual effect — one debit and one credit.
    \item \textbf{Balance Maintenance:} Total debits always equal total credits.
    \item \textbf{Scientific Recording:} Systematic, chronological, and consistent.
    \item \textbf{Error Detection:} Helps locate posting and recording errors.
\end{itemize}

\vspace{0.5cm}
\textbf{Debit-Credit Rules for Accounting Elements:}

\begin{center}
\begin{tabular}{|c|c|c|}
\hline
\textbf{Account Type} & \textbf{Increase (Dr/Cr)} & \textbf{Decrease (Dr/Cr)} \\
\hline
Assets & Debit & Credit \\
Liabilities & Credit & Debit \\
Owner’s Equity & Credit & Debit \\
Revenue & Credit & Debit \\
Expenses & Debit & Credit \\
Dividends & Debit & Credit \\
\hline
\end{tabular}
\end{center}

\vspace{0.5cm}
\textbf{Writing Conventions for Debit-Credit Sheets:}
\begin{itemize}
    \item \textbf{Left side (Debit):} Inflows, asset acquisition, expense recognition, or owner withdrawals.
    \item \textbf{Right side (Credit):} Outflows, liabilities incurred, revenue earned, or capital added.
    \item \textbf{Format:} Traditional T-account — left side for debit entries and right side for credit entries.
    \item \textbf{Chronological Entry:} Transactions are recorded in the order they occur.
    \item \textbf{Narration:} Brief explanation is provided under the entry for clarity.
\end{itemize}

\vspace{1cm}
\clearpage

\section*{Question}
On January 1, 2025, Mr. Tanvir started a business named \textbf{Tanvir Traders}. The following transactions occurred during January:
\begin{itemize}
    \item Jan 1: Invested BDT 100,000 cash and BDT 50,000 worth of furniture.
    \item Jan 3: Purchased goods for BDT 30,000 in cash.
    \item Jan 5: Sold goods for BDT 20,000 on credit.
    \item Jan 10: Paid salaries BDT 5,000.
    \item Jan 12: Received BDT 15,000 from customers.
    \item Jan 15: Paid BDT 3,000 for rent.
    \item Jan 20: Purchased goods on credit for BDT 10,000.
    \item Jan 25: Sold goods for BDT 12,000 in cash.
    \item Jan 28: Withdrew BDT 2,000 for personal use.
\end{itemize}

\section*{Journal Entries}

\begin{center}
\begin{tabular}{|p{2cm}|p{8cm}|p{2.5cm}|p{2.7cm}|}
\hline
\textbf{Date} & \textbf{Particulars} & \textbf{Debit (BDT)} & \textbf{Credit (BDT)} \\
\hline
Jan 1 & Cash A/C Dr. & 100,000 & \\
      & Furniture A/C Dr. & 50,000 & \\
      & \quad To Capital A/C & & 150,000 \\
\hline
Jan 3 & Purchase A/C Dr. & 30,000 & \\
      & \quad To Cash A/C & & 30,000 \\
\hline
Jan 5 & Accounts Receivable A/C Dr. & 20,000 & \\
      & \quad To Sales A/C & & 20,000 \\
\hline
Jan 10 & Salary Expense A/C Dr. & 5,000 & \\
       & \quad To Cash A/C & & 5,000 \\
\hline
Jan 12 & Cash A/C Dr. & 15,000 & \\
       & \quad To Accounts Receivable A/C & & 15,000 \\
\hline
Jan 15 & Rent Expense A/C Dr. & 3,000 & \\
       & \quad To Cash A/C & & 3,000 \\
\hline
Jan 20 & Purchase A/C Dr. & 10,000 & \\
       & \quad To Accounts Payable A/C & & 10,000 \\
\hline
Jan 25 & Cash A/C Dr. & 12,000 & \\
       & \quad To Sales A/C & & 12,000 \\
\hline
Jan 28 & Drawings A/C Dr. & 2,000 & \\
       & \quad To Cash A/C & & 2,000 \\
\hline
\end{tabular}
\end{center}

\clearpage
\textbf{Ledger Accounts for June Transactions}
\begin{center}
\textbf{Cash Account}

\begin{tabular}{|p{2cm}|p{8cm}|p{2.5cm}|p{2.7cm}|}
\hline
\textbf{Date} & \textbf{Particulars} & \textbf{Debit (BDT)} & \textbf{Credit (BDT)} \\
\hline
June 1 & Owner's Capital & 80,000 & \\
June 6 & Service Revenue & 4,000 & \\
June 22 & Accounts Receivable & 4,400 & \\
June 2 & Prepaid Rent & & 9,000 \\
June 13 & Accounts Payable & & 11,600 \\
June 19 & Prepaid Insurance & & 2,400 \\
June 28 & Dividends & & 5,500 \\
June 30 & Utilities Expense & & 435 \\
\hline
\textbf{Total} & & 88,400 & 28,935 \\
\hline
\end{tabular}
\end{center}

\vspace{0.5cm}

\begin{center}
\textbf{Office Equipment Account} \\ 
\begin{tabular}{|p{2cm}|p{8cm}|p{2.5cm}|p{2.7cm}|}
\hline
\textbf{Date} & \textbf{Particulars} & \textbf{Debit (BDT)} & \textbf{Credit (BDT)} \\
\hline
June 1 & Owner’s Capital & 26,000 & \\
June 3 & Accounts Payable & 8,000 & \\
\hline
\textbf{Total} & & 34,000 &  \\
\hline
\end{tabular}
\end{center}

\vspace{0.5cm}

\begin{center}
\textbf{Office Supplies Account}\\ 
\begin{tabular}{|p{2cm}|p{8cm}|p{2.5cm}|p{2.7cm}|}
\hline
\textbf{Date} & \textbf{Particulars} & \textbf{Debit (BDT)} & \textbf{Credit (BDT)} \\
\hline
June 3 & Accounts Payable & 3,600 & \\
June 29 & Accounts Payable & 600 & \\
\hline
\textbf{Total} & & 4,200 & \\
\hline
\end{tabular}
\end{center}

\vspace{0.5cm}

\begin{center}
\textbf{Prepaid Rent Account}\\
\begin{tabular}{|p{2cm}|p{8cm}|p{2.5cm}|p{2.7cm}|}
\hline
\textbf{Date} & \textbf{Particulars} & \textbf{Debit (BDT)} & \textbf{Credit (BDT)} \\
\hline
June 2 & Cash & 9,000 & \\
\hline
\end{tabular}
\end{center}

\vspace{0.5cm}

\begin{center}
\textbf{Prepaid Insurance Account}\\
\begin{tabular}{|p{2cm}|p{8cm}|p{2.5cm}|p{2.7cm}|}
\hline
\textbf{Date} & \textbf{Particulars} & \textbf{Debit (BDT)} & \textbf{Credit (BDT)} \\
\hline
June 19 & Cash & 2,400 & \\
\hline
\end{tabular}
\end{center}

\vspace{0.5cm}

\begin{center}
\textbf{Accounts Receivable Account}\\
\begin{tabular}{|p{2cm}|p{8cm}|p{2.5cm}|p{2.7cm}|}
\hline
\textbf{Date} & \textbf{Particulars} & \textbf{Debit (BDT)} & \textbf{Credit (BDT)} \\
\hline
June 9 & Service Revenue & 6,000 & \\
June 25 & Service Revenue & 2,890 & \\
June 22 & Cash & & 4,400 \\
\hline
\textbf{Total} & & 8,890 & 4,400 \\
\hline
\end{tabular}
\end{center}

\vspace{0.5cm}

\begin{center}
\textbf{Accounts Payable Account}\\
\begin{tabular}{|p{2cm}|p{8cm}|p{2.5cm}|p{2.7cm}|}
\hline
\textbf{Date} & \textbf{Particulars} & \textbf{Debit (BDT)} & \textbf{Credit (BDT)} \\
\hline
June 13 & Cash & 11,600 & \\
June 3 & Office Equipment & & 8,000 \\
June 3 & Office Supplies & & 3,600 \\
June 29 & Office Supplies & & 600 \\
\hline
\textbf{Total} & & 11,600 & 12,200 \\
\hline
\end{tabular}
\end{center}

\vspace{0.5cm}

\begin{center}
\textbf{Service Revenue Account}\\
\begin{tabular}{|p{2cm}|p{8cm}|p{2.5cm}|p{2.7cm}|}
\hline
\textbf{Date} & \textbf{Particulars} & \textbf{Debit (BDT)} & \textbf{Credit (BDT)} \\
\hline
June 6 & Cash & & 4,000 \\
June 9 & Accounts Receivable & & 6,000 \\
June 25 & Accounts Receivable & & 2,890 \\
\hline
\textbf{Total} & & & 12,890 \\
\hline
\end{tabular}
\end{center}

\vspace{0.5cm}

\begin{center}
\textbf{Dividends Account}\\
\begin{tabular}{|p{2cm}|p{8cm}|p{2.5cm}|p{2.7cm}|}
\hline
\textbf{Date} & \textbf{Particulars} & \textbf{Debit (BDT)} & \textbf{Credit (BDT)} \\
\hline
June 28 & Cash & 5,500 & \\
\hline
\end{tabular}
\end{center}

\vspace{0.5cm}

\begin{center}
\textbf{Utilities Expense Account}\\
\begin{tabular}{|p{2cm}|p{8cm}|p{2.5cm}|p{2.7cm}|}
\hline
\textbf{Date} & \textbf{Particulars} & \textbf{Debit (BDT)} & \textbf{Credit (BDT)} \\
\hline
June 30 & Cash & 435 & \\
\hline
\end{tabular}
\end{center}

\vspace{0.5cm}
\clearpage




\vspace{1cm}
\clearpage

% end of the note 


\end{document}
