\documentclass[12pt,a4paper]{book}

% Fonts & Typography — Elegant and Professional
\usepackage[T1]{fontenc}
\usepackage{kpfonts} % Sleek, modern font
\usepackage{microtype}

% Essential Packages
\usepackage{graphicx}
\usepackage{fancyhdr}
\usepackage{tocloft}
\usepackage{titlesec}
\usepackage{datetime}
\usepackage{hyperref}
\usepackage{geometry}
\usepackage{parskip}
\usepackage{array, booktabs}
\usepackage{amsmath}
\usepackage{multicol}
\usepackage{longtable}
\usepackage{footnote}
\usepackage{multicol}

% Page Geometry — Slim, Clean Margins
\geometry{
  a4paper,
  left=20mm,
  right=20mm,
  top=20mm,
  bottom=20mm
}

% Header & Footer Styling
\pagestyle{fancy}
\fancyhf{}
\fancyhead[L]{\small \textit{\nouppercase{\leftmark}}}
\fancyhead[R]{\small A501: Financial Accounting} 
\fancyfoot[C]{\small \thepage}
\renewcommand{\headrulewidth}{0.3pt}
\renewcommand{\footrulewidth}{0.3pt}

% Chapter Title Styling
\titleformat{\chapter}[block]
  {\normalfont\Huge\bfseries}
  {\thechapter.}{12pt}{}

\titleformat{\section}
  {\normalfont\Large\bfseries}
  {\thesection}{1em}{}

% Table of Contents Styling
\renewcommand{\cftchapfont}{\bfseries}
\renewcommand{\cftsecfont}{}
\setlength{\cftbeforechapskip}{5pt}
\setlength{\cftbeforesecskip}{2pt}
\setlength{\cftaftertoctitleskip}{1em}

% Hyperlink Styling
\hypersetup{
    colorlinks=true,
    linkcolor=blue,
    urlcolor=blue,
    pdftitle={A501: Financial Accounting},
    pdfauthor={Md Hasibul Islam},
    pdfpagemode=FullScreen,
}

% Custom Command for Notes 
\newcommand{\notesection}[2]{
  \section*{#1\\ \small \textit{#2}}
  \phantomsection
  \addcontentsline{toc}{section}{#1 - #2}
}

% Document Start 
\begin{document}

% Title Page
\begin{titlepage}
    \centering
    \vspace*{3.5cm}
    \includegraphics[width=0.28\textwidth]{logo.png}\par\vspace{1.5cm}
    {\scshape\LARGE University of Dhaka\par}
    \vspace{0.5cm}
    {\Large Institute of Business Administration (IBA)\par}
    \vspace{1.5cm}
    {\Huge\bfseries Master of Business Administration (MBA)\par}
    \vspace{1cm}
    {\Large A501: \textit{Financial Accounting}\par}
    \vfill
    {\large Last Updated: \today\par}
\end{titlepage}

% Author Details Section 
\section*{Author Details}
\phantomsection
\addcontentsline{toc}{section}{Author Details}

\begin{center}
    \vspace{1em}
    \begin{tabular}{lll}
        \textbf{Name} & : & Md Hasibul Islam \\
        \textbf{Student ID} & : & 201-67-011 \\
        \textbf{Program} & : & Master of Business Administration (MBA) \\
        \textbf{Institute} & : & Institute of Business Administration (IBA) \\
        \textbf{University} & : & University of Dhaka \\
        \textbf{Email} & : & \href{mailto:hasiee8004@gmail.com}{hasiee8004@gmail.com} \\
        \textbf{LinkedIn} & : & \href{https://www.linkedin.com/in/hasib009}{linkedin.com/in/hasib009} \\
        \textbf{GitHub} & : & \href{https://github.com/HasibRockie}{github.com/HasibRockie} \\
        \textbf{Website} & : & \href{https://hasibrockie.github.io}{hasibrockie.github.io} \\
    \end{tabular}
    \vspace{1em}
\end{center}

\clearpage

% Table of Contents
\tableofcontents
\clearpage 

% Notes Sections
\notesection{Definition and Characteristics: Accounting and Transaction}{10-04-25 Thursday}

\textbf{Accounting} is the systematic process of identifying, recording, classifying, summarizing, and communicating financial information to aid economic decision-making. It serves as the language of business and provides a framework for understanding a firm's financial position and performance.

\textbf{Key Characteristics of Accounting:}
\begin{itemize}
    \item \textbf{Systematic}: Follows a standard process and principles (GAAP or IFRS).
    \item \textbf{Historical and Predictive}: Reflects past events and supports future planning.
    \item \textbf{Quantitative}: Concerned primarily with monetary information.
    \item \textbf{Reliable and Verifiable}: Based on evidence (e.g., invoices, receipts).
    \item \textbf{Comparable and Consistent}: Enables year-on-year or firm-to-firm comparison.
\end{itemize}

\textbf{Example:} A company sells a product for \$1,000. The sale is recorded in the accounting books under revenue (income), and cash or receivable is increased by \$1,000.

\textbf{Transaction:} Any business event that has a financial impact on the entity and can be reliably measured.

\textbf{Characteristics of a Transaction:}
\begin{itemize}
    \item \textbf{Financial Impact}: Affects assets, liabilities, equity, income, or expenses.
    \item \textbf{Dual Aspect}: Every transaction has a debit and a credit effect.
    \item \textbf{Measurable in Money}: Must be quantifiable in monetary terms.
\end{itemize}

\textbf{Example:} Paying salary of \$5,000 – decreases cash (asset) and increases expense.

\vspace{0.5cm}

\textbf{Examples: Is it a Transaction?}
\begin{itemize}
    \item \textbf{Purchased machinery for cash} – \textit{Yes, it affects assets (machinery and cash)}.
    \item \textbf{Signed a contract to deliver goods next month (no advance received)} – \textit{No, no financial impact yet}.
    \item \textbf{Paid salaries to employees} – \textit{Yes, it reduces cash and increases expenses}.
    \item \textbf{Owner invests capital into the business} – \textit{Yes, it increases assets and equity}.
    \item \textbf{Employee promoted, no monetary change} – \textit{No, it is not measurable in monetary terms}.
    \item \textbf{Received utility bill (not yet paid)} – \textit{Yes, it increases liabilities and expenses}.
    \item \textbf{Customer places an order (no payment yet)} – \textit{No, it's a future event with no current financial effect}.
    \item \textbf{Sold goods on credit} – \textit{Yes, it increases accounts receivable and revenue}.
    \item \textbf{Declared dividend (not paid yet)} – \textit{Yes, it creates a liability}.
    \item \textbf{Paid electricity bill} – \textit{Yes, it decreases cash and increases expense}.
    \item \textbf{Depreciation of machinery recorded} – \textit{Yes, it increases expense and reduces asset value}.
    \item \textbf{Appointed new auditor} – \textit{No, unless fees are paid or accrued}.
\end{itemize}

\vspace{0.5cm}

\textbf{Types of Accounting Reports and Their Uses}

\begin{itemize}
    \item \textbf{Income Statement (Profit \& Loss Statement)}: Shows revenues and expenses over a period; indicates profitability.\\
    \textit{Use: Assess operational performance.}
    
    \item \textbf{Balance Sheet (Statement of Financial Position)}: Displays assets, liabilities, and owner’s equity at a specific point in time.\\
    \textit{Use: Understand financial position and solvency.}
    
    \item \textbf{Cash Flow Statement (Statement of Cashflows)}: Reports cash inflows and outflows from operating, investing, and financing activities.\\
    \textit{Use: Monitor liquidity and cash management.}
    
    \item \textbf{Statement of changes in Equity (Remained earned statement)}: Explains changes in equity from investments, withdrawals, and retained earnings.\\
    \textit{Use: Analyze changes in owner’s interest in the firm.}
\end{itemize}

\textbf{Characteristics of Good Accounting Reports:}
\begin{itemize}
    \item \textbf{Relevance}: Information must aid decision-making.
    \item \textbf{Faithful Representation}: Complete, neutral, and free from error.
    \item \textbf{Understandability}: Clear presentation for users.
    \item \textbf{Comparability}: Allows evaluation across periods and entities.
    \item \textbf{Timeliness}: Provided in time to be useful.
\end{itemize}

\vspace{0.5cm}
\textbf{Users of Accounting Information}

\begin{itemize}
    \item \textbf{Internal Users:}
    \begin{itemize}
        \item Managers – for planning and control.
        \item Employees – job security and performance.
        \item Owners – profit evaluation.
    \end{itemize}
    
    \item \textbf{External Users:}
    \begin{itemize}
        \item Investors – profitability and risk analysis.
        \item Creditors – repayment ability.
        \item Government – taxation and regulatory compliance.
        \item Customers – supplier stability.
        \item Regulatory Bodies – ensure legal conformity.
    \end{itemize}
\end{itemize}

\vspace{0.5cm}
\textbf{Examples of Elements in Accounting :}

\begin{tabular}{|p{4cm}|p{12cm}|}
\hline
\textbf{Category} & \textbf{Examples} \\
\hline
\textbf{Assets} & 
Cash, Bank Balance, Accounts Receivable, Inventory, Prepaid Insurance, Office Supplies, Furniture, Equipment, Buildings, Land, Vehicles, Patents, Copyrights, 
Trademarks, Goodwill, Long-term Investments, Marketable Securities \\
\hline
\textbf{Liabilities} & 
Accounts Payable, Notes Payable, Accrued Expenses (e.g., Wages Payable, Interest Payable), Unearned Revenue, Loans Payable, Bonds Payable, Taxes Payable, Deferred Revenue, Lease Obligations, Credit Card Payable, Mortgage Payable \\
\hline
\textbf{Owner’s Equity} & 
Capital, Retained Earnings, Drawings, Owner’s Contributions, Common Stock, Preferred Stock, Additional Paid-in Capital, Treasury Stock, Accumulated Other Comprehensive Income \\
\hline
\textbf{Income (Revenue)} & 
Sales Revenue, Service Revenue, Interest Income, Rental Income, Commission Revenue, Dividend Income, Consulting Revenue, Royalties Earned, Investment Income, Subscription Revenue, Gain on Sale of Asset, Foreign Exchange Gain \\
\hline
\textbf{Expenses} & 
Salaries and Wages Expense, Rent Expense, Utilities Expense, Insurance Expense, Depreciation Expense, Amortization Expense, Interest Expense, Advertising Expense, Supplies Expense, Repairs and Maintenance Expense, Legal Fees, Audit Fees, Delivery Expense, Telephone Expense, Training Expense, Bad Debts Expense, Loss on Disposal of Asset \\
\hline
\end{tabular}

\vspace{0.5cm}

\textbf{Sub-types of Owner’s Equity:}

\begin{tabular}{|p{4cm}|p{6cm}|p{6cm}|}
\hline
\textbf{Type} & \textbf{Definition} & \textbf{Example} \\
\hline
\textbf{Share Capital} & Amount invested by shareholders in exchange for shares of ownership. & An investor purchases 1,000 shares of a company at \$10 each; \$10,000 is recorded as share capital. \\
\hline
\textbf{Retained Earnings} & Portion of net income retained in the business after dividends are paid. & A company earns \$50,000 profit and pays \$10,000 in dividends; \$40,000 is retained earnings. \\
\hline
\end{tabular}

\vspace{0.5cm} 

\textbf{Types of Revenue and Expenses:}

\begin{itemize}
    \item \textbf{Revenue:}
    \begin{itemize}
        \item \textbf{Sales Revenue} – Income from selling goods (e.g., merchandise sales by a retailer).
        \item \textbf{Service Revenue} – Income from providing services (e.g., consulting fees, legal services).
    \end{itemize}

    \item \textbf{Operating Expenses:}
    \begin{itemize}
        \item Rent Expense
        \item Salaries and Wages
        \item Utilities Expense
        \item Insurance Expense
        \item Advertising Expense
        \item Depreciation and Amortization
        \item Repairs and Maintenance
        \item Office Supplies
    \end{itemize}

    \item \textbf{Non-Operating Income:}
    \begin{itemize}
        \item Interest Income
        \item Dividend Income
        \item Gain on Sale of Assets
        \item Rental Income (if not core business)
        \item Foreign Exchange Gain
    \end{itemize}
\end{itemize}

\vspace{0.5cm} 

\textbf{Types of Business Entities and Comparison:}

\begin{tabular}{|p{3cm}|p{4cm}|p{4cm}|p{5cm}|}
\hline
\textbf{Criteria} & \textbf{Proprietorship} & \textbf{Partnership} & \textbf{Company (Corporation)} \\
\hline
\textbf{Ownership} & Single individual & 2 - 20 partners & 2 - unlimited Shareholders \\
\hline
\textbf{Legal Identity} & No separate legal entity & Not a separate legal entity (except LLP) & Separate legal entity \\
\hline
\textbf{Liability} & Unlimited liability & Unlimited (except LLP) & Limited to shareholding \\
\hline
\textbf{Capital Source} & Personal funds & Partner contributions & Share issuance, retained earnings \\
\hline
\textbf{Regulation} & Minimal legal formalities & Moderate regulation & Heavily regulated by company laws \\
\hline
\textbf{Continuity} & Ends on owner's death or withdrawal & Ends on partner exit (unless reconstituted) & Perpetual succession \\
\hline
\textbf{Taxation} & Taxed as personal income & Taxed as personal income of partners & Separate entity, corporate tax applies \\
\hline
\end{tabular}

\vspace{1cm}
\clearpage

\vspace{0.5cm}

\notesection{Standard Accounting Reports with Examples}{17-04-25 Thursday}
\textbf{Standard Accounting Reports with Examples:}

\begin{itemize}
    \item \textbf{1. Income Statement (Profit and Loss Statement):}

    \begin{tabular}{|p{10cm}|p{5.5cm}|}
    \hline
    \textbf{Particulars} & \textbf{Amount (USD)} \\
    \hline
    Sales Revenue & 120,000 \\
    \hline
    Less: Cost of Goods Sold (COGS) & 70,000 \\
    \hline
    \textbf{Gross Profit\footnotemark[1]} & \textbf{50,000} \\
    \hline
    Less: Operating Expenses & 25,000 \\
    \hline
    \textbf{Net Operating Income\footnotemark[2]} & \textbf{25,000} \\
    \hline
    Add: Non-operating Income (Interest Income) & 3,000 \\
    \hline
    Less: Interest Expense & 2,000 \\
    \hline
    \textbf{Profit Before Tax (PBT)\footnotemark[3]} & \textbf{26,000} \\
    \hline
    Less: Income Tax Expense & 6,000 \\
    \hline
    \textbf{Net Income\footnotemark[4]} & \textbf{20,000} \\
    \hline
    \end{tabular}

    \footnotetext[1]{Also known as: Gross Margin}
    \footnotetext[2]{Also known as: Operating Profit, EBIT (Earnings Before Interest and Tax)}
    \footnotetext[3]{Also known as: Earnings Before Tax (EBT)}
    \footnotetext[4]{Also known as: Net Profit, Net Earnings, Bottom Line}

    \vspace{0.5cm}
    \item \textbf{2. Balance Sheet (as of a specific date):}

    \begin{tabular}{|p{5.5cm}|p{5cm}|p{5cm}|}
    \hline
    \textbf{Assets} & \textbf{Liabilities} & \textbf{Owner’s Equity} \\
    \hline
    Cash: 15,000 & Accounts Payable: 20,000 & Capital: 10,000 \\
    Accounts Receivable: 10,000 & Bank Loan: 30,000 & Retained Earnings: 15,000 \\
    Inventory: 25,000 &  &  \\
    Equipment: 35,000 &  &  \\
    \hline
    \textbf{Total: 85,000} & \textbf{Total: 50,000} & \textbf{Total: 35,000} \\
    \hline
    \end{tabular}

    \vspace{0.5cm}
    \item \textbf{3. Cash Flow Statement:}

    \begin{tabular}{|p{10cm}|p{5cm}|}
    \hline
    \textbf{Cash Flows from Operating Activities} & \textbf{Amount (USD)} \\
    \hline
    Cash received from customers & 110,000 \\
    Cash paid for operating expenses & (80,000) \\
    \textbf{Net Cash from Operating Activities} & \textbf{30,000} \\
    \hline
    \textbf{Cash Flows from Investing Activities} & \\
    \hline
    Purchase of equipment & (20,000) \\
    \textbf{Net Cash from Investing Activities} & \textbf{(20,000)} \\
    \hline
    \textbf{Cash Flows from Financing Activities} & \\
    \hline
    Proceeds from bank loan & 10,000 \\
    Owner capital contribution & 5,000 \\
    \textbf{Net Cash from Financing Activities} & \textbf{15,000} \\
    \hline
    \textbf{Net Increase in Cash} & \textbf{25,000} \\
    \hline
    \end{tabular}

    \vspace{0.5cm}
    \item \textbf{4. Statement of Owner’s Equity:}

    \begin{tabular}{|p{8cm}|p{5cm}|}
    \hline
    \textbf{Particulars} & \textbf{Amount (USD)} \\
    \hline
    Beginning Capital & 30,000 \\
    
    Add: Owner Contribution & 5,000 \\
    Less: Owner Withdrawals & (10,000) \\
    Add: Net Profit & 20,000 \\
    Less: Divident & (10,000) \\
    \hline
    \textbf{Ending Capital} & \textbf{35,000} \\
    \hline
    \end{tabular}

\end{itemize}


\vspace{0.5cm} 

\vspace{0.5cm}
\section*{Question}
Sakiful Alam and Aziz Khan opened a web consulting business called \textbf{Money Laundering Services Pvt. Ltd.} and completed the following transactions in its first month of operations:

\begin{itemize}
    \item \textbf{June 1:} Alam and Khan invested BDT 80,000 cash along with office equipment valued at BDT 26,000 in the company.
    \item \textbf{June 2:} The company prepaid BDT 9,000 cash for 12 months' rent for office space. \textit{(Hint: Debit Prepaid Rent for BDT 9,000)}
    \item \textbf{June 3:} The company made credit purchases for BDT 8,000 in office equipment and BDT 3,600 in office supplies. Payment is due within 10 days.
    \item \textbf{June 6:} The company completed services for a client and immediately received BDT 4,000 cash.
    \item \textbf{June 9:} The company completed a BDT 6,000 project for a client, who must pay within 30 days.
    \item \textbf{June 13:} The company paid BDT 11,600 cash to settle the account payable created on June 3.
    \item \textbf{June 19:} The company paid BDT 2,400 cash for the premium on a 12-month insurance policy. \textit{(Hint: It is prepayment)}
    \item \textbf{June 22:} The company received BDT 4,400 cash as partial payment for the work completed on June 9.
    \item \textbf{June 25:} The company completed work for another client for BDT 2,890 on credit.
    \item \textbf{June 28:} The company gave BDT 5,500 to Alam and Sakiful in cash as dividends.
    \item \textbf{June 29:} The company purchased BDT 600 of additional office supplies on credit.
    \item \textbf{June 30:} The company paid BDT 435 cash for this month's utility bill.
\end{itemize}

\vspace{0.5cm}
\clearpage
\begin{center}
\textbf{Balance Sheet of Money Laundering Services Pvt. Ltd.} \\
\textbf{As at June 30, 2025} 

\vspace{0.5cm}
\begin{tabular}{|p{8cm}|p{8cm}|}
\hline
\multicolumn{1}{|c|}{\textbf{Assets}} & \multicolumn{1}{c|}{\textbf{Liabilities and Owner’s Equity}} \\
\hline
\textbf{Current Assets} & \textbf{Current Liabilities} \\
Cash \dotfill BDT 66,465 & Accounts Payable \dotfill BDT 600 \\
Accounts Receivable \dotfill BDT 4,490 &  \\
Office Supplies \dotfill BDT 4,200 & \\
Prepaid Rent \dotfill BDT 9,000 & \\
Prepaid Insurance \dotfill BDT 2,400 & \\
\hline
\textbf{Non-current Assets} & \textbf{Owner’s Equity} \\
Office Equipment \dotfill BDT 60,000 & Owner Capital \dotfill BDT 106,000 \\
 & Retained Earnings \dotfill BDT 39,955 \\
\hline
\textbf{Total Assets} \dotfill \textbf{BDT 146,555} & \textbf{Total Liabilities and Equity} \dotfill \textbf{BDT 146,555} \\
\hline
\end{tabular}
\end{center}
\vspace{0.5cm} 

\textbf{Clarification Notes:}

\begin{multicols}{2}

    \textbf{Cash (BDT 66,465):}
    \begin{itemize}
        \item Initial investment: BDT 80,000
        \item Service revenue (June 6): BDT 4,000
        \item Partial payment from client (June 22): BDT 4,400
        \item Prepaid rent (June 2): BDT (9,000)
        \item Payment for June 3 purchases: BDT (11,600)
        \item Prepaid insurance (June 19): BDT (2,400)
        \item Dividends (June 28): BDT (5,500)
        \item Utilities expense (June 30): BDT (435)
    \end{itemize}
    
    \textbf{Office Equipment (BDT 34,000):}
    \begin{itemize}
        \item Initial contribution: BDT 26,000
        \item Purchased on credit (June 3): BDT 8,000
    \end{itemize}
    
    \textbf{Office Supplies (BDT 4,200):}
    \begin{itemize}
        \item Purchased (June 3): BDT 3,600
        \item Purchased (June 29): BDT 600
    \end{itemize}
    
    \textbf{Prepaid Rent (BDT 9,000):}
    \begin{itemize}
        \item Paid in advance for 12 months (June 2)
    \end{itemize}
    
    \textbf{Prepaid Insurance (BDT 2,400):}
    \begin{itemize}
        \item 12-month insurance premium paid (June 19)
    \end{itemize}
    
    \textbf{Accounts Receivable (BDT 4,490):}
    \begin{itemize}
        \item Service provided (June 9): BDT 6,000
        \item Service provided (June 25): BDT 2,890
        \item Partial collection (June 22): BDT (4,400)
    \end{itemize}
    
    \textbf{Accounts Payable (BDT 600):}
    \begin{itemize}
        \item Equipment purchased (June 3): BDT 8,000
        \item Supplies purchased (June 3): BDT 3,600
        \item Payment made (June 13): BDT (11,600)
        \item Supplies purchased (June 29): BDT 600
    \end{itemize}
    
    \textbf{Owner Capital (BDT 106,000):}
    \begin{itemize}
        \item Cash invested: BDT 80,000
        \item Equipment invested: BDT 26,000
    \end{itemize}
    
    \textbf{Retained Earnings (BDT 6,955):}
    \begin{itemize}
        \item Total Revenue: BDT 12,890
        \item Total Expenses \& Dividends: BDT (5,935)
        \item Net Income: BDT 6,955
    \end{itemize}
    
    \end{multicols}
    


\vspace{1cm} 
\clearpage 

% double entry accounting system 
\notesection{Double Entry System and Debit-Credit Rules}{23-04-2025 Wednesday}

\textbf{Double Entry System:}
\begin{itemize}
    \item Every transaction affects at least two accounts.
    \item Based on the accounting equation: \textbf{Assets = Liabilities + Owner’s Equity}.
    \item Ensures accounting records are balanced.
    \item Promotes accuracy and reduces fraud through self-checking mechanisms.
\end{itemize}

\textbf{Key Characteristics:}
\begin{itemize}
    \item \textbf{Dual Aspect:} Every transaction has a dual effect — one debit and one credit.
    \item \textbf{Balance Maintenance:} Total debits always equal total credits.
    \item \textbf{Scientific Recording:} Systematic, chronological, and consistent.
    \item \textbf{Error Detection:} Helps locate posting and recording errors.
\end{itemize}

\vspace{0.5cm}
\textbf{Debit-Credit Rules for Accounting Elements:}

\begin{center}
\begin{tabular}{|c|c|c|}
\hline
\textbf{Account Type} & \textbf{Increase (Dr/Cr)} & \textbf{Decrease (Dr/Cr)} \\
\hline
Assets & Debit & Credit \\
Liabilities & Credit & Debit \\
Owner’s Equity & Credit & Debit \\
Revenue & Credit & Debit \\
Expenses & Debit & Credit \\
Dividends & Debit & Credit \\
\hline
\end{tabular}
\end{center}

\vspace{0.5cm}
\textbf{Writing Conventions for Debit-Credit Sheets:}
\begin{itemize}
    \item \textbf{Left side (Debit):} Inflows, asset acquisition, expense recognition, or owner withdrawals.
    \item \textbf{Right side (Credit):} Outflows, liabilities incurred, revenue earned, or capital added.
    \item \textbf{Format:} Traditional T-account — left side for debit entries and right side for credit entries.
    \item \textbf{Chronological Entry:} Transactions are recorded in the order they occur.
    \item \textbf{Narration:} Brief explanation is provided under the entry for clarity.
\end{itemize}

\vspace{1cm}
\clearpage

\section*{Question}
On January 1, 2025, Mr. Tanvir started a business named \textbf{Tanvir Traders}. The following transactions occurred during January:
\begin{itemize}
    \item Jan 1: Invested BDT 100,000 cash and BDT 50,000 worth of furniture.
    \item Jan 3: Purchased goods for BDT 30,000 in cash.
    \item Jan 5: Sold goods for BDT 20,000 on credit.
    \item Jan 10: Paid salaries BDT 5,000.
    \item Jan 12: Received BDT 15,000 from customers.
    \item Jan 15: Paid BDT 3,000 for rent.
    \item Jan 20: Purchased goods on credit for BDT 10,000.
    \item Jan 25: Sold goods for BDT 12,000 in cash.
    \item Jan 28: Withdrew BDT 2,000 for personal use.
\end{itemize}

\section*{Journal Entries}

\begin{center}
\begin{tabular}{|p{2cm}|p{8cm}|p{2.5cm}|p{2.7cm}|}
\hline
\textbf{Date} & \textbf{Particulars} & \textbf{Debit (BDT)} & \textbf{Credit (BDT)} \\
\hline
Jan 1 & Cash A/C Dr. & 100,000 & \\
      & Furniture A/C Dr. & 50,000 & \\
      & \quad To Capital A/C & & 150,000 \\
\hline
Jan 3 & Purchase A/C Dr. & 30,000 & \\
      & \quad To Cash A/C & & 30,000 \\
\hline
Jan 5 & Accounts Receivable A/C Dr. & 20,000 & \\
      & \quad To Sales A/C & & 20,000 \\
\hline
Jan 10 & Salary Expense A/C Dr. & 5,000 & \\
       & \quad To Cash A/C & & 5,000 \\
\hline
Jan 12 & Cash A/C Dr. & 15,000 & \\
       & \quad To Accounts Receivable A/C & & 15,000 \\
\hline
Jan 15 & Rent Expense A/C Dr. & 3,000 & \\
       & \quad To Cash A/C & & 3,000 \\
\hline
Jan 20 & Purchase A/C Dr. & 10,000 & \\
       & \quad To Accounts Payable A/C & & 10,000 \\
\hline
Jan 25 & Cash A/C Dr. & 12,000 & \\
       & \quad To Sales A/C & & 12,000 \\
\hline
Jan 28 & Drawings A/C Dr. & 2,000 & \\
       & \quad To Cash A/C & & 2,000 \\
\hline
\end{tabular}
\end{center}

\clearpage
\textbf{Ledger Accounts for June Transactions}
\begin{center}
\textbf{Cash Account}

\begin{tabular}{|p{2cm}|p{8cm}|p{2.5cm}|p{2.7cm}|}
\hline
\textbf{Date} & \textbf{Particulars} & \textbf{Debit (BDT)} & \textbf{Credit (BDT)} \\
\hline
June 1 & Owner's Capital & 80,000 & \\
June 6 & Service Revenue & 4,000 & \\
June 22 & Accounts Receivable & 4,400 & \\
June 2 & Prepaid Rent & & 9,000 \\
June 13 & Accounts Payable & & 11,600 \\
June 19 & Prepaid Insurance & & 2,400 \\
June 28 & Dividends & & 5,500 \\
June 30 & Utilities Expense & & 435 \\
\hline
\textbf{Total} & & 88,400 & 28,935 \\
\hline
\end{tabular}
\end{center}

\vspace{0.5cm}

\begin{center}
\textbf{Office Equipment Account} \\ 
\begin{tabular}{|p{2cm}|p{8cm}|p{2.5cm}|p{2.7cm}|}
\hline
\textbf{Date} & \textbf{Particulars} & \textbf{Debit (BDT)} & \textbf{Credit (BDT)} \\
\hline
June 1 & Owner’s Capital & 26,000 & \\
June 3 & Accounts Payable & 8,000 & \\
\hline
\textbf{Total} & & 34,000 &  \\
\hline
\end{tabular}
\end{center}

\vspace{0.5cm}

\begin{center}
\textbf{Office Supplies Account}\\ 
\begin{tabular}{|p{2cm}|p{8cm}|p{2.5cm}|p{2.7cm}|}
\hline
\textbf{Date} & \textbf{Particulars} & \textbf{Debit (BDT)} & \textbf{Credit (BDT)} \\
\hline
June 3 & Accounts Payable & 3,600 & \\
June 29 & Accounts Payable & 600 & \\
\hline
\textbf{Total} & & 4,200 & \\
\hline
\end{tabular}
\end{center}

\vspace{0.5cm}

\begin{center}
\textbf{Prepaid Rent Account}\\
\begin{tabular}{|p{2cm}|p{8cm}|p{2.5cm}|p{2.7cm}|}
\hline
\textbf{Date} & \textbf{Particulars} & \textbf{Debit (BDT)} & \textbf{Credit (BDT)} \\
\hline
June 2 & Cash & 9,000 & \\
\hline
\end{tabular}
\end{center}

\vspace{0.5cm}

\begin{center}
\textbf{Prepaid Insurance Account}\\
\begin{tabular}{|p{2cm}|p{8cm}|p{2.5cm}|p{2.7cm}|}
\hline
\textbf{Date} & \textbf{Particulars} & \textbf{Debit (BDT)} & \textbf{Credit (BDT)} \\
\hline
June 19 & Cash & 2,400 & \\
\hline
\end{tabular}
\end{center}

\vspace{0.5cm}

\begin{center}
\textbf{Accounts Receivable Account}\\
\begin{tabular}{|p{2cm}|p{8cm}|p{2.5cm}|p{2.7cm}|}
\hline
\textbf{Date} & \textbf{Particulars} & \textbf{Debit (BDT)} & \textbf{Credit (BDT)} \\
\hline
June 9 & Service Revenue & 6,000 & \\
June 25 & Service Revenue & 2,890 & \\
June 22 & Cash & & 4,400 \\
\hline
\textbf{Total} & & 8,890 & 4,400 \\
\hline
\end{tabular}
\end{center}

\vspace{0.5cm}

\begin{center}
\textbf{Accounts Payable Account}\\
\begin{tabular}{|p{2cm}|p{8cm}|p{2.5cm}|p{2.7cm}|}
\hline
\textbf{Date} & \textbf{Particulars} & \textbf{Debit (BDT)} & \textbf{Credit (BDT)} \\
\hline
June 13 & Cash & 11,600 & \\
June 3 & Office Equipment & & 8,000 \\
June 3 & Office Supplies & & 3,600 \\
June 29 & Office Supplies & & 600 \\
\hline
\textbf{Total} & & 11,600 & 12,200 \\
\hline
\end{tabular}
\end{center}

\vspace{0.5cm}

\begin{center}
\textbf{Service Revenue Account}\\
\begin{tabular}{|p{2cm}|p{8cm}|p{2.5cm}|p{2.7cm}|}
\hline
\textbf{Date} & \textbf{Particulars} & \textbf{Debit (BDT)} & \textbf{Credit (BDT)} \\
\hline
June 6 & Cash & & 4,000 \\
June 9 & Accounts Receivable & & 6,000 \\
June 25 & Accounts Receivable & & 2,890 \\
\hline
\textbf{Total} & & & 12,890 \\
\hline
\end{tabular}
\end{center}

\vspace{0.5cm}

\begin{center}
\textbf{Dividends Account}\\
\begin{tabular}{|p{2cm}|p{8cm}|p{2.5cm}|p{2.7cm}|}
\hline
\textbf{Date} & \textbf{Particulars} & \textbf{Debit (BDT)} & \textbf{Credit (BDT)} \\
\hline
June 28 & Cash & 5,500 & \\
\hline
\end{tabular}
\end{center}

\vspace{0.5cm}

\begin{center}
\textbf{Utilities Expense Account}\\
\begin{tabular}{|p{2cm}|p{8cm}|p{2.5cm}|p{2.7cm}|}
\hline
\textbf{Date} & \textbf{Particulars} & \textbf{Debit (BDT)} & \textbf{Credit (BDT)} \\
\hline
June 30 & Cash & 435 & \\
\hline
\end{tabular}
\end{center}

\vspace{0.5cm}
\clearpage

\vspace{0.5cm}
\notesection{Journal \& Ledger}{28-04-2025 Monday}

\textbf{Effects on Shared Capital (SC) and Retained Earnings (RE):}

\begin{itemize}
    \item \textbf{Additional Investment:} 
    \begin{itemize}
        \item \textbf{SC:} Increases (Positive Impact)
        \item \textbf{RE:} No direct impact
    \end{itemize}
    \item \textbf{Net Income:} 
    \begin{itemize}
        \item \textbf{RE:} Increases (Positive Impact)
        \item \textbf{SC:} No direct impact
    \end{itemize}
    \item \textbf{Dividends Paid:} 
    \begin{itemize}
        \item \textbf{RE:} Decreases (Negative Impact)
        \item \textbf{SC:} No direct impact
    \end{itemize}
\end{itemize}

\vspace{0.5cm}
\textbf{Relations between Net Income, Revenue, and Expenses:}

\begin{itemize}
    \item \textbf{Net Income (NI) Formula:} 
    \[
    \text{Net Income} = \text{Total Revenue} - \text{Total Expenses}
    \]
    \item \textbf{Revenue:} Increases Net Income (Positive Relationship)
    \item \textbf{Expenses:} Decreases Net Income (Negative Relationship)
\end{itemize}

\vspace{0.5cm}
\textbf{Impact of Expenses on Retained Earnings:}

\begin{itemize}
    \item Higher expenses lower Net Income.
    \item Lower Net Income results in a lower addition to Retained Earnings.
    \item Thus, \textbf{Expenses indirectly reduce Retained Earnings}.
\end{itemize}

\vspace{0.5cm}
\begin{center}
\textbf{Summary Table}\\
\begin{tabular}{|c|c|c|}
\hline
\textbf{Event} & \textbf{Impact on SC} & \textbf{Impact on RE} \\
\hline
Additional Investment & Increase $(+)$ & No impact \\
Net Income & No impact & Increase $(+)$ \\
Dividends & No impact & Decrease $(-)$ \\
Revenue Earned & No direct impact & Increase (via Net Income) \\
Expenses Incurred & No direct impact & Decrease (via Net Income) \\
\hline
\end{tabular}
\end{center}

\vspace{1cm}
\clearpage

\section*{Question}

Prepare the Journal Entries for the following transactions:

\begin{itemize}
    \item Nov 4: John Smith, the major shareholder of real estate company, received 121,000 cash from an inheritance.
    \item Nov 5: Smith deposited 61,000 cash in a new business bank account titled Smith Real Estate Co. The business issued ordinary shares to Smith.
    \item Nov 6: The business paid 700 cash for letterhead stationery for the new office.
    \item Nov 7: The business purchased office equipment. The company paid cash of 13,500 and agreed to pay the account payable for the remainder, 7,500, within three months.
    \item Nov 10: Smith sold DLD shares, which he owned for several years, receiving 68,000 cash from his stockbroker.
    \item Nov 11: Smith deposited the 68,000 cash from sale of the DLD shares in his personal bank account.
    \item Nov 12: A representative of a large company telephoned Smith and told him of the company's intention to transfer 13,000 of business to Smith.
    \item Nov 18: Smith finished a real estate deal for a client and submitted his bill for services, 5,000. Smith expects to collect from the client within two weeks.
    \item Nov 21: The business paid half its account payable for the equipment purchased on November 7.
    \item Nov 25: The business paid office rent of 700.
    \item Nov 30: The business declared and paid a cash dividend of 2,000.
\end{itemize}

\section*{Answer}

\textbf{Journal Entries:}

\begin{center}
\begin{tabular}{|c|l|c|c|}
\hline
\textbf{Date} & \textbf{Account Titles and Explanation} & \textbf{Debit} & \textbf{Credit} \\
\hline
Nov 5 & Cash \newline Ordinary Share Capital & 61,000 \newline \quad & \quad \quad \quad 61,000 \\
\hline
Nov 6 & Office Supplies (Stationery) \newline Cash & 700 \newline \quad & \quad \quad \quad 700 \\
\hline
Nov 7 & Office Equipment \newline \quad Cash \newline \quad Accounts Payable & 21,000 \newline  & 13,500 \newline \quad 7,500 \quad \\
\hline
Nov 18 & Accounts Receivable \newline Service Revenue & 5,000 \newline \quad & \quad \quad \quad 5,000 \\
\hline
Nov 21 & Accounts Payable \newline Cash & 3,750 \newline \quad & \quad \quad \quad 3,750 \\
\hline
Nov 25 & Rent Expense \newline Cash & 700 \newline \quad & \quad \quad \quad 700 \\
\hline
Nov 30 & Dividends \newline Cash & 2,000 \newline \quad & \quad \quad \quad 2,000 \\
\hline
\end{tabular}
\end{center}

\vspace{0.5cm}

\textbf{Important Clarifications:}
\begin{itemize}
    \item \textbf{November 4 and November 10-11:} These involve John Smith’s personal transactions, hence no journal entry is needed for the business.
    \item \textbf{November 12:} Only a telephone conversation happened; no journal entry recorded.
\end{itemize}

\vspace{1cm}
\clearpage


\notesection{Journalize, Post to Ledger, and Prepare Trial Balance}{28-04-2025 Monday}

\textbf{Problem:}

S. Alam and Aziz Khan opened a web consulting business called \textbf{Money Laundering Services Pvt. Ltd.} and completed the following transactions:

\begin{itemize}
    \item June 1: Invested cash BDT 80,000 and office equipment BDT 26,000, issuing common stock.
    \item June 2: Prepaid BDT 9,000 rent.
    \item June 3: Purchased office equipment (BDT 8,000) and office supplies (BDT 3,600) on credit.
    \item June 6: Completed services for BDT 4,000 cash.
    \item June 9: Completed services for BDT 6,000 on credit.
    \item June 13: Paid BDT 11,600 cash to settle accounts payable.
    \item June 19: Paid BDT 2,400 cash for prepaid insurance.
    \item June 22: Received BDT 4,400 as partial payment from client.
    \item June 25: Completed services for BDT 2,890 on credit.
    \item June 28: Paid BDT 5,500 cash as dividends.
    \item June 29: Purchased BDT 600 office supplies on credit.
    \item June 30: Paid BDT 435 cash for utilities.
    \item June 30: Hired employee, salary of BDT 1,500 due in August (accrued salary).
\end{itemize}

\vspace{0.5cm}
\textbf{Journal Entries:}

\begin{center}
\begin{tabular}{|l|l|r|r|}
\hline
\textbf{Date} & \textbf{Account Titles and Explanation} & \textbf{Debit} & \textbf{Credit} \\
\hline
June 1 & Cash & 80,000 & \\
       & Office Equipment & 26,000 & \\
       & \quad Common Stock & & 106,000 \\
       & (Investment of cash and equipment for shares) & & \\
\hline
June 2 & Prepaid Rent & 9,000 & \\
       & \quad Cash & & 9,000 \\
       & (Payment for 12 months' rent) & & \\
\hline
June 3 & Office Equipment & 8,000 & \\
       & Office Supplies & 3,600 & \\
       & \quad Accounts Payable & & 11,600 \\
       & (Credit purchase of equipment and supplies) & & \\
\hline
June 6 & Cash & 4,000 & \\
       & \quad Service Revenue & & 4,000 \\
       & (Cash services completed) & & \\
\hline
June 9 & Accounts Receivable & 6,000 & \\
       & \quad Service Revenue & & 6,000 \\
       & (Services completed on account) & & \\
\hline
June 13 & Accounts Payable & 11,600 & \\
        & \quad Cash & & 11,600 \\
        & (Paid accounts payable) & & \\
\hline
June 19 & Prepaid Insurance & 2,400 & \\
        & \quad Cash & & 2,400 \\
        & (Payment for insurance policy) & & \\
\hline
June 22 & Cash & 4,400 & \\
        & \quad Accounts Receivable & & 4,400 \\
        & (Partial collection from client) & & \\
\hline
June 25 & Accounts Receivable & 2,890 & \\
        & \quad Service Revenue & & 2,890 \\
        & (Services completed on account) & & \\
\hline
June 28 & Dividends & 5,500 & \\
        & \quad Cash & & 5,500 \\
        & (Payment of dividends) & & \\
\hline
June 29 & Office Supplies & 600 & \\
        & \quad Accounts Payable & & 600 \\
        & (Purchase of additional office supplies) & & \\
\hline
June 30 & Utilities Expense & 435 & \\
        & \quad Cash & & 435 \\
        & (Paid for utilities) & & \\
\hline
June 30 & Salaries Expense & 1,500 & \\
        & \quad Salaries Payable & & 1,500 \\
        & (Accrued salary for June) & & \\
\hline
\end{tabular}
\end{center}

\vspace{0.5cm}
\textbf{Ledger Accounts:}

\vspace{0.3cm}
\textbf{Cash}
\[
\begin{array}{|l|l|}
\hline
\text{Debit} & \text{Credit} \\
\hline
80,000 & 9,000 \ (\text{Prepaid Rent}) \\
4,000  & 11,600 \ (\text{Payables}) \\
4,400  & 2,400 \ (\text{Prepaid Insurance}) \\
       & 5,500 \ (\text{Dividends}) \\
       & 435 \ (\text{Utilities}) \\
\hline
\end{array}
\]

\vspace{0.3cm}
\textbf{Common Stock}
\[
\begin{array}{|l|l|}
\hline
\text{Debit} & \text{Credit} \\
\hline
 & 106,000 \ (\text{Investment}) \\
\hline
\end{array}
\]

\vspace{0.3cm}
\textbf{Office Equipment}
\[
\begin{array}{|l|l|}
\hline
\text{Debit} & \text{Credit} \\
\hline
26,000 & \\
8,000  & \\
\hline
\end{array}
\]

\vspace{0.3cm}
\textbf{Office Supplies}
\[
\begin{array}{|l|l|}
\hline
\text{Debit} & \text{Credit} \\
\hline
3,600 & \\
600   & \\
\hline
\end{array}
\]

\vspace{0.3cm}
\textbf{Prepaid Rent}
\[
\begin{array}{|l|l|}
\hline
9,000 & \\
\hline
\end{array}
\]

\vspace{0.3cm}
\textbf{Prepaid Insurance}
\[
\begin{array}{|l|l|}
\hline
2,400 & \\
\hline
\end{array}
\]

\vspace{0.3cm}
\textbf{Accounts Payable}
\[
\begin{array}{|l|l|}
\hline
\text{Debit} & \text{Credit} \\
\hline
11,600\ (\text{Paid}) & 11,600\ (\text{Purchase}) \\
    & 600\ (\text{Supplies Purchase}) \\
\hline
\end{array}
\]

\vspace{0.3cm}
\textbf{Service Revenue}
\[
\begin{array}{|l|l|}
\hline
\text{Debit} & \text{Credit} \\
\hline
 & 4,000 \\
 & 6,000 \\
 & 2,890 \\
\hline
\end{array}
\]

\vspace{0.3cm}
\textbf{Accounts Receivable}
\[
\begin{array}{|l|l|}
\hline
\text{Debit} & \text{Credit} \\
\hline
6,000 & 4,400 \\
2,890 & \\
\hline
\end{array}
\]

\vspace{0.3cm}
\textbf{Dividends}
\[
\begin{array}{|l|l|}
\hline
5,500 & \\
\hline
\end{array}
\]

\vspace{0.3cm}
\textbf{Utilities Expense}
\[
\begin{array}{|l|l|}
\hline
435 & \\
\hline
\end{array}
\]

\vspace{0.3cm}
\textbf{Salaries Expense}
\[
\begin{array}{|l|l|}
\hline
1,500 & \\
\hline
\end{array}
\]

\vspace{0.3cm}
\textbf{Salaries Payable}
\[
\begin{array}{|l|l|}
\hline
 & 1,500 \\
\hline
\end{array}
\]

\vspace{0.5cm}
\textbf{Trial Balance as of June 30, 2025:}

\begin{center}
\begin{tabular}{|l|r|r|}
\hline
\textbf{Account Title} & \textbf{Debit} & \textbf{Credit} \\
\hline
Cash & 66,465 & \\
Accounts Receivable & 4,490 & \\
Office Supplies & 4,200 & \\
Prepaid Rent & 9,000 & \\
Prepaid Insurance & 2,400 & \\
Office Equipment & 34,000 & \\
Accounts Payable & & 600 \\
Salaries Payable & & 1,500 \\
Common Stock & & 106,000 \\
Dividends & 5,500 & \\
Service Revenue & & 12,890 \\
Utilities Expense & 435 & \\
Salaries Expense & 1,500 & \\
\hline
\textbf{Total} & \textbf{128,990} & \textbf{128,990} \\
\hline
\end{tabular}
\end{center}

\vspace{1cm}

\textbf{Common Techniques for Finding Errors in Trial Balance:}

\begin{enumerate}
    \item \textbf{Divisibility by 2 Technique:}
    \begin{itemize}
        \item If the difference between total debits and credits is divisible by 2, suspect that a debit has been mistakenly posted as credit or vice versa.
        \item \textbf{Example:} Difference = 540; 540 $\div$ 2 = 270; look for an entry of 270 posted to the wrong side.
    \end{itemize}

    \item \textbf{Divisibility by 9 Technique (Transposition Errors):}
    \begin{itemize}
        \item If the difference is divisible by 9, it may indicate a transposition error (e.g., writing 54 instead of 45).
        \item \textbf{Example:} Difference = 81; 81 $\div$ 9 = 9 (an integer); suspect swapped digits in entries.
    \end{itemize}
\end{enumerate}

\vspace{1cm}
\clearpage


\notesection{Basics of Accrual Accounting}{05-05-2025 Sunday}

\textbf{Accrual Accounting:}
\begin{itemize}
    \item Accrual accounting recognizes revenue when it is earned and expenses when they are incurred, regardless of when cash transactions occur.
    \item The main principle is to match revenues with related expenses in the period they occur, providing a more accurate picture of a company's financial position.
    \item \textbf{Example:} If a company completes a service in December but receives payment in January, the revenue will be recognized in December, not January.
\end{itemize}

\vspace{0.5cm}
\textbf{Key Characteristics of Accrual Accounting:}
\begin{itemize}
    \item \textbf{Matching Principle:} Revenues and expenses are matched to the period in which they occur.
    \item \textbf{Revenue Recognition Principle:} Revenue is recognized when earned, not when cash is received.
    \item \textbf{Expense Recognition Principle (or Matching Principle):} Expenses are recorded when they are incurred, regardless of when cash is paid.
    \item Provides a more accurate financial picture for decision-making.
\end{itemize}

\vspace{0.5cm}
\textbf{Different Accounting Standards:}

\begin{itemize}
    \item \textbf{International Accounting Standards (IAS):}
    \begin{itemize}
        \item A set of accounting standards used globally to standardize accounting practices and financial reporting.
        \item Provides the framework for reporting financial statements for companies operating across borders.
    \end{itemize}
    
    \item \textbf{International Accounting Standards Board (IASB):}
    \begin{itemize}
        \item An independent organization responsible for developing and promoting the adoption of IFRS (International Financial Reporting Standards).
    \end{itemize}
    
    \item \textbf{Generally Accepted Accounting Principles (GAAP):}
    \begin{itemize}
        \item A set of accounting principles, standards, and procedures used in the United States.
        \item Includes standards for recognizing revenue, matching expenses, and measuring financial position.
    \end{itemize}

    \item \textbf{International Financial Reporting Standards (IFRS):}
    \begin{itemize}
        \item A set of global accounting standards developed by the IASB to ensure consistency and transparency in financial reporting across different countries.
        \item Used by companies in over 100 countries worldwide.
    \end{itemize}
\end{itemize}

\vspace{0.5cm}
\textbf{Cash Basis vs Accrual Basis of Accounting:}

\begin{itemize}
    \item \textbf{Cash Basis Accounting:}
    \begin{itemize}
        \item Recognizes revenue when cash is received and expenses when cash is paid.
        \item Simple to use and typically used by smaller businesses or individuals.
        \item \textbf{Example:} If a company receives cash in January for services rendered in December, under cash basis accounting, the revenue is recognized in January.
    \end{itemize}
    
    \item \textbf{Accrual Basis Accounting:}
    \begin{itemize}
        \item Recognizes revenue when earned (not necessarily when cash is received) and expenses when incurred (not necessarily when cash is paid).
        \item Provides a more accurate picture of financial health, as it accounts for receivables and payables.
        \item \textbf{Example:} If a company completes a service in December but receives payment in January, the revenue is recognized in December under accrual accounting.
    \end{itemize}
    
    \item \textbf{Comparison:}
    \begin{itemize}
        \item \textbf{Cash Basis:} Simple, easier to implement, less comprehensive.
        \item \textbf{Accrual Basis:} More complex, but provides a better financial picture, required for larger businesses and public companies.
    \end{itemize}
\end{itemize}

\vspace{0.5cm}
\textbf{Key Differences between Cash and Accrual Basis:}

\begin{center}
\resizebox{\textwidth}{!}{
\begin{tabular}{|p{4.7cm}|p{5cm}|p{5.5cm}|}
\hline
\textbf{Feature} & \textbf{Cash Basis} & \textbf{Accrual Basis} \\
\hline
\textbf{Revenue Recognition} & When cash is received & When revenue is earned, regardless of cash receipt \\
\hline
\textbf{Expense Recognition} & When cash is paid & When expenses are incurred, regardless of cash payment \\
\hline
\textbf{Complexity} & Simple & More complex \\
\hline
\textbf{Accuracy in Financial Reporting} & Less accurate & More accurate \\
\hline
\textbf{Used By} & Small businesses, individuals & Larger businesses, corporations \\
\hline
\end{tabular}
}
\end{center}


\vspace{0.5cm}
\textbf{Conclusion:}

\textit{"While cash basis accounting is simpler and easier for small businesses, accrual accounting is the preferred method for larger entities and public companies due to its ability to provide a more comprehensive view of financial performance."}

\vspace{0.5cm} 

\textbf{Cash vs Accrual Basis:}

Below is a table demonstrating the difference between Cash Basis and Accrual Basis for various transactions. The first column lists the transactions, while the second and third columns indicate whether the event is recognized under Cash Basis or Accrual Basis.

\begin{center}
\begin{tabular}{|p{12cm}|p{2cm}|p{2cm}|}
\hline
\textbf{Transaction} & \textbf{Cash Basis} & \textbf{Accrual Basis} \\
\hline
Sold services worth 2000 USD on credit & No & Yes \\
\hline
Received cash for services rendered in previous month & Yes & Yes \\
\hline
Paid rent for the month & Yes & Yes \\
\hline
Purchased office supplies on credit & No & Yes \\
\hline
Received payment for previous month's credit sales & Yes & Yes \\
\hline
Paid salary to employees for the month & Yes & Yes \\
\hline
Completed a service but client will pay next month & No & Yes \\
\hline
Received advance payment for services not yet performed & Yes & Yes \\
\hline
Paid a supplier for goods purchased last month & Yes & Yes \\
\hline
Sold goods on credit for 5000 USD & No & Yes \\
\hline
Paid for utilities used in the month & Yes & Yes \\
\hline
\end{tabular}
\end{center}

\vspace{0.5cm}
\textbf{Time Period Concept and Revenue Recognition Principle (IFRS 15):}

\begin{itemize}
    \item \textbf{Time Period Concept:} This accounting concept asserts that financial transactions should be recorded in the period in which they occur, not when cash is exchanged. This ensures that the financial statements reflect the true financial position during that time.
    \item \textbf{Revenue Recognition Principle (IFRS 15):} Revenue should be recognized when control of the goods or services is transferred to the customer. This principle ensures that revenue is reported in the correct period, matching expenses incurred to earn the revenue.
\end{itemize}

\vspace{0.5cm}
\textbf{Five-Step Revenue Recognition Process (IFRS 15):}

The five steps provide a framework for recognizing revenue, as outlined in IFRS 15:

\begin{enumerate}
    \item \textbf{Identify the contract with a customer:} A contract must be enforceable and agreed upon by both parties.
    \item \textbf{Identify the performance obligations in the contract:} These are the promises to transfer goods or services to the customer.
    \item \textbf{Determine the transaction price:} This is the amount of consideration the company expects to receive for transferring goods or services.
    \item \textbf{Allocate the transaction price to the performance obligations:} If the contract has multiple performance obligations, allocate the transaction price to each based on relative standalone selling prices.
    \item \textbf{Recognize revenue when (or as) the performance obligation is satisfied:} Revenue is recognized when control of the good or service is transferred to the customer.
\end{enumerate}

\vspace{0.5cm}
\textbf{Example of Recognizing Revenue (IFRS 15):}

Assume a company signs a contract to provide a service worth 10,000 USD over a period of 5 months. The customer agrees to pay the full amount up front, but the service is provided monthly.

\begin{itemize}
    \item \textbf{Step 1:} Identify the contract — The agreement is a valid and enforceable contract between the company and the customer.
    \item \textbf{Step 2:} Identify the performance obligations — The service provided over 5 months is one performance obligation.
    \item \textbf{Step 3:} Determine the transaction price — The total transaction price is 10,000 USD.
    \item \textbf{Step 4:} Allocate the transaction price — The price is allocated evenly over the 5 months (2,000 USD per month).
    \item \textbf{Step 5:} Recognize revenue — The company recognizes 2,000 USD revenue each month as the service is provided.
\end{itemize}

\vspace{0.5cm}
\clearpage

\notesection{Adjusting Accounts, and Balance Sheet Recognition}{08-05-2025 Sunday}

\textbf{Additional Examples of Revenue Recognition:}

\begin{itemize}
    \item \textbf{Subscription Services:}
    \begin{itemize}
        \item A magazine company receives an upfront payment of \$240 for a one-year subscription. The company must recognize revenue as it delivers each issue of the magazine.
        \item \textbf{Cash Basis:} Revenue is recognized when cash is received (upfront payment of \$240).
        \item \textbf{Accrual Basis:} Revenue is recognized monthly as the magazine is delivered (recognize \$20 per month).
    \end{itemize}
    
    \item \textbf{Construction Contract:}
    \begin{itemize}
        \item A construction company is hired to build a building for \$1,000,000, with payments made as milestones are achieved.
        \item \textbf{Cash Basis:} Revenue is recognized when the milestone payment is received.
        \item \textbf{Accrual Basis:} Revenue is recognized as the work progresses, even if payments are received in advance or after milestones are completed.
    \end{itemize}

    \item \textbf{Retail Sales with Returns:}
    \begin{itemize}
        \item A retail company sells goods for \$100,000 in December, but allows returns for 30 days.
        \item \textbf{Cash Basis:} Revenue is recognized when the cash is received (December).
        \item \textbf{Accrual Basis:} Revenue is recognized in December but with an allowance for returns (e.g., 5\% of total sales as returns).
    \end{itemize}

    \item \textbf{Real Estate Sale:}
    \begin{itemize}
        \item A real estate company sells a property for \$500,000, and the buyer pays the full amount at the closing.
        \item \textbf{Cash Basis:} Revenue is recognized when the cash is received.
        \item \textbf{Accrual Basis:} Revenue is recognized at the closing date when control of the property is transferred, regardless of when the payment is received.
    \end{itemize}
\end{itemize}

\vspace{0.5cm}
\textbf{Adjusting the Accounts: Revenue \& Expense Recognition}

Adjusting entries are made at the end of an accounting period to ensure that revenue and expenses are recognized in the period in which they are incurred.

\begin{itemize}
    \item \textbf{Revenue Adjustments:}
    \begin{itemize}
        \item \textbf{Unearned Revenue (Advance Revenue):}
        \begin{itemize}
            \item A company receives \$10,000 in advance for services to be performed over the next 6 months.
            \item \textbf{Journal Entry:}
            \[
            \text{Debit: Cash} \quad 10,000, \quad \text{Credit: Unearned Revenue} \quad 10,000
            \]
            \item Each month, \$1,667 is recognized as revenue as the service is performed.
            \[
            \text{Debit: Unearned Revenue} \quad 1,667, \quad \text{Credit: Service Revenue} \quad 1,667
            \]
        \end{itemize}
        
        \item \textbf{Accrued Revenue (Earned but not yet received):}
        \begin{itemize}
            \item A company completes a project worth \$5,000 in December, but the client will pay in January.
            \item \textbf{Journal Entry:}
            \[
            \text{Debit: Accounts Receivable} \quad 5,000, \quad \text{Credit: Service Revenue} \quad 5,000
            \]
        \end{itemize}
    \end{itemize}

    \item \textbf{Expense Adjustments:}
    \begin{itemize}
        \item \textbf{Prepaid Expenses (Paid in Advance):}
        \begin{itemize}
            \item A company pays \$12,000 for 12 months of insurance coverage in advance.
            \item \textbf{Journal Entry (initial):}
            \[
            \text{Debit: Prepaid Insurance} \quad 12,000, \quad \text{Credit: Cash} \quad 12,000
            \]
            \item Each month, \$1,000 of insurance expense is recognized.
            \[
            \text{Debit: Insurance Expense} \quad 1,000, \quad \text{Credit: Prepaid Insurance} \quad 1,000
            \]
        \end{itemize}
        
        \item \textbf{Accrued Expenses (Incurred but not yet paid):}
        \begin{itemize}
            \item A company owes \$3,000 in wages for work done in December, to be paid in January.
            \item \textbf{Journal Entry:}
            \[
            \text{Debit: Wages Expense} \quad 3,000, \quad \text{Credit: Accrued Wages Payable} \quad 3,000
            \]
        \end{itemize}
    \end{itemize}
\end{itemize}

\vspace{0.5cm}
\textbf{Example of Complete Transactions and Balance Sheet Recognition:}

Let's assume a company engages in several transactions that affect its balance sheet at the end of the period.

\begin{itemize}
    \item \textbf{Transaction 1:} A company sells goods worth \$50,000 on credit.
    \begin{itemize}
        \item \textbf{Balance Sheet Effect:} 
        \begin{itemize}
            \item \text{Increase in Accounts Receivable (Asset):} \$50,000
            \item \text{Increase in Revenue (Equity via retained earnings):} \$50,000
        \end{itemize}
    \end{itemize}
    
    \item \textbf{Transaction 2:} The company receives \$30,000 in cash from the client.
    \begin{itemize}
        \item \textbf{Balance Sheet Effect:} 
        \begin{itemize}
            \item \text{Decrease in Accounts Receivable (Asset):} \$30,000
            \item \text{Increase in Cash (Asset):} \$30,000
        \end{itemize}
    \end{itemize}

    \item \textbf{Transaction 3:} The company pays \$5,000 for supplies purchased earlier on credit.
    \begin{itemize}
        \item \textbf{Balance Sheet Effect:} 
        \begin{itemize}
            \item \text{Decrease in Accounts Payable (Liability):} \$5,000
            \item \text{Decrease in Cash (Asset):} \$5,000
        \end{itemize}
    \end{itemize}

    \item \textbf{Transaction 4:} The company pays a utility bill of \$1,500 in cash.
    \begin{itemize}
        \item \textbf{Balance Sheet Effect:} 
        \begin{itemize}
            \item \text{Decrease in Cash (Asset):} \$1,500
            \item \text{Increase in Utilities Expense (Equity through retained earnings):} \$1,500
        \end{itemize}
    \end{itemize}
\end{itemize}

\vspace{0.5cm}
\textbf{Conclusion:}

\textit{"Accrual accounting provides a comprehensive and accurate picture of a company's financial health, ensuring revenues and expenses are recognized when they occur, not when cash changes hands."}

\vspace{1cm}

\textbf{Depreciation:}

Depreciation is the allocation of the cost of a tangible asset over its useful life. The purpose is to match the cost of the asset with the revenue it generates over time, in accordance with the matching principle in accrual accounting.

\begin{itemize}
    \item \textbf{Straight-Line Depreciation:}
    \begin{itemize}
        \item \textbf{Formula:} 
        \[
        \text{Depreciation Expense} = \frac{\text{Cost of Asset} - \text{Residual Value}}{\text{Useful Life}}
        \]
        \item Example: A machine costing \$10,000, with no residual value, and a useful life of 5 years, will have a yearly depreciation expense of:
        \[
        \frac{10,000}{5} = 2,000 \text{ per year}
        \]
    \end{itemize}

    \item \textbf{Double-Declining Balance Depreciation:}
    \begin{itemize}
        \item \textbf{Formula:} 
        \[
        \text{Depreciation Expense} = 2 \times \frac{1}{\text{Useful Life}} \times \text{Book Value at Beginning of Year}
        \]
        \item Example: A machine purchased for \$10,000 with a 5-year useful life. The first year depreciation is:
        \[
        2 \times \frac{1}{5} \times 10,000 = 4,000
        \]
    \end{itemize}
\end{itemize}

\vspace{0.5cm}
\textbf{Comprehensive Depreciation Example:}

Consider a piece of equipment purchased for \$150,000 with a useful life of 5 years and no residual value. Using the straight-line method, the depreciation expense for each year is:

\[
\text{Depreciation Expense} = \frac{150,000 - 0}{5} = 30,000 \text{ per year}
\]

Thus, the accumulated depreciation at the end of 2025 (after 1 year) is \$30,000, and the carrying amount of the equipment on the balance sheet is:

\[
\text{Carrying Amount} = 150,000 - 30,000 = 120,000
\]

\vspace{0.5cm}

\textbf{Depreciation Journal Entries:}

Depreciation is recorded periodically, typically at the end of each accounting year, to allocate the cost of an asset over its useful life.

\textbf{Example:}

Consider a company purchasing equipment for \$30,000, with an estimated useful life of 5 years and no residual value. The company uses the straight-line method for depreciation.

\textbf{Journal Entries for Depreciation:}

For the first 3 years, the journal entries for depreciation would be the same each year:

\begin{center}
\begin{tabular}{|l|l|r|r|}
\hline
\textbf{Date} & \textbf{Account Titles and Explanation} & \textbf{Debit} & \textbf{Credit} \\
\hline
\text{Year 1 (End of Year)} & Depreciation Expense & 6,000 & \\
                             & \quad Accumulated Depreciation—Equipment & & 6,000 \\
                             & (To record depreciation for the year) & & \\
\hline
\text{Year 2 (End of Year)} & Depreciation Expense & 6,000 & \\
                             & \quad Accumulated Depreciation—Equipment & & 6,000 \\
                             & (To record depreciation for the year) & & \\
\hline
\text{Year 3 (End of Year)} & Depreciation Expense & 6,000 & \\
                             & \quad Accumulated Depreciation—Equipment & & 6,000 \\
                             & (To record depreciation for the year) & & \\
\hline
\end{tabular}
\end{center}

\textbf{Total Depreciation after 3 Years:}  
\[
\text{Total Depreciation} = 6,000 \times 3 = 18,000
\]

Thus, after 3 years, the accumulated depreciation for the equipment would be \$18,000, and the book value of the equipment will be:

\[
\text{Book Value} = 30,000 - 18,000 = 12,000
\]

\vspace{0.5cm}

\textbf{Depreciation and Carrying Value for 3 Years:}

The following table presents the equipment's depreciation over 3 years and shows the accumulated depreciation each year, along with the final carrying value at the end of each year.

\vspace{0.5cm}
\begin{center}
\begin{tabular}{|l|r|r|r|}
\hline
\textbf{Particulars} & \textbf{Year 1 (BDT)} & \textbf{Year 2 (BDT)} & \textbf{Year 3 (BDT)} \\
\hline
\textbf{Equipment (Acquisition Cost)} & 30,000 & 30,000 & 30,000 \\
\hline
\textbf{Accumulated Depreciation—Equipment} & (6,000) & (12,000) & (18,000) \\
\hline
\textbf{Book Value / Carrying Value} & 24,000 & 18,000 & 12,000 \\
\hline
\end{tabular}
\end{center}

\vspace{0.5cm}
\textbf{Explanation:}
\begin{itemize}
    \item In \textbf{Year 1}, the company records depreciation of BDT 6,000. The carrying value at the end of Year 1 is 30,000 - 6,000 = BDT 24,000.
    \item In \textbf{Year 2}, an additional BDT 6,000 of depreciation is recorded, bringing the accumulated depreciation to BDT 12,000. The carrying value at the end of Year 2 is 30,000 - 12,000 = BDT 18,000.
    \item In \textbf{Year 3}, another BDT 6,000 is depreciated, bringing the accumulated depreciation to BDT 18,000. The carrying value at the end of Year 3 is 30,000 - 18,000 = BDT 12,000.
\end{itemize}


\vspace{1cm}
\textbf{Exceptions to Depreciation:}

There are certain assets that are **not depreciated** because they do not lose value over time, or their value is considered to be indefinite.

\begin{itemize}
    \item \textbf{Land:}
    \begin{itemize}
        \item Land is not depreciated because its useful life is considered indefinite. While land may appreciate in value over time, it doesn't wear out or get consumed in the same way as buildings or equipment.
        \item \textbf{Example:} A company purchases land for \$100,000. This land will not be depreciated.
    \end{itemize}
    
    \item \textbf{Intangible Assets with Indefinite Life:}
    \begin{itemize}
        \item Intangible assets like trademarks or goodwill with indefinite useful lives are not depreciated. Instead, these assets are tested for impairment periodically.
    \end{itemize}
    
    \item \textbf{Assets Under Construction:}
    \begin{itemize}
        \item Depreciation is not charged on assets that are under construction until they are completed and put into use.
    \end{itemize}
\end{itemize}

\vspace{0.5cm}
\clearpage

\textbf{Assignment Question (08-05-2025):}

Bashundhara City Shopping Complex Ltd. faced the following situations. Journalize the adjusting entry needed on December 31, 2025, for each situation. Consider each fact separately.

\begin{itemize}
    \item \textbf{a.} The business has an interest expense of 9,000 that it must pay early in January 2026.
    \item \textbf{b.} Interest revenue of 4,800 has been earned but not yet received.
    \item \textbf{c.} On July 1, 2025, 13,200 rent in advance was collected; Cash was debited, and Unearned Rent Revenue was credited. The tenant was paying two years' rent.
    \item \textbf{d.} Salary expense is 1,800 per day—Monday through Friday—and the business pays employees each Friday. This year, December 31 falls on Wednesday.
    \item \textbf{e.} The unadjusted balance of the Supplies account is 3,300. The total cost of supplies on hand is 1,200.
    \item \textbf{f.} Equipment was purchased at the beginning of this year at a cost of 150,000. The equipment's useful life is five years. There is no residual value. Record depreciation for this year and then determine the equipment's carrying amount.
\end{itemize}

\vspace{0.5cm}
\textbf{Adjusting Entries:}

\begin{center}
\begin{tabular}{|l|l|r|r|}
\hline
\textbf{Date} & \textbf{Account Titles and Explanation} & \textbf{Debit} & \textbf{Credit} \\
\hline
Dec 31 & Interest Expense & 9,000 & \\
       & Interest Payable & & 9,000 \\
       & (To accrue interest expense for the year) & & \\
\hline
Dec 31 & Interest Receivable & 4,800 & \\
       & Interest Revenue & & 4,800 \\
       & (To recognize earned interest revenue) & & \\
\hline
Dec 31 & Unearned Rent Revenue & 3,300 & \\
       & Rent Revenue & & 3,300 \\
       & (To recognize 6 months' rent revenue) & & \\
\hline
Dec 31 & Salary Expense & 5,400 & \\
       & Salaries Payable & & 5,400 \\
       & (To accrue salary expense for 3 days) & & \\
\hline
Dec 31 & Supplies Expense & 2,100 & \\
       & Supplies & & 2,100 \\
       & (To adjust supplies to the actual amount on hand) & & \\
\hline
Dec 31 & Depreciation Expense & 30,000 & \\
       & Accumulated Depreciation—Equipment & & 30,000 \\
       & (To record depreciation for the year) & & \\
\hline
\end{tabular}
\end{center}


\vspace{0.5cm}

\textbf{Determination of Carrying Amount of the Equipment}

\begin{center}
\begin{tabular}{|p{12cm}|p{3cm}|}
\hline
\textbf{Particulars} & \textbf{Amount} \\
\hline
A) Acquisition Cost & 150,000 \\
B) Accumulated Depreciation on December 31, 2025 & 30,000 \\
\hline
\textbf{Carrying Amount or Book Value of Equipment (A-B)} & \textbf{120,000} \\
\hline
\end{tabular}
\end{center}

\vspace{0.5cm}

\clearpage

\notesection{Adjusting Journal Entries for Real Madrid Training Academy (RMTA)}{12-05-2025 Monday}

\textbf{Question:}

Real Madrid Training Academy (RMTA) is a football school in the USA. On the next page is its unadjusted trial balance as of December 31, along with items (A) through (F) that require necessary year-end adjusting entries.

\textbf{Unadjusted Trial Balance as of December 31:}

\begin{center}
\begin{tabular}{|l|r|r|}
\hline
\textbf{Account Title} & \textbf{Debit} & \textbf{Credit} \\
\hline
Cash & 34,000 & \\
Accounts Receivable & 0 & \\
Supplies & 8,000 & \\
Prepaid Insurance & 12,000 & \\
Prepaid Rent & 3,000 & \\
Equipment & 115,000 & \\
Accumulated Depreciation—Equipment & & 25,000 \\
Accounts Payable & & 26,000 \\
Salaries Payable & & 0 \\
Unearned Service Revenue & & 12,500 \\
Share Capital & & 90,000 \\
Dividends & 50,000 & \\
Service Revenue & & 163,900 \\
Depreciation Expense—Professional Library & 0 & \\
Depreciation Expense—Equipment & 50,000 & \\
Insurance Expense & 33,000 & \\
Teaching Supplies Expense & 0 & \\
Advertising Expense & 6,000 & \\
Utilities Expense & 6,400 & \\
\hline
\textbf{Total} & \textbf{317,400} & \textbf{317,400} \\
\hline
\end{tabular}
\end{center}

\begin{itemize}
    \item A) An analysis of the insurance policies shows that \$2,400 of coverage has expired.
    \item B) An inventory count shows that supplies costing \$2,800 are available at year-end.
    \item C) Annual depreciation on the equipment is \$20,400.
    \item D) On September 1, RMTA agreed to train five kids for \$2,500 each. Training for two kids will start immediately and finish before the end of the year. Training for three other kids will not begin until next year. The client paid \$12,500 in advance for all five courses on September 1, and RMTA credited Unearned Service Revenue.
    \item E) RMTA’s two employees are paid weekly. At the end of the year, two days’ salaries have accrued at the rate of \$100 per day for each employee.
    \item F) The balance in the Prepaid Rent account represents rent for December.
\end{itemize}

Prepare the adjusting journal entries and determine the updated balances of the affected accounts.

\vspace{0.5cm}

\textbf{Solution:}

\textbf{Adjusting Journal Entries:}

\begin{center}
\begin{tabular}{|l|l|r|r|}
\hline
\textbf{Date} & \textbf{Account Titles and Explanation} & \textbf{Debit} & \textbf{Credit} \\
\hline
Dec 31 & Insurance Expense & 2,400 & \\
       & \quad Prepaid Insurance & & 2,400 \\
       & (To adjust prepaid insurance for expired coverage) & & \\
\hline
Dec 31 & Supplies Expense & 2,800 & \\
       & \quad Supplies & & 2,800 \\
       & (To adjust supplies inventory at year-end) & & \\
\hline
Dec 31 & Depreciation Expense—Equipment & 20,400 & \\
       & \quad Accumulated Depreciation—Equipment & & 20,400 \\
       & (To record annual depreciation on equipment) & & \\
\hline
Dec 31 & Unearned Service Revenue & 5,000 & \\
       & \quad Service Revenue & & 5,000 \\
       & (To recognize earned revenue from training two kids) & & \\
\hline
Dec 31 & Salaries Expense & 400 & \\
       & \quad Salaries Payable & & 400 \\
       & (To accrue salaries for 2 days' work for 2 employees) & & \\
\hline
Dec 31 & Rent Expense & 3,000 & \\
       & \quad Prepaid Rent & & 3,000 \\
       & (To record December's rental expense) & & \\
\hline
\end{tabular}
\end{center}

\vspace{0.5cm}

\textbf{Updated Ledger Accounts:}
\begin{multicols}{3}

\vspace{0.5cm}

\textbf{Cash:}
\[
\begin{array}{|l|l|}
\hline
\text{Debit} & \text{Credit} \\
\hline
34,000 & \\
\hline \hline 
34,000  &  \\
\hline
\end{array}
\]

\vspace{0.5cm}

\textbf{Accounts Receivable:}
\[
\begin{array}{|l|l|}
\hline
\text{Debit} & \text{Credit} \\
\hline
0 & 0 \\
\hline \hline
 & 0 \\
\hline
\end{array}
\]

\vspace{0.5cm}

\textbf{Supplies:}
\[
\begin{array}{|l|l|}
\hline
\text{Debit} & \text{Credit} \\
\hline
8,000 & 5,200 \\
\hline \hline 
2,800 &  \\
\hline
\end{array}
\]

\vspace{0.5cm}

\textbf{Prepaid Insurance:}
\[
\begin{array}{|l|l|}
\hline
\text{Debit} & \text{Credit} \\
\hline
12,000 & 2,400 \\
\hline \hline 
9,600 &  \\
\hline
\end{array}
\]

\vspace{0.5cm}

\textbf{Insurance Expense:}
\[
\begin{array}{|l|l|}
\hline
\text{Debit} & \text{Credit} \\
\hline
2,400 & \\
\hline \hline
 2,400 & \\
\hline
\end{array}
\]

\vspace{0.5cm}

\textbf{Prepaid Rent:}
\[
\begin{array}{|l|l|}
\hline
\text{Debit} & \text{Credit} \\
\hline
3,000 & 3,000 \\
\hline \hline 
 & 0 \\
\hline
\end{array}
\]

\vspace{0.5cm}

\textbf{Rent Expense:}
\[
\begin{array}{|l|l|}
\hline
\text{Debit} & \text{Credit} \\
\hline
33,000 & \\
3,000 & \\
\hline \hline 
36,000 & \\
\hline
\end{array}
\]

\vspace{0.5cm}

\textbf{Equipment:}
\[
\begin{array}{|l|l|}
\hline
\text{Debit} & \text{Credit} \\
\hline
115,000 & \\
\hline \hline 
115,000 & \\
\hline
\end{array}
\]

\vspace{0.5cm}

\textbf{Accumulated Depreciation—Equipment:}
\[
\begin{array}{|l|l|}
\hline
\text{Debit} & \text{Credit} \\
\hline
 & 25,000 \\
 & 20,400 \\
\hline \hline 
 & 45,400 \\
\hline
\end{array}
\]

\vspace{0.5cm}

\textbf{Accounts Payable:}
\[
\begin{array}{|l|l|}
\hline
\text{Debit} & \text{Credit} \\
\hline
 & 26,000 \\
\hline \hline
 & 26,000 \\
\hline
\end{array}
\]

\vspace{0.5cm}

\textbf{Salaries Payable:}
\[
\begin{array}{|l|l|}
\hline
\text{Debit} & \text{Credit} \\
\hline
 & 400 \\
\hline \hline 
 & 400 \\
\hline
\end{array}
\]

\vspace{0.5cm}

\textbf{Unearned Service Revenue:}
\[
\begin{array}{|l|l|}
\hline
\text{Debit} & \text{Credit} \\
\hline
5,000 & 12,500 \\
\hline \hline
 & 7,500 \\
\hline
\end{array}
\]

\vspace{0.5cm}

\textbf{Service Revenue:}
\[
\begin{array}{|l|l|}
\hline
\text{Debit} & \text{Credit} \\
\hline
 & 163,900 \\
 & 5,000 \\
\hline \hline 
 & 168,900 \\
\hline
\end{array}
\]

\vspace{0.5cm}

\textbf{Dividends:}
\[
\begin{array}{|l|l|}
\hline
\text{Debit} & \text{Credit} \\
\hline
50,000 & \\
\hline \hline 
50,000 & \\
\hline
\end{array}
\]

\vspace{0.5cm}

\textbf{Depreciation Expense —Equipment:}
\[
\begin{array}{|l|l|}
\hline
\text{Debit} & \text{Credit} \\
\hline
20,400 & \\
\hline \hline 
20,400 & \\
\hline
\end{array}
\]

\vspace{0.5cm}

\textbf{Teaching Supplies Expense:}
\[
\begin{array}{|l|l|}
\hline
\text{Debit} & \text{Credit} \\
\hline
5,200 & \\
\hline \hline 
5,200 & \\
\hline
\end{array}
\]

\vspace{0.5cm}

\textbf{Utilities Expense:}
\[
\begin{array}{|l|l|}
\hline
\text{Debit} & \text{Credit} \\
\hline
6,400 & \\
\hline \hline 
6,400  & \\
\hline
\end{array}
\]

\vspace{0.5cm}

\textbf{Salaries Expense:}
\[
\begin{array}{|l|l|}
\hline
\text{Debit} & \text{Credit} \\
\hline
50,000 & \\
400 & \\
\hline \hline 
50,400 & \\
\hline
\end{array}
\]

\end{multicols} 

\textbf{Corrected Adjusted Trial Balance as of December 31:}

\begin{center}
\begin{tabular}{|l|r|r|}
\hline
\textbf{Account Title} & \textbf{Debit} & \textbf{Credit} \\
\hline
Cash & 34,000 & \\
Accounts Receivable & 0 & \\
Supplies & 2,800 & \\
Prepaid Insurance & 9,600 & \\
Prepaid Rent & 0 & \\
Equipment & 115,000 & \\
Accumulated Depreciation—Equipment & & 45,400 \\
Accounts Payable & & 26,000 \\
Salaries Payable & & 400 \\
Unearned Service Revenue & & 7,500 \\
Share Capital & & 90,000 \\
Dividends & 50,000 & \\
Service Revenue &  & 168,900 \\
Depreciation Expense—Professional Library & 0 & \\
Depreciation Expense—Equipment & 20,400 & \\
Salaries Expense & 50,400 & \\
Insurance Expense & 2,400 & \\
Rent Expense & 36,000 & \\
Teaching Supplies Expense & 5,200 & \\
Advertising Expense & 6,000 & \\
Utilities Expense & 6,400 & \\
\hline
\textbf{Total} & \textbf{338,200} & \textbf{338,200} \\
\hline
\end{tabular}
\end{center}

\clearpage


\textbf{Rectification Entries Types with Examples}

Rectification entries are used to correct errors in the accounting books. Below are some common types of errors along with their corresponding rectification entries.

\vspace{0.5cm}

\textbf{1. Omission of Entries:}

\begin{multicols}{2}

\textbf{Case:} \\
RMTA forgot to record the payment of \$500 for utilities in December.

\textbf{Ideal Scenario:}
\[
\begin{array}{|l|r|r|}
\hline
\textbf{Account Title} & \textbf{Debit} & \textbf{Credit} \\
\hline
\text{Utilities Expense} & 500 & \\
\text{Cash} & & 500 \\
\hline
\end{array}
\]

\textbf{My Previous Action:}
\[
\begin{array}{|l|r|r|}
\hline
\textbf{Account Title} & \textbf{Debit} & \textbf{Credit} \\
\hline
 & & \\
\hline
\end{array}
\]

\textbf{Rectification Entry:}
\[
\begin{array}{|l|r|r|}
\hline
\textbf{Account Title} & \textbf{Debit} & \textbf{Credit} \\
\hline
\text{Utilities Expense} & 500 & \\
\text{Cash} & & 500 \\
\hline
\end{array}
\]
\end{multicols}
\vspace{0.5cm}

\textbf{2. Error in Commission:}
\begin{multicols}{2}
\textbf{Case:} \\
An amount of \$2,000 received from UPS was correctly recorded in the cash account but was wrongly posted to the \textbf{Service Revenue} Account instead of the \textbf{A/R} in the ledger.

\textbf{Ideal Scenario:}
\[
\begin{array}{|l|r|r|}
\hline
\textbf{Account Title} & \textbf{Debit} & \textbf{Credit} \\
\hline
\text{Cash} & 2,000 & \\
\text{Account Receivable} & & 2,000 \\
\hline
\end{array}
\]

\textbf{My Previous Action:}
\[
\begin{array}{|l|r|r|}
\hline
\textbf{Account Title} & \textbf{Debit} & \textbf{Credit} \\
\hline
\text{Cash} & 2,000 & \\
\text{Service Revenue} & & 2,000 \\
\hline
\end{array}
\]

\textbf{Rectification Entry:}
\[
\begin{array}{|l|r|r|}
\hline
\textbf{Account Title} & \textbf{Debit} & \textbf{Credit} \\
\hline
\text{Service Revenue} & 2,000 & \\
\text{Account Receivable} & & 2,000 \\
\hline
\end{array}
\]
\end{multicols}
\vspace{0.5cm}

\textbf{3. Error of Principle:}
\begin{multicols}{2}
\textbf{Case:} \\
\$500 purchase of vehicle was wrongly debited to \textbf{transportation expense}.

\textbf{Ideal Scenario:}
\[
\begin{array}{|l|r|r|}
\hline
\textbf{Account Title} & \textbf{Debit} & \textbf{Credit} \\
\hline
\text{Vehicle} & 500 & \\
\text{Cash} & & 500 \\
\hline
\end{array}
\]

\textbf{My Previous Action:}
\[
\begin{array}{|l|r|r|}
\hline
\textbf{Account Title} & \textbf{Debit} & \textbf{Credit} \\
\hline
\text{Transport Expense} & 500 & \\
\text{Cash} & & 500 \\
\hline
\end{array}
\]

\textbf{Rectification Entry:}
\[
\begin{array}{|l|r|r|}
\hline
\textbf{Account Title} & \textbf{Debit} & \textbf{Credit} \\
\hline
\text{Vehicle} & 500 & \\
\text{Transport Expense} & & 500 \\
\hline
\end{array}
\]
\end{multicols} 
\vspace{0.5cm}

\textbf{4. Error in Amount:}
\begin{multicols}{2}
    
    \textbf{Case:} \\
    The salary expense of \$3,000 was recorded as \$2,000.

    \textbf{Ideal Scenario:}
    \[
\begin{array}{|l|r|r|}
\hline
\textbf{Account Title} & \textbf{Debit} & \textbf{Credit} \\
\hline
\text{Salary Expense} & 3,000 & \\
\text{Cash} & & 3,000 \\
\hline
\end{array}
\]

\textbf{My Previous Action:}
\[
\begin{array}{|l|r|r|}
\hline
\textbf{Account Title} & \textbf{Debit} & \textbf{Credit} \\
\hline
\text{Salary Expense} & 2,000 & \\
\text{Cash} & & 2,000 \\
\hline
\end{array}
\]

\textbf{Rectification Entry:}
\[
\begin{array}{|l|r|r|}
\hline
\textbf{Account Title} & \textbf{Debit} & \textbf{Credit} \\
\hline
\text{Salary Expense} & 1,000 & \\
\text{Cash} & & 1,000 \\
\hline
\end{array}
\]
\end{multicols}

\vspace{0.5cm}

\textbf{5. Reversal of Entry:}
\begin{multicols}{2}
\textbf{Case:} \\
A \$600 credit purchase of cleaning supplies was wrongly recorded
as a service revenue for cleaning services on credit.

\textbf{Ideal Scenario:}
The correct entry should have been:
\[
\begin{array}{|l|r|r|}
\hline
\textbf{Account Title} & \textbf{Debit} & \textbf{Credit} \\
\hline
\text{Supplies} & 600 & \\
\text{Account Payable} & & 600 \\
\hline
\end{array}
\]

\textbf{My Previous Action:}
\[
\begin{array}{|l|r|r|}
\hline
\textbf{Account Title} & \textbf{Debit} & \textbf{Credit} \\
\hline
\text{Accounts Receivable} & 600 & \\
\text{Service Revenue} & & 600 \\
\hline
\end{array}
\]

\textbf{Rectification Entry:}
\[
\begin{array}{|l|r|r|}
\hline
\textbf{Account Titles and Explanation} & \textbf{Debit} & \textbf{Credit} \\
\hline
\text{Service Revenue} & 600 & \\
\text{Accounts Receivable} & & 600 \\
& & \\
\text{Supplies} & 600 & \\
\text{Accounts Payable} & & 600 \\
\hline
\end{array}
\]
\end{multicols}
\vspace{0.5cm}

\textbf{6. Alternative Treatment of Accruals:}
\textbf{Case:}
\begin{itemize}
    \item Supplies are purchased on June 1 for \$2,000 cash. The entire amount was shown as an expense.
    \item A physical count on June 30 indicates that it has \$1,600 of supplies available.
\end{itemize}


\textbf{Ideal Scenario:}
\[
\begin{array}{|l|r|r|}
\hline
\textbf{Account Title} & \textbf{Debit} & \textbf{Credit} \\
\hline
\text{Supplies} & 2,000 & \\
\text{Cash} & & 5,000 \\
& & \\
\text{Supplies Expense} & 400 & \\
\text{Supplies} & & 400 \\
\hline \hline 
\textbf{Net Impact:} & & \\
\text{Supplies} & 1,600 & \\
\text{Supplies Expense} & 400 & \\
\text{Cash} & & 2,000 \\
\hline 
\end{array}
\]

\begin{multicols}{2}
\textbf{My Previous Action:}
\[
\begin{array}{|l|r|r|}
\hline
\textbf{Account Title} & \textbf{Debit} & \textbf{Credit} \\
\hline
\text{Supplies Expense} & 2,000 & \\
\text{Cash} & & 2,000 \\
\hline
\end{array}
\]

\textbf{Rectification Entry:}
\[
\begin{array}{|l|r|r|}
\hline
\textbf{Account Title} & \textbf{Debit} & \textbf{Credit} \\
\hline
\text{Supplies} & 1,600 & \\
\text{Supplies Expense} & & 1,600 \\
\hline
\end{array}
\]

\end{multicols}
\vspace{0.5cm}

\clearpage

\section*{Problems:} 

\textbf{1. Case:}
\begin{itemize}
    \item The company purchased office furniture for \$2,500, but mistakenly recorded it as **Office Supplies** instead of **Office Equipment**.
    \item The transaction should have been classified as an asset, not an expense.
\end{itemize}
\begin{multicols}{2}
    
    \textbf{Ideal Scenario:}
    \[
    \begin{array}{|l|r|r|}
        \hline
        \textbf{Account Title} & \textbf{Debit} & \textbf{Credit} \\
        \hline
        \text{Office Equipment} & 2,500 & \\
        \text{Cash} & & 2,500 \\
        \hline
    \end{array}
    \]
    
    \textbf{My Previous Action:}
\[
\begin{array}{|l|r|r|}
\hline
\textbf{Account Title} & \textbf{Debit} & \textbf{Credit} \\
\hline
\text{Office Supplies Expense} & 2,500 & \\
\text{Cash} & & 2,500 \\
\hline
\end{array}
\]
\end{multicols}

\textbf{Rectification Entry:}
\[
\begin{array}{|l|r|r|}
\hline
\textbf{Account Title} & \textbf{Debit} & \textbf{Credit} \\
\hline
\text{Office Supplies Expense} & 2,500 & \\
\text{Office Equipment} & & 2,500 \\
\hline
\end{array}
\]

\vspace{0.5cm}

\textbf{2. Case:}
\begin{itemize}
    \item A sales transaction of \$1,200 was made on credit, but it was incorrectly recorded as **Cash Sales**.
    \item The transaction should have been recorded as **Accounts Receivable**.
\end{itemize}

\begin{multicols}{2}
    
    
    \textbf{Ideal Scenario:}
    \[
    \begin{array}{|l|r|r|}
\hline
\textbf{Account Title} & \textbf{Debit} & \textbf{Credit} \\
\hline
\text{Accounts Receivable} & 1,200 & \\
\text{Sales Revenue} & & 1,200 \\
\hline
\end{array}
\]

\textbf{My Previous Action:}
\[
\begin{array}{|l|r|r|}
    \hline
\textbf{Account Title} & \textbf{Debit} & \textbf{Credit} \\
\hline
\text{Cash} & 1,200 & \\
\text{Sales Revenue} & & 1,200 \\
\hline
\end{array}
\]
\end{multicols}

\textbf{Rectification Entry:}
\[
\begin{array}{|l|r|r|}
\hline
\textbf{Account Title} & \textbf{Debit} & \textbf{Credit} \\
\hline
\text{Accounts Receivable} & 1,200 & \\
\text{Cash} & & 1,200 \\
\hline
\end{array}
\]

\vspace{0.5cm}

\textbf{3. Case:}
\begin{itemize}
    \item A payment of \$300 was recorded in the wrong account. It was meant to be recorded as **Office Supplies** but was recorded as **Advertising Expense**.
    \item The correct classification should be **Office Supplies**.
\end{itemize}

\begin{multicols}{2}
\textbf{Ideal Scenario:}
\[
\begin{array}{|l|r|r|}
\hline
\textbf{Account Title} & \textbf{Debit} & \textbf{Credit} \\
\hline
\text{Office Supplies} & 300 & \\
\text{Cash} & & 300 \\
\hline
\end{array}
\]

\textbf{My Previous Action:}
\[
\begin{array}{|l|r|r|}
\hline
\textbf{Account Title} & \textbf{Debit} & \textbf{Credit} \\
\hline
\text{Advertising Expense} & 300 & \\
\text{Cash} & & 300 \\
\hline
\end{array}
\]
\end{multicols}

\textbf{Rectification Entry:}
\[
\begin{array}{|l|r|r|}
\hline
\textbf{Account Title} & \textbf{Debit} & \textbf{Credit} \\
\hline
\text{Advertising Expense} & 300 & \\
\text{Office Supplies} & & 300 \\
\hline
\end{array}
\]

\vspace{0.5cm}
\clearpage


% end of the note 


\end{document}
