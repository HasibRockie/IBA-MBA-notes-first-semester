\documentclass[12pt,a4paper]{book}

% Fonts & Typography — Elegant and Professional
\usepackage[T1]{fontenc}
\usepackage{kpfonts} % Sleek, modern font
\usepackage{microtype}

% Essential Packages
\usepackage{graphicx}
\usepackage{fancyhdr}
\usepackage{tocloft}
\usepackage{titlesec}
\usepackage{datetime}
\usepackage{hyperref}
\usepackage{geometry}
\usepackage{parskip}

% Page Geometry — Slim, Clean Margins
\geometry{
  a4paper,
  left=10mm,
  right=10mm,
  top=20mm,
  bottom=20mm
}

% Header & Footer Styling
\pagestyle{fancy}
\fancyhf{}
\fancyhead[L]{\small \textit{\nouppercase{\leftmark}}}
\fancyhead[R]{\small K501: Quantative Analysis for Business Decisions}
\fancyfoot[C]{\small \thepage}
\renewcommand{\headrulewidth}{0.3pt}
\renewcommand{\footrulewidth}{0.3pt}

% Chapter Title Styling
\titleformat{\chapter}[block]
  {\normalfont\Huge\bfseries}
  {\thechapter.}{12pt}{}

\titleformat{\section}
  {\normalfont\Large\bfseries}
  {\thesection}{1em}{}

% Table of Contents Styling
\renewcommand{\cftchapfont}{\bfseries}
\renewcommand{\cftsecfont}{}
\setlength{\cftbeforechapskip}{5pt}
\setlength{\cftbeforesecskip}{2pt}
\setlength{\cftaftertoctitleskip}{1em}

% Hyperlink Styling
\hypersetup{
    colorlinks=true,
    linkcolor=blue,
    urlcolor=blue,
    pdftitle={K501: Quantative Analysis for Business Decisions},
    pdfpagemode=FullScreen,
}

% Custom Command for Notes 
\newcommand{\notesection}[2]{
  \section*{#1\\ \small \textit{#2}}
  \phantomsection
  \addcontentsline{toc}{section}{#1 - #2}
} 

% Document Start 
\begin{document}

% Title Page
\begin{titlepage}
    \centering
    \vspace*{3.5cm}
    \includegraphics[width=0.28\textwidth]{logo.png}\par\vspace{1.5cm}
    {\scshape\LARGE University of Dhaka\par}
    \vspace{0.5cm}
    {\Large Institute of Business Administration (IBA)\par}
    \vspace{1.5cm}
    {\Huge\bfseries Master of Business Administration (MBA)\par}
    \vspace{1cm}
    {\Large K501: \textit{Quantative Analysis for Business Decision}\par}
    \vfill
    {\large Last Updated: \today\par}
\end{titlepage}

% Author Details Section 
\section*{Author Details}
\phantomsection
\addcontentsline{toc}{section}{Author Details}

\begin{center}
    \vspace{1em}
    \begin{tabular}{lll}
        \textbf{Name} & : & Md Hasibul Islam \\
        \textbf{Student ID} & : & 201-67-011 \\
        \textbf{Program} & : & Master of Business Administration (MBA) \\
        \textbf{Institute} & : & Institute of Business Administration (IBA) \\
        \textbf{University} & : & University of Dhaka \\
        \textbf{Email} & : & \href{mailto:hasiee8004@gmail.com}{hasiee8004@gmail.com} \\
        \textbf{LinkedIn} & : & \href{https://www.linkedin.com/in/hasib009}{linkedin.com/in/hasib009} \\
        \textbf{GitHub} & : & \href{https://github.com/HasibRockie}{github.com/HasibRockie} \\
        \textbf{Website} & : & \href{https://hasibrockie.github.io}{hasibrockie.github.io} \\
    \end{tabular}
    \vspace{1em}
\end{center}

\clearpage

% Table of Contents
\tableofcontents
\clearpage

% Notes Sections
\notesection{Statistics and Its Fundamental Concepts}{21-04-25 Monday}

\textbf{Statistics} is the science of collecting, organizing, analyzing, interpreting, and presenting data to support decision-making and problem-solving in business and other domains. It provides a quantitative foundation for managerial decisions by offering meaningful insights from data.

\vspace{0.5cm}
\textbf{Types of Statistics:}
\begin{enumerate}
    \item \textbf{Descriptive Statistics:} This involves methods of organizing, summarizing, and displaying data. 
    \begin{itemize}
        \item Examples: Mean, Median, Mode, Standard Deviation, Frequency tables, Pie charts, Histograms.
        \item Use Case: A retail manager summarizes last month's sales performance using a bar chart and average daily revenue.
    \end{itemize}

    \item \textbf{Inferential Statistics:} This refers to techniques for making generalizations from a sample to a population using probability theory.
    \begin{itemize}
        \item Examples: Hypothesis testing, Confidence intervals, Regression analysis.
        \item Use Case: A pharmaceutical company tests a new drug on a sample group to infer its effectiveness on the broader population.
    \end{itemize}
\end{enumerate}

\vspace{0.5cm}
\textbf{Types of Variables:}
\begin{itemize}
    \item \textbf{Qualitative (Categorical) Variables:} Represent categories or labels.
    \begin{itemize}
        \item Examples: Gender, Brand name, Type of customer (new/returning).
    \end{itemize}
    \item \textbf{Quantitative Variables:} Represent numeric values.
    \begin{itemize}
        \item \textbf{Discrete Variables:} Countable values (e.g., Number of employees).
        \item \textbf{Continuous Variables:} Measurable and can take any value within a range (e.g., Sales revenue, Temperature).
    \end{itemize}
\end{itemize}

\vspace{0.5cm}
\textbf{Levels of Measurement:}

\begin{table}[h!]
\centering
\resizebox{\textwidth}{!}{%
\begin{tabular}{|p{3cm}|p{5cm}|p{5cm}|p{5cm}|p{5cm}|}
\hline
\textbf{Characteristic} & \textbf{Nominal} & \textbf{Ordinal} & \textbf{Interval} & \textbf{Ratio} \\
\hline
\textbf{Definition} & 
Categorical data without any order & 
Categorical data with a logical order & 
Numeric data with equal intervals, no true zero & 
Numeric data with equal intervals and a true zero \\
\hline

\textbf{Nature of Data} & 
Labels or names & 
Ordered categories & 
Quantitative & 
Quantitative \\
\hline

\textbf{Mathematical Operations} & 
Equality only & 
Comparisons (>, <) & 
Addition, subtraction & 
All mathematical operations \\
\hline

\textbf{Meaningful Zero} & 
No & 
No & 
No & 
Yes \\
\hline

\textbf{Can Calculate Mean?} & 
No & 
No (median preferred) & 
Yes & 
Yes \\
\hline

\textbf{Examples} & 
Gender, Blood Type, Product Type & 
Socioeconomic Status, Education Level & 
Temperature (Celsius/Fahrenheit), IQ Score & 
Height, Weight, Age, Sales Revenue \\
\hline

\textbf{Applicable Statistics} & 
Mode, Frequency & 
Mode, Median, Percentile & 
Mean, SD, Correlation & 
All descriptive and inferential statistics \\
\hline

\textbf{Distance between values is meaningful?} & 
No & 
Not always & 
Yes & 
Yes \\
\hline

\textbf{Has absolute zero?} & 
No & 
No & 
No & 
Yes \\
\hline
\end{tabular}%
}
\caption{Comparison of Levels of Measurement}
\end{table}


\vspace{0.5cm}
\textbf{Other Key Concepts:}
\begin{itemize}
    \item \textbf{Population:} The entire group of individuals or instances about whom we hope to learn.
    \item \textbf{Sample:} A subset of the population, selected for analysis.
    \item \textbf{Parameter:} A numerical summary or measure that describes a characteristic of a population (e.g., population mean $\mu$).
    \item \textbf{Statistic:} A numerical summary derived from a sample (e.g., sample mean $\bar{x}$). Statistics are used to estimate parameters.
\end{itemize}

\vspace{2cm}

% new day 
\notesection{Classification of Variables with Examples}{28-04-25 Monday}

\textbf{Table 2: Classification of Variables — Qualitative vs Quantitative, Discrete vs Continuous}

\vspace{0.5cm}

\begin{table}[h!]
\centering
\begin{tabular}{|p{3cm}|p{7.5cm}|p{5.5cm}|}
\hline
\textbf{Variable Type} & \textbf{Discrete Examples} & \textbf{Continuous Examples} \\
\hline
\textbf{Qualitative } &
Shirt size (S, M, L) \newline
Product category (A, B, C) \newline
Number of children category (None, One, Two+) \newline
Room type (Single, Double) \newline
Education level (High School, UG, PG)
&
Skin tone spectrum \newline
Customer feedback scale \newline
Dialect variation \newline
Shade of color preferences \newline
Accent variation \\
\hline
\textbf{Quantitative} &
Number of cars owned \newline
Number of transactions \newline
Exam scores (out of 100) \newline
Number of employees \newline
Number of visits
&
Height (cm) \newline
Weight (kg) \newline
Income (\$) \newline
Temperature (°C) \newline
Time spent (hours) \\
\hline
\end{tabular}
\caption{Classification of Variables: Qualitative vs Quantitative and Discrete vs Continuous}
\end{table}
    
\vspace{1cm}

\textbf{Table 3: Classification by Levels of Measurement — Nominal, Ordinal, Interval, Ratio with Discrete and Continuous Types}

\vspace{0.3cm}

\begin{table}[h!]
\centering

\begin{tabular}{|p{3cm}|p{6cm}|p{6cm}|}
\hline
\textbf{Level of Measurement} & \textbf{Discrete Examples} & \textbf{Continuous Examples} \\
\hline
\textbf{Nominal} &
\begin{itemize}
    \item Jersey number
    \item Postal code
    \item Nationality
    \item Product ID
    \item Car model
\end{itemize}
&
\begin{itemize}
    \item Color shade
    \item Accent pattern
    \item Logo design variation
    \item Pattern of speech
    \item Ink density
\end{itemize}
\\
\hline
\textbf{Ordinal} &
\begin{itemize}
    \item Customer rating (1–5 stars)
    \item Survey rank (Strongly disagree to Agree)
    \item Academic grade (A, B, C)
    \item Job level (Junior, Mid, Senior)
    \item Market tier (Low, Mid, High)
\end{itemize}
&
\begin{itemize}
    \item Satisfaction level on 0–10 scale
    \item Credit score bands
    \item Health condition severity
    \item Employee performance level
    \item Risk tolerance scale
\end{itemize}
\\
\hline
\textbf{Interval} &
\begin{itemize}
    \item Test scores (e.g., IQ, SAT)
    \item Temperature recorded hourly
    \item Credit scores in discrete brackets
    \item Year of birth
    \item Calendar dates
\end{itemize}
&
\begin{itemize}
    \item Temperature (°C or °F)
    \item Time of day (without AM/PM)
    \item Financial index points
    \item Sound intensity
    \item Wind speed variation
\end{itemize}
\\
\hline
\textbf{Ratio} &
\begin{itemize}
    \item Number of products sold
    \item Number of goals scored
    \item Number of books owned
    \item Defect counts in production
    \item Visitors per day
\end{itemize}
&
\begin{itemize}
    \item Income
    \item Distance traveled
    \item Weight
    \item Time
    \item Age
\end{itemize}
\\
\hline
\end{tabular}

\caption{50 Examples Categorized by Level of Measurement and Variable Type}
\end{table}

\vspace{1cm}
\clearpage




% new note 
\notesection{2025-04-23}{Wednesday}
\textbf{Topics Covered:}

\clearpage

% end of the note 


\end{document}
